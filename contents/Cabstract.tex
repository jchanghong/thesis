% !Mode:: "TeX:UTF-8"

\begin{Cabstract}{分布式数据库}{关系数据库}{非关系型数据库}{数据安全}{负载均衡}
关系数据库系统自从上世界被发明以来,就成为了企业存储数据的首选数据库系统。但是在移动互联网时代,大量非结构化数据的存储和处理并不是关系数据库的强项。针对关系数据库系统的缺点,已经出现了各种各样的非关系型数据库,比如,键值数据库用来存储关系最简单的数据;文档数据库用来存储大量图片或者文档数据;图形数据库系统用来存储用户关联数据。
每一种数据库系统都有其用武之地,但是企业用各种不同类型的数据库来存储不同类型的数据,导致了服务器资源的浪费,增加了数据库系统的维护成本,所以,设计和实现一种能存储和管理多种数据类型的数据库管理系统具有重要的意义。

当把所有数据都存储在一个数据库系统的时候,必须保证系统能够部署在分布式环境下以扩展其读写能力,还需要保证系统的数据安全。基于以上需求,本文设计和实现了一个分布式数据库系统JSQL,其结合关系型数据库和非关系数据库系统的优点,能存储结构化和非结构化数据类型,支持多主分布式架构 ,能够动态扩展读写能力,还从底层加入审计功能以保证数据的安全。

在分析系统需求和相关技术以后,论文首先对系统的功能和架构进行阐述,
JSQL采用常用的客服端-服务器架构,分为客服端和服务器端,
服务器端又分为分布式管理节点和分布式数据库集群。
分布式管理节点实现了分布式数据库系统的负载均衡功能和监控管理功能,分布式数据库集群实现数据的存储和管理。其次,详细设计和实现了客户端和分布式管理节点中的负载均衡模块、数据审计模块和分布式数据库集群节点中的数据库功能模块、分布式集群架构模块和审计日志存储功能等模块。为了均衡服务器资源,论文设计和实现了新的动态负载均衡算法,为了解决集群节点间数据一致性问题,论文提出利用分布式队列实现的新的分布式架构。再次,对分布式数据库的结构化存储功能和审计功能等进行了测试,给出测试结果并进行分析。最后,总结本论文所述系统的不足之处和论文后续工作,展望数据库系统的未来发展。
\end{Cabstract}