% !Mode:: "TeX:UTF-8"

\begin{Cabstract}{分布式数据库}{mysql}{安全审计}{OLTP}{NOSQL}
数据库、操作系统和编译器并称为三大系统,可以说是整个计算机软件的基石。其中数据库更靠近应用层,是很多业务的支撑。这一领域经过了几十年的发展,
不断的有新的进展。从最开始的层次数据库和关系数据库,到近几年火热的Nosql数据库,再到最近以Google Spanner和F1为代表的NewSql数据库。

在互联网时代,海量数据的存储与访问成为系统设计与使用的瓶颈问题,对于海量数据处理,按照使用场
景,主要分为两种类型:联机事务处理(OLTP)和联机分析处理(OLAP)。

关系型数据库,是建立在关系模型基础上的数据库,其借助于集合代数等数学概念和方法来处理数据库中的
数据。其中以MySql最为流行,MySql是一个开源的关系型数据库,其优势在于源代码开放,任何企业和个人都可以根据自身的需求对mysql的源码做修改。

NoSQL 数据库,全称为 Not Only SQL,意思就是适用关系型数据库的时候就使用关系型数据库,不适用的
时候也没有必要非使用关系型数据库不可,可以考虑使用更加合适的数据存储。

Oracle,mysql 等传统的关系数据库非常成熟并且已大规模商用,为什么还要用 NoSQL 数据库呢?主要是
由于随着互联网发展,数据量越来越大,对性能要求越来越高,传统数据库存在着先天性的缺陷,即单机(单
库)性能瓶颈,并且扩展困难。这样既有单机单库瓶颈,却又扩展困难,自然无法满足日益增长的海量数据存储
及其性能要求,所以才会出现了各种不同的NoSQL产品。

虽然在云计算时代,传统数据库存在着先天性的弊端,但是 NoSQL 数据库又无法将其替代,NoSQL 只能作
为传统数据的补充而不能将其替代,所以规避传统数据库的缺点是目前大数据时代必须要解决的问题。

为了解决现在MySql等关系型数据库的这些问题,本文描述了如何设计和实现一个兼容MySql协议的分布式数据库。他能自动的发现分布式数据库集群中的节点,
自动的分配数据,支持海量的数据存储。同时考虑到如今数据库的安全变得越来越重要,不时的发生数据库管理员或者其他攻击者恶意更改数据库中的数据的情况,
所以本文开发的数据库从底层加入数据库的审计功能,监控所有数据库的更改情况,而且可以以图形化的方式显示监控审计结果。

本论文涉及的主要工作如下:

\hspace{8mm}基于java语言实现的兼容mysql通信协议的数据库;

\hspace{8mm}实现了数据库集群,实现了数据库的高可用性,可扩展性,负载均衡等特性;

\hspace{8mm}从数据库底层考虑了数据库安全,加入了数据审计等特性。
\end{Cabstract}
