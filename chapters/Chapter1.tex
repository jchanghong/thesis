% !Mode:: "TeX:UTF-8"

\chapter{绪论}
\section{研究背景和研究意义}
互联网诞生以来在全球迅速蔓延。
2017年我国工业和信息化部最新发布的通信业经济运行情况显示,
2月末,我国移动电话用户总数达到13.3亿户,移动互联网用户总数达到11.2亿户,使用手机上网的用户数接近10.6亿户。
互联网用户数量还有很大的增长空间,特别是亚洲人口众多的发展中国家。同时,智能手机的革命发展
通过移动互联网大大提升用户体验,使移动互联网迅速发展。伴随着互联网的发展
出现了各种基于互联网的应用服务,从传统媒体门户网站
到BBS以及近年来社会媒体的兴起,电子商务的巨大发展,还有各种各样的移动互联网应用程序。

随着互联网和互联网应用的发展,数据存储的需求不断增长。
IDC报告显示,预计到2020年全球数据总量将超过40ZB,
这一数据量是2011年的22倍。在过去几年,全球的数据量以每年百分之58的速度增长,
在未来这个速度会更快。如果按照现在存储容量每年百分之40的增长速度计算,
到2018年需要存储的数据量甚至会大于存储设备的总容量。

未来是“大数据”时代,这么大的存储需求,给数据存储技术
的发展带来了很大的压力。有很多应用基于互联网提供的各种服务正在进入井喷
时代的发展,而这些应用,背景的很大一部分面临同样的问题:怎么样
以尽可能廉价的方式实现大规模数据存储和查询。如搜索服务,需要存储页面
而分析,微博等社交网络需求的实时用户来表达查询的观点,在线旅游
需要玩家的信息和操作来存储和查询,银行需要用户帐户信息和用户
用于实时存储和查询。这种应用所面临的挑战总结为大数据可用性和可靠性要求
高,部分应用需要实时查询响应和高并发性和数据一致性要求。在这样的
环境下,分布式数据库近年来也取得了飞速发展。分布式数据库是一种数据库技术
和网络技术结合的产品,相对于单节点数据库,分布式数据库容量和可用性
具有很大的优势,因此能够更好的应对大规模和超大规模的数据存储需求的应用。
分布式数据库系统通常使用大量廉价,独立的计算机系统构建,
经济和性能上,可以满足互联网应用程序的对数据大小,可用性和性能的
需求。
\section{国内外研究历史与现状}
数据库技术的发展始于20世纪60年代。在没有数据库的情况下,使用计算机
存储数据的用户(主要是财务科研单位)以操作系统中的文件的形式存储数据
。随着业务的发展,应用程序变得越来越复杂,人们开始需要开发管理数据
在通用软件,这促成了数据库的诞生。 20世纪60年代以来,人们探索数据模型
实现数据库,后来开发出三大数据库模型:层次数据模型,网格数据
模型,关系数据模型,后来也出现在对象数据模型中。其中,可以使用关系数据模型
严格的数学理论来描述数据库的组织和操作,具有简单灵活,数据库独立性高的特点
特征。从20世纪70年代到90年代,关系数据库理论成熟并得到广泛应用,
这个里程碑是1974年,IBM的圣荷西,加利福尼亚研究实验室的D.D.
Chamberlin和Ray Boyce开发了SEQUEL结
结构化查询语言,后来在1980年更名为SQL。 SQL是数据库中的标准数据查询
语言,在1986年,由美国国家标准学会规范SQL,就这样成为了
数据库系统的标准语言。 SQL包含三个部分:数据定义
语言,数据操作语言,数据控制语言。 SQL是一种全面的通用关系数据语言
,可以当它是一种高级的非程序语言,它允许用户在高级数据结构中工作。使用SQL
用户无需知道数据的具体存储方式。在SQL中一个简单的语句
实现效果,使用其他编程语言需要很大一部分程序才能实现。另外,SQL通过
使用这种语言允许用户掌握这种语言就可以使用这种语言来操作任何一个标准数据库
产品。到20世纪90年代,关系数据库标准几乎适用于任何数据存储需求
的应用程序。

分布式数据库系统是数据库系统技术与网络技术的结合。分布式数据库系统(DDBS)的研究始于20世纪70年代。但是,
分布式数据库的理论与应用成为一个热门话题,是20世纪90年代的事情。
 90年代,互联网
网络出现爆炸式增长,同时各种应用程序对存储的需求与日俱增。在这样的环境下
,分布式数据库系统的研究成为了热门话题,人们探索分布式数据库系统
理论和关键技术,并快速应用理论实践,开发了各种商业应用价值
的数据库产品。 2002年,Eric Brewer提出了引导分布式数据库研究的重要理论,
并且后来证明是正确的,这个理论就是CAP定理,CAP定理的核心观点是:
在分布式计算系统中,不可能同时满足以下三点:一致性,
可用性,分区容忍性。 CAP理论认为
分布式数据库产品的设计必须介于三者之间,其中至少有两个可以满足,不可能
同时满足三个。根据CAP理论的指导,有学者认为,基于传统的关系数据库
构建分布式数据库,很难满足当前大型数据存储需求的各类互联网应用,
如搜索引擎,社交网络等。由于传统的关系数据库非常重视数据一致性,
在传统的关系数据库中,交易的四点必须保证要求:ACID(Atomicity,
一致性,隔离,持久性)。有部分人认为,
根据CAP理论,在满足强大的一致性之后,也希望获得互联网应用的高可用性
性是非常困难的。因此,在二十世纪九十年代末和二十一世纪初,互联网应用大幅增长
出现了一些人认为是革命性的分布式数据库概念:NoSQL(not only SQL)。
NoSQL和传统的关系数据库非常的不同,对于一个概念,NoSQL的显着特点是:
非关系型,
分布式,不提供ACID数据库设计模式。 NoSQL不支持复杂的关系操作,
不提供对交易一致性的支持。由于摆脱了必须满足支持一致性
的功能要求,根据NoSQL概念设计产品生成的高可扩展性高并发高可用性
性支持得到了很大的改善,各种类似的开源或商业的NoSQL产品出现了,并且被大量和
被各种互联网应用程序所使用。这些NoSQL产品被应用到新的产品,如博客,论坛,
微博等
其他社交服务。这些应用程序具有高度可扩展性,高可用性
,并发性。高可用性这些功能很重要,因为这些应用压力主要来自大规模数据
存储,大量并发访问及时响应,
NoSQL拒绝一致性支持是合理的,因为像社交网络应用程序一样
用户丢失数据库的后果不会太严重,
毕竟,这不会导致用户遭受巨大的损失,如经济利益。作为分布式数据库
,NoSQL在过去两年爆发式增长,均受益于互联网
高速发展,也是各种应用到“大数据”的发展趋势,CAP理论再一次证明了它的正确性。

毕竟,NoSQL放弃了一致性的支持,这些产品可以应用到请求的一致性不高
在应用上,随着互联网的快速发展和“大数据”时代的到来,那些需要关系操作,
需要高级一致性并且面临大规模数据存储查询要求的应用,比如
电子商务,这样的应用程序不能使用NoSQL产品。所以,目前的分布式数据库开发
方向,根据CAP理论,一个是尽可能的减少一致性,提高可用性,
另一个方向是需要尝试确保操作之间的关系,一致性的支持,同时利用分布式技术的优势
,提供尽量高性能的服务。
\section{论文的主要工作}
通过对分布式理论、关系型数据库理论及相关技术的学习和研究,论文基于Java语言实现了一个分布式的关系型数据库JSQL
,jsql是一个兼容mysqk通信协议的分布式数据库。论文的主要工作包括:
\begin{enumerate}
	\item 利用orientdb开源项目设计和实现了分布式数据库的本地存储。
	
	\item 实现了mysql通信协议,使得可以通过mysql客服端来连接jsql分布式数据库,
	方便mysql用户迁移到本数据库系统。
	\item 实现了对sql语句的解析和执行。
	
	\item 实现了数据库的分布式架构,使得数据可以存储在多台计算机上面。
	
	\item 实现了系统的审计系统,使得系统的安全性得到增强。
\end{enumerate}
\section{本论文的结构安排}
第一章作为本论文的绪论,首先介绍了论文的选题背景和本论文进行研究的意
义。接着阐述了数据库的发展历史和现状。最后一节里介绍了论文的主
要工作。

第二章,主要介绍了本论文相关的理论基础和相关的技术,首先给数据库分类,给出数据库相关的概念,
然后对分布式数据库相关的技术进行了介绍,主要包括负载均衡技术,数据分片技术以及数据库高可用技术。
本章最后给出了mysql的体系结构,作为本系统的参考架构。

第三章,作为论文的需求分析,本章给出了分布式数据库jsql的实现目标,同时也
包括对通信协议的实现和编程模式的分析。

第四章,是系统的设计,包括总体设计和模块划分,本章对系统的总体架构和各个模块的详细架构给出了阐述。
系统主要包括数据库模块,集群架构,已经数据审计功能模块。

第五章,是关于系统的具体实现,本章分别从数据库模块,集群架构模块,数据库审计模块详细说明了每个模块的实现。
接着就每个关键模块的实现进行了讲解。最后对系统主要功能的流程进行讲解。

第六章里,论文对jsq分布式数据库进行测试,包括功能测试和性能测试。

最后部分,是关于致谢和参考资料。
\section{本章小结}
本章首先阐述了论文的背景,并表明本论文的研究对于该领域的发展具有非常重要的意义
。 然后介绍本课题的国内外研究历史和发展现状。 最后介绍了本文的主要工作和章节
部分安排。