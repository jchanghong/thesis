% !Mode:: "TeX:UTF-8"

\chapter{绪论}
在互联网时代,海量数据的存储与访问成为系统设计与使用的瓶颈问题,对于海量数据处理,按照使用场
景,主要分为两种类型:联机事务处理(OLTP)和联机分析处理(OLAP)。

联机事务处理(OLTP)也称为面向交易的处理系统,其基本特征是原始数据可以立即传送到计算中心进行处
理,并在很短的时间内给出处理结果。

联机分析处理(OLAP)是指通过多维的方式对数据进行分析、查询和报表,可以同数据挖掘工具、统计分析
工具配合使用,增强决策分析功能。
\section{研究工作的背景与意义}
web2.0的到来,存储海量数据的需求越来越大。

在互联网时代,海量数据的存储与访问成为系统设计与使用的瓶颈问题,对于海量数据处理,按照使用场
景,主要分为两种类型:联机事务处理(OLTP)和联机分析处理(OLAP)。

联机事务处理(OLTP)也称为面向交易的处理系统,其基本特征是原始数据可以立即传送到计算中心进行处
理,并在很短的时间内给出处理结果。

联机分析处理(OLAP)是指通过多维的方式对数据进行分析、查询和报表,可以同数据挖掘工具、统计分析
工具配合使用,增强决策分析功能。
\section{国内外研究历史与现状}
google spanner和t1等系统。
在互联网时代,海量数据的存储与访问成为系统设计与使用的瓶颈问题,对于海量数据处理,按照使用场
景,主要分为两种类型:联机事务处理(OLTP)和联机分析处理(OLAP)。

联机事务处理(OLTP)也称为面向交易的处理系统,其基本特征是原始数据可以立即传送到计算中心进行处
理,并在很短的时间内给出处理结果。

联机分析处理(OLAP)是指通过多维的方式对数据进行分析、查询和报表,可以同数据挖掘工具、统计分析
工具配合使用,增强决策分析功能。
\section{论文的主要工作}

结合nosql的关系型数据库的特点,同时从数据库自身考虑数据库安全。
...
\section{本论文的结构安排}
本论文后续各章结构如下:

第二章介绍了

第三章介绍了

第四章描述了审计子系统是如何实现,同时将涉及的技术方法作了介绍和分
析。

第五章对审计管理器模块进行了分析和设计。首先提出该模块的概述,并结
合这些目标对系统总体结构和重要的功能进行了详细的分析和设计,从而保证设
计目标的实现。

第六章主要通过介绍审计子系统的具体应用,说明了审计子系统的实用价值
和地位。

结论部分详细总结了本论文的研究工作,并针对存在的主要不足指出了改进
的重点和今后的方向。
\section{本章要求}
...