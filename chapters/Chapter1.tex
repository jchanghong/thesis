% !Mode:: "TeX:UTF-8"

\chapter{绪论}
\section{研究背景和研究意义}
近年来,移动互联网经过了快速的发展,安卓和苹果等智能手机的普及让移动互联网用户快速增长。
随着无线网络的发展,智能手机的体验也越来越好,对手机移动应用程序的需求也与日俱增。
根据我国最近的调查数据显示,我国移动电话用户的数量已经达到13亿,智能手机用户则达到了11亿。
而且随着网络技术和移动电话技术的发展,这些数据还会继续增长,智能手机迟早会取代功能手机,甚至在很多场景下取代传统的计算机。
从全世界来看,还有很多非发展中国家的智能手机用户数量占比还很低,发展的空间也更大。
智能手机的出现,导致出现了各种各样的移动互联网应用程序,比如微信、QQ等聊天应用,腾讯视频等视频应用。
大量移动用户使得这些移动应用程序对存储的需求变着越来越大,这对数据库系统提出了新的要求。

智能手机移动互联网应用的快速发展,伴随着应用程序数据量的快速增长,这就需要存储设备和软件来存储。
现在主流大型应用程序的数据都是存储在数据库系统当中的,这方便数据在不同的应用之间共享,方便对数据进行统一的管理。
在移动搜索应用程序中,需要利用数据库来存储各种网页和日志数据,对其进行存储和分析;
在微信等聊天应用程序中,需要用数据库来存储用户和聊天数据等信息,而且需要对其进行加密和其他处理来满足用户对聊天应用的需求;美团等移动应用程序则需要利用数据库存储全国各地的店铺信息和各种交易信息。
随着智能手机用户数量的快速增长,利用传统关系型数据库来存储这些数据已经出现困难。
传统关系型数据库不方便在集群中部署,从而不方便对数据库存储系统进行扩展。

为了解决传统关系型数据的问题,出现了各种各样的非关系型数据库系统。
在非关系型数据库中,数据库不再满足强一致性和事务要求,更加重视互联网集群的可扩展性和性能的需求,更加适合大量移动应用程序数据存储场景,特别是非关系型数据的存储,比如各种移动应用程序日志等数据存储。
在这几年里,非关系型数据库系统得到了非常大的发展,出现了各种各样的非关系型数据库系统。在企业中,常常会出现这样的情况:用键值数据库系统存储用户数据;用文档数据库系统存储图片和文档资料;用图形数据库系统存储推荐引擎等用户关联数据;用关系数据库系统存储在线商城中的交易信息。总而言之,不同的数据库系统适合不同的数据类型,每种数据库系统都有其优点。但是用多种数据库系统来存储多种企业数据类型,这对服务器资源是严重的浪费,此外,新的数据库系统常常需要新的维护人员,这增加了企业的数据库系统的人力维护成本。所以研究一种能存储所有数据库系统的数据库系统是非常有意义的事情。

非关系型数据库具有很多传统关系型数据库所不具备的优点,非关系型数据库抛弃了关系型数据库中严格的事务和强一致性要求,存储的一般都是没有关系的非结构化数据,所以非关系型数据库更加容易在分布式环境下部署。非关系型数据库没有关系型数据库固有的关系模型,存储的数据不局限于关系型结构化数据,使得更加容易使用。非关系型数据库不必要求强一致性和分布式事务,所以在分布式环境下,更加容易扩展,更加适应大数据的存储和处理。这些特点都使得非关系型数据库变得越来越流行,使用的公司越来越多。
虽然新型非关系型数据库在大数据量的存储和处理上有很大的优势,
但是关系数据库在很多关键领域又是无可替代的,
所以如何结合关系型数据库和非关系型数据库的优点来开发一种新的分布式数据库系统成为当前数据库领域的关键问题。

基于移动互联网的快速发展对数据存储的需求,需要开发一种能结合关系型数据库和非关系型数据库的优点,能同时满足应用开发的便利性和应用程序对数据存储的高性能需求,其能存储多种类型的数据。
如何将关系数据库技术、非关系数据库技术
和分布式技术的优点相集合来开发一个新的分布式数据库具有极现实的意义。相对于传统的关系型数据库,这样的数据库结合分布式非关系型数据库的优点,能更加容易的部署在集群环境下,能更加容易存储大量的结构化数据,同时也能存储多种非结构化的数据类型;相对于非关系型数据库,这样的数据库结合了关系型数据库的优点,兼容SQL接口,能存储结构化数据,更加容易满足传统应用程序的开发需求。本文设计和实现的分布数据库系统JSQL就是一种将分布式非关系数据库技术和关系
数据库技术相结合的系统,和传统的关系型数据库和其他非关系型数据库系统相比,JSQL具有下面这些优点:

\begin{enumerate}
	\item 结合关系数据库和非关系型数据库的优点
\end{enumerate}

	JSQL分布式数据库系统是一种将非关系型数据技术和关系数据库技术相结合的系统,首先,它有非关系数据库系统的优点,能够处理非常大的数据,有很好性能和可扩展性,能支持键值模型,文档模型,能存储无结构化的数据对象。然后,JSQL作为关系数据库系统,能
	满足应用程序的开发要求,为应用程序提供SQL接口,提供关系操作和事务支持。提供的SQL接口让所有数据库管理员能快速入手操作数据库,提高数据库和应用软件开发效率。
	
	\begin{enumerate}[resume]
		\item 提高数据库系统资源使用效率
	\end{enumerate}

一方面,JSQL是一种多模型数据库系统,其能存储结构化数据库系统代替传统的关系型数据库系统,其能存储很多非关系型数据库系统所能存储的数据类型。将所有数据存储在一个系统中,提高了服务器资源的使用效率。另一方面,JSQL分布式数据库系统具有负载均衡功能,实现了分布式动态负载均衡算法,能根据当前系统的负载信息状态,将客户端的请求路由到正确的分布式数据库节点上面去,能均衡利用分布式集群系统中的数据库的处理能力。采用分布式的负载均衡算法,所有分布式集群中的机器都能参加到整个算法中来,每个集群节点都能成为分布式管理节点来实现分布式负载均衡算法,这样就能避免单点故障问题。通过实现动态负载均衡算法,分布式数据库系统的资源能均衡使用,提高了系统中各种网络和存储资源的高效利用。
	
	\begin{enumerate}[resume]
		\item 提高系统的访问效率
	\end{enumerate}

	本文设计的分布式系统能为客户端选择最合适的分布式数据库节点,这样能就够提高前端客户端的效率,使得其能更加高效的访问分布式数据库的资源。系统为客户端选择最近的节点,同时也节约了系统的网络资源。
	
	\begin{enumerate}[resume]
		\item 提高系统的可扩展性
	\end{enumerate}

	本文设计的分布式数据库系统,其分布式数据库节点集群采用多主分布式架构,能动态
	的增加节点,而不需要用户任何的配置。在动态扩展能力的前提下,当我们需要增加分布式系统的整体存储和性能时,只需要在当前分布式集群中增加一台计算机节点,就比如淘宝双十一的时候需要暂时增加系统的处理能力,就可以动态增加节点。而当数据库的读写需求降低时,只需要减少分布式数据库的节点就能动态的减少系统的资源,提高系统的使用效率。这种可以动态增加和减少系统节点的能力使得系统可以弹性扩展。
	
	\begin{enumerate}[resume]
		\item 提高系统可移植性
	\end{enumerate}

	本文设计和实现的分布式数据库系统全套代码由JAVA语言实现,而JAVA是跨平台的开发语言,所以用JAVA语言开发的分布式系统JSQL能够部署到更广泛的平台上。使得分布式数据库系统可以在不同的平台上迁移和移植。
	
	\begin{enumerate}[resume]
		\item 提高数据库系统安全性和易用性
	\end{enumerate}

将所有数据类型存储在一个数据库系统中,能够方便数据库管理员更加容易的维护系统,一个系统发生错误的概率比多个系统发生错误的概率要大的多,也更容易维护。
	当今数据的重要性越来越重要,如何保证数据库的安全性是现在急切需要解决的问题,如何防止用户随意篡改数据在系统的设计上作为一个重要的功能需求。JSQL从数据库底层实现了数据库的审计功能,提供比其他方法更高的性能。管理员通过管理客户端能够监视数据库系统的运行状态,能监控数据的更改情况。
	
		\begin{enumerate}[resume]
		\item 减少企业和单位的数据库维护成本
	\end{enumerate}
	
	JSQL是一个结合关系型和非关系型数据库系统优点的系统,其能存储多种数据类型,能一个分布式的多模型数据库。企业在部署这样的系统以后,就不再需要部署不同的数据库系统来管理不同的数据类型。每种数据库系统都需要新的维护人员,非关系型数据库因为更新非常快,对维护人员的要求也很高,如果所有数据都能存储到一个系统里面,那么将减少企业和单位的人力成本。
\section{分布式数据库研究历史与现状}
数据库技术出现在上世界中期以后。在没有数据库技术的时候,使用系统的用户主要是用操作系统的文件存储数据。但是文件存储有太多的缺点,比如不容易在不同应用程序之间共享,不容易更改数据的格式,文件存储方式的导致数据库管理系统的出现。从60年代开始,出现了各种各样的数据库,其建立在不同的数据模型下,其中最有名的是层数据模型、网格数据模型、对象数据模型和关系数据模型。其中关系数据模型建立在严格的关系代数理论之上,具有数据独立性更高等优点而变得越发流行。随着关系数据库系统的理论逐渐成熟,关系型数据库系统得到了广泛的应用。在1974年,美国学者发明了结构化查询语言,这就是现在非常流行的SQL,SQL是现今大多数数据库系统,特别是关系型数据库系统的标准操作语言。SQL包括数据定义和数据操作等语言,使用这种语言不需要用户具有数据库和计算机程序设计相关的知识,这使得其在数据库管理人员中变得越来越流行,如今,每个数据库管理员都基本掌握这种数据操作语言。在90年代以后,关系型数据库在企业中成为标准,其操作语言的很多优点,使得其更加容易使用,容易被企业数据管理人员所接受。

分布式数据库系统是部署在网络中多台计算机中的数据库系统,其结合现在的互联网网络技术和数据库技术,使得数据库的存储和处理能力在分布式环境下得到了极大的增强。分布式数据库技术的发展开始的很早,但是分布式系统在企业中的应用还是20世纪90年代的事情。
因为在90年代以后,互联网才进入高速的发展,对数据存储的需求得到了爆发的增长。在互联网的快速发展下,分布式数据库系统在市面上出现了,在这其中出现了各种各样的分布式理论,这又促进了分布式数据库系统的发展。 在2002,美国有学者提出了CAP理论,CAP理论认为,在分布式数据库系统中,不可能同时满足CAP这3个条件,这三个条件分别是一致性、可用性和分区容忍性。 分布式系统理论CAP认为,在分布式系统中,不可能同时满足一致性、可用性和分区容忍性这三个要求。根据CAP理论的指导,我们知道,传统的关系型数据库系统不能很好的满足现状分布式数据库的需求,因为现状分布式系统的发展是为了存储大容量数据而产生的。但是在关系型数据库中,其满足必须满足强一致性,所以分布式关系型数据库系统是不能再同时满足可用性和分区容忍性的。为了满足大数据量的存储,出现了很多非关系型数据库系统。

在二十一世纪以后,移动应用存储的快速发展,导致了非关系型数据库的出现。
和关系型数据库不同的是,非关系型数据库抛弃了强一致性,所以根据CAP理论,其可以同时满足可用性和分区容忍性,这样就使得非关系型数据库系统更加容易部署到分布式的环境下,使得其更加适合存储大量的数据。 非关系型数据库不支持强一致性,不支持强事务。因为抛弃了强一致性
非关系型数据库支持高可用性和分区容忍性,使得其更加容易使用。在CAP理论出现以后,出现了很多非关系型系统,这些系统因为抛弃了强一致性,所以其可用性和性能都比传统的关系型数据库得到了很大的提高。特别使用很多不需要事务和强一致性的应用程序的需求,就比如现在的贴吧和微信的互联网应用,其得事务没有要求。在这些应用中,用户数据的丢失和错误并不会产生很大的影响。


在过去几年,随着移动互联网的快速发展,应用程序对数据存储的需求越来越高,分布式非关系型数据库得到了广发的使用,同时,CAP理论也证明了它的正确性。
非关系型数据库抛弃了对强一致性的支持,非关系型数据库可以应用到对数据的强一致性和对事物不高的应用程序上面。但是,那些对强一致性和事物有要求的应用,比如
电子商务,就不能使用非关系型数据库,因为其对事物有严格的要求,用户数据要保证正确和保证强一致性。但是,这些应用也发展的毕竟快,对大数据量的存储需求也在提高,所以如何存储这种数据成了一种新的挑战。要存储大量的数据,而单机系统的存储能力是有限度的,所以系统必须要能部署在分布式的环境下,其就必须要满足可用性和分区容忍性,但是分布式系统理论认为,在满足了可用性和分区容忍性这两个特点以后,就不能满足一致性。不能满足一致性对电子商务这样的应用就不能使用。所以现在的做法是尽量减少可用性和一致性,来开发新的分布式关系型数据库。在之前几年,每个大的公司都对这样的数据库开发提高了兴趣,这种数据库不但要满足关系型数据库的强一致性和事物的需求,还要部署在分布式环境下,满足对大数据存储的要求。
下面对对目前市面上比较有代表性的分布式关系数据库系统进行简单的介绍。
这些系统就是为了克服关系型数据库的缺点,同时能部署在分布式环境下存储大数据量结构化数据的分布式数据库。

\begin{enumerate}
	\item Google Spanner
\end{enumerate}

	Spanner是Google开发的一个可扩展的、全球分布式的数据库。
	在最高层面,Spanner就是一个分布式数据数据库,其将数据存储在分布式环境下的很多台计算机上,这些机器被部署在全球各地。在这个分布式数据库中,通过分片的复制,满足了数据的可用性。当一个副本失败以后,其他节点会自动恢复副本。在系统中,分布式系统会自动的将用户存储的数据进行分片,然后复制存储到不同的计算机节点。Spanner作为google开发的分布式关系数据库系统,其能部署在全球的各国国家和地区,能够存储海量的关系型结构化数据,还能支持全球化的分布式事务。
		
	通过在每个不同的洲之间复制数据,可以保证,即使发生特大的自然灾害,存储在分布式数据库系统中的数据依然是可以用的。
	Spanner是谷歌开发的一种全球化的分布式数据库,其提供了很多关系型数据库的特性。使用Spanner分布式数据库系统的应用程序可以对存储在系统中的应用数据进行详细的规划和配置,可以选择副本和复制策略。应用可以详细规定,哪些数据存储在哪些分区、哪个洲,不同的数据副本保存在什么节点上。还能配置,在副本失效的时候,从什么地方恢复。分布式数据库也可以在不同的分布式机器之间来回迁移数据,达到资源更加平衡的使用。作为一个分布式的数据库,其不但保证了每个数据分配的强一致性,而且通过原子操作,实现了分布式的事务。
	
	\begin{enumerate}[resume]
		\item 阿里巴巴OceanBase
	\end{enumerate}
	
	
	OceanBase是一个阿里公司开发的一个分布式数据库系统,其能存储结构化的数据,是一个分布式关系型数据库系统,支持关系数据库系统中的常用操作,比如关系表的连接和事务。阿里巴巴在开发OceanBase的时候,把开发精力集中在数据库的主要功能上,对不常用的数据库功能,比如视图和存储过程,暂时抛弃,其主要关注分布式跨节点的事务功能上。
	
	OceanBase在阿里的内部很多系统中已经得到了应用,比如最为出名的支付宝。支付宝在中国移动支付市场占有很高的比例,拥有上亿的用户,支付宝的成功运行使得OceanBase的可用性得到了证明。等待上线的应用还包括其他很多阿里巴巴内部和外部的应用程序中,每天更新的数据量超过20亿,更新数据量更是超过2.5TB。
	
	OceanBase设计和实现的时候抛弃了很多不关键的数据库功能,使得其能很快的应用于生成实践中,而且根据CAP理论,在满足可用性和分区容忍性的时候,对强一致性要求必须要选择抛弃,阿里巴巴根据自身应用的特点,对不关键的功能进行了选择的抛弃。
	
	\begin{enumerate}[resume]
		\item 微软SQL Azure
	\end{enumerate}
	
	
	SQL Azure 是微软开发的部署在云上的分布式关系数据库系统,为应用程序提高分布式的存储服务。其主要部署在windows操作系统之上,为企业用户提高服务。
	
	SQL Azure是一个强大的分布式的数据库,其建立的传统关系型数据库技术之上,很多传统关系型系统的功能在这上面得到了支持。SQL Azure是一个云数据库系统,将数据存储基础结构托管在云上,大大地降低了企业在IT方面的资源投入。企业用户按需付费使用分布式关系数据库,不用自己维护分布式数据库的部署,不用自己管理系统。
	
	数据在企业中的变得越来越重要,所以要确保数据的安全,同时也要确保数据的高可用性,这样才能满足现在企业应用发展的需要。SQL Azure完美的支持可用性,在分布式数据库SQL Azure系统中,任何的数据库操作都会同步在其他分布式数据库节点,以保证数据的高可用性;SQL Azure支持分布式集群,支持故障自动转移,而且不需要人工手动配置。
	
	这些数据库系统都是为存储结构化数据而开发的,而没有考虑到当前各种新型数据类型的存储。导致企业在部署这样的系统的时候,还需要再部署其他系统,增加了服务器和人力成本,不能达到资源的高效利用。
\section{论文主要工作与创新}
在学习了有关关系型数据和非关系型数据库的有关理论和技术以后,
论文作者基于Java语言设计和实现了一个分布式多模型数据库JSQL
,JSQL结合了非关系型数据库和关系型数据库的优点,能存储多种数据类型,能部署在分布式环境下,动态扩展读写能力,从数据库系统的底层加入审计功能以保证系统的安全。在系统的设计和实现过程中做的主要工作包括:
\begin{enumerate}[fullwidth,itemindent=2em]
	\item 利用OrientDB和ElasticSearch非关系型存储引擎设计和实现了关系型数据库的本地存储系统。其结合了非关系型数据库系统的优点,底层采用新的存储结构,能够很好的避免关系型存储引擎在多表连接下性能的严重下降。
	\item 实现了Mysql通信协议,使得可以通过Mysql客户端来连接JSQL分布式数据库,
	方便Mysql用户迁移到本数据库系统。在jsql之上,可以实现各种通信协议以连接不同的数据库客户端,考虑到开源mysql数据库系统使用的广泛性,论文暂时只实现了mysql的通信协议。
	\item 实现了对SQL语句的解析和执行,完成对关系数据库接口的支持。在非关系型存储引擎之上支持关系数据库操作就必须要实现SQL语句的解析和执行。因为SQL是关系型数据库系统通用的操作语言。
	\item 实现了数据库的多主分布式架构,使得数据可以存储在多台计算机上面,能够动态扩展分布式数据库系统的读写能力。论文利用开源的Hazelcast框架实现集群节点的自动发现,利用新的多队列算法来解决分布式系统的数据一致性问题。
	\item 实现了系统的审计系统,使得系统的安全性得到增强。系统管理员通过连接数据库管理系统,通过图形化的界面能够看到分布式数据库系统的数据更改历史情况。
	\item 设计和实现了分布式负载均衡算法,完成了分布式数据库节点的动态负载均衡功能。分布式系统重要的功能就是如何在分布式集群中找到合适的节点,均衡分配服务器资源,已达到最佳系统总体最佳性能水平。
	\item 在分布式数据库系统开发完成以后,对系统进行了功能和性能测试,验证系统满足需求分析中的功能要求,在数据存储空间占用量的测试上,系统选择和阿里巴巴的OceanBase分布式数据库系统作对比。
\end{enumerate}


与现有的大多数分布式系统相比,论文所述分布式数据库系统具有以下创新:
\begin{enumerate}[fullwidth,itemindent=2em]
	\item 系统结合了非关系型数据库系统和关系型数据库系统的优点,在一个数据库系统里面提供多种数据操纵接口,能够存储多种数据类型的数据。
	\item 系统利用分布式队列实现多主分布式架构,能够动态增加系统读写能力。而传统的关系型数据库系统在分布式部署环境下,只有复制架构,其只能增加系统的读取能力而不能增加系统的更新能力。
	\item 在分布式系统内部实现负载均衡,实现分布式集群节点间服务器资源的均衡使用。能够提高系统的整体性能。
	\item 系统从数据库引擎层开发数据库的安全审计功能。现有的数据库安全产品是在数据库系统之上做的另外一套系统,其需要通过网络连接数据库系统,需要解析数据库的通信协议,所以其整体监控审计性能会受到影响。
\end{enumerate}
\section{论文结构安排}
第一章作为论文的绪论,在这一章里,介绍了论文的选题背景和本论文所述系统研究和开发的意
义。然后阐述了关系数据库和非关系型数据库的发展历史和现状,对当前几个流行的分布式数据库系统进行了简单的介绍。在最好一节,对论文的主要工作与创新进行了阐述。

第二章,主要介绍了本论文相关的理论基础和相关的技术。这一掌首先对数据库系统方面的理论知识进行了介绍,包括数据库系统和关系数据库系统的概念,简单的讨论了NoSQL数据库系统;然后简单介绍了分布式数据库相关理论,
对分布式系统中数据分布和数据复制技术进行了介绍,最后介绍了常用的动态负载均衡算法和分布式MVCC技术。

第三章,作为分布式系统JSQL的分析。本章首先给出分布式数据库JSQL的实现目标,并对系统进行了详细的需求分析。站在分布式数据库系统的使用者角度,对分布式数据库JSQL进行功能说明,并用用例图的形式对其进行进一步说明。在对重要模块进行分析以后,本章后面对分布式系统所用技术进行了介绍和说明。

第四章,作为分布式数据库系统的设计部分,本章包括JSQL的总体设计和模块划分,本章对系统的总体架构和各个模块的详细架构给出了阐述。
系统主要包括分布式管理节点和分布式数据库节点集群。本章对分布式管理节点和分布式数据库节点的各个模块进行了详细的设计。

第五章,是关于系统的具体实现。根据第四章的总体设计和详细设计,本章对设计部分划分的关键模块的实现进行了说明阐述。重点阐述了客户端关键技术实现、管理节点关键技术实现和分布式架构关键技术的实现。对每个关键技术的实现给出详细的流程图,并加以文字分析。

第六章,作为系统的测试部分,本章对论文所述JSQL分布式数据库进行了测试,测试包括数据库功能测试、集群功能测试和性能测试。
对测试的环境和方法步骤进行详细说明,记录测试结果并对其进行分析。

第七章,论文总结了论文所做的工作,分析了所述分布式数据库系统的不足,对未来的工作进行了说明。最后,作者对数据库系统的未来进行了展望。

最后部分,是关于致谢和参考资料。