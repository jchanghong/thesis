% !Mode:: "TeX:UTF-8"

\chapter{系统实现}
本文前面一章设计了本系统的架构图和各个模块的详细功能,本章给出
每个功能模块的具体实现。
\section{代码规范和总体结构}
本数据库系统采用和JAVA语言类似的
Kotlin语言开发。JAVA中的代码以包为组织单位,
这样组织代码的好处是逻辑结构分明。表\ref{codepdf/allpackage}描述了本系统所有的包
和每个包的详细功能。
\pictable[htbp]{本系统所有包和各个包的作用}{}{codepdf/allpackage}
\pictable[htbp]{config包下面每个类的作用}{}{codepdf/config}
图\ref{codepdf/config}给出了config包下面每个类的具体作用,这个包主要是让用户可以配置
数据库的各个方面,比如配置数据库的端口号,配置最大的连接数等等。
后面讲详细解释其他包功能具体的实现。
\section{数据库系统实现}
SQL包下面的类主要是作为数据库管理类,
让用户启动数据库服务器或者关闭数据库服务器。
表\ref{codepdf/sql}描述了SQL包下每个类的作用。
\pictable[htbp]{SQL包的每个类的作用}{}{codepdf/sql}

数据库系统主要包括网络模块,SQL解析模块和存储引擎模块,
下面分别描述每个模块具体的实现。
\subsection{网络模块}
网络模块主要是用了Netty来实现的,所有和网络有关的代码都在netty包下面。表\ref{codepdf/netty}描述了每个类的
功能。
\pictable[htbp]{网络模块中各个类的作用}{}{codepdf/netty}
下面的代码是netty包下面主要的代码,用来启动网络服务器,接受用户的连接请求。

\noindent
\ttfamily
\hlstd{}\hllin{01\ }\hlstd{}\hlcom{/{*}{*}}\\
\hllin{02\ }\hlcom{\ {*}\ }\\
\hllin{03\ }\hlcom{\ {*}\ 服务器}\\
\hllin{04\ }\hlcom{\ {*}/}\hlstd{}\\
\hllin{05\ }\hlstd{}\hlkwc{@Service}\\
\hllin{06\ }\hlstd{}\hlkwa{class\ }\hlstd{NettyServer\ }\hlopt{\symbol{123}}\\
\hllin{07\ }\hlstd{}\hlstd{\ \ \ \ }\hlstd{}\hlkwa{fun\ }\hlstd{}\hlkwd{start}\hlstd{}\hlopt{()\ \symbol{123}}\\
\hllin{08\ }\hlstd{}\hlstd{\ \ \ \ \ \ \ \ }\hlstd{}\hlkwa{val\ }\hlstd{group\ }\hlopt{=\ }\hlstd{}\hlkwd{DefaultEventExecutorGroup}\hlstd{}\hlopt{(}\hlstd{Runtime}\hlopt{.}\hlstd{}\hlkwd{getRuntime}\hlstd{}\hlopt{().}\hlstd{}\hlkwd{availableProcessors}\hlstd{}\hlopt{())}\\
\hllin{09\ }\hlstd{}\hlstd{\ \ \ \ \ \ \ \ }\hlstd{}\hlkwa{val\ }\hlstd{bossGroup\ }\hlopt{=\ }\hlstd{}\hlkwd{NioEventLoopGroup}\hlstd{}\hlopt{(}\hlstd{}\hlnum{1}\hlstd{}\hlopt{)}\\
\hllin{10\ }\hlstd{}\hlstd{\ \ \ \ \ \ \ \ }\hlstd{}\hlkwa{val\ }\hlstd{workerGroup\ }\hlopt{=\ }\hlstd{}\hlkwd{NioEventLoopGroup}\hlstd{}\hlopt{()}\\
\hllin{11\ }\hlstd{}\hlstd{\ \ \ \ \ \ \ \ }\hlstd{}\hlkwa{try\ }\hlstd{}\hlopt{\symbol{123}}\\
\hllin{12\ }\hlstd{}\hlstd{\ \ \ \ \ \ \ \ \ \ \ \ }\hlstd{}\hlkwa{val\ }\hlstd{b\ }\hlopt{=\ }\hlstd{}\hlkwd{ServerBootstrap}\hlstd{}\hlopt{()}\\
\hllin{13\ }\hlstd{}\hlstd{\ \ \ \ \ \ \ \ \ \ \ \ }\hlstd{b}\hlopt{.}\hlstd{}\hlkwd{group}\hlstd{}\hlopt{(}\hlstd{bossGroup}\hlopt{,\ }\hlstd{workerGroup}\hlopt{)}\\
\hllin{14\ }\hlstd{}\hlstd{\ \ \ \ \ \ \ \ \ \ \ \ \ \ \ \ \ \ \ \ }\hlstd{}\hlopt{.}\hlstd{}\hlkwd{channel}\hlstd{}\hlopt{(}\hlstd{NioServerSocketChannel}\hlopt{::}\hlstd{}\hlkwa{class}\hlstd{}\hlopt{.}\hlstd{java}\hlopt{)}\\
\hllin{15\ }\hlstd{}\hlstd{\ \ \ \ \ \ \ \ \ \ \ \ \ \ \ \ \ \ \ \ }\hlstd{}\hlopt{.}\hlstd{}\hlkwd{option}\hlstd{}\hlopt{(}\hlstd{ChannelOption}\hlopt{.}\hlstd{SO\symbol{95}BACKLOG}\hlopt{,\ }\hlstd{}\hlnum{100}\hlstd{}\hlopt{)}\\
\hllin{16\ }\hlstd{}\hlstd{\ \ \ \ \ \ \ \ \ \ \ \ \ \ \ \ \ \ \ \ }\hlstd{}\hlopt{.}\hlstd{}\hlkwd{childHandler}\hlstd{}\hlopt{(}\hlstd{}\hlkwa{object\ }\hlstd{}\hlopt{:\ }\hlstd{ChannelInitializer}\hlopt{\symbol{60}}\hlstd{SocketChannel}\hlopt{\symbol{62}()\ \symbol{123}}\\
\hllin{17\ }\hlstd{}\hlstd{\ \ \ \ \ \ \ \ \ \ \ \ \ \ \ \ \ \ \ \ \ \ \ \ }\hlstd{}\hlkwc{@Throws}\hlstd{}\hlopt{(}\hlstd{Exception}\hlopt{::}\hlstd{}\hlkwa{class}\hlstd{}\hlopt{)}\\
\hllin{18\ }\hlstd{}\hlstd{\ \ \ \ \ \ \ \ \ \ \ \ \ \ \ \ \ \ \ \ \ \ \ \ }\hlstd{}\hlkwa{public\ override\ fun\ }\hlstd{}\hlkwd{initChannel}\hlstd{}\hlopt{(}\hlstd{ch}\hlopt{:\ }\hlstd{SocketChannel}\hlopt{)\ \symbol{123}}\\
\hllin{19\ }\hlstd{}\hlstd{\ \ \ \ \ \ \ \ \ \ \ \ \ \ \ \ \ \ \ \ \ \ \ \ \ \ \ \ }\hlstd{}\hlkwa{val\ }\hlstd{p\ }\hlopt{=\ }\hlstd{ch}\hlopt{.}\hlstd{}\hlkwd{pipeline}\hlstd{}\hlopt{()}\\
\hllin{20\ }\hlstd{}\hlstd{\ \ \ \ \ \ \ \ \ \ \ \ \ \ \ \ \ \ \ \ \ \ \ \ \ \ \ \ }\hlstd{}\hlslc{//p.addLast(new\ LoggingHandler(LogLevel.INFO));}\\
\hllin{21\ }\hlstd{}\hlstd{\ \ \ \ \ \ \ \ \ \ \ \ \ \ \ \ \ \ \ \ \ \ \ \ \ \ \ \ }\hlstd{p}\hlopt{.}\hlstd{}\hlkwd{addLast}\hlstd{}\hlopt{(}\hlstd{}\hlstr{"idle"}\hlstd{}\hlopt{,\ }\hlstd{}\hlkwd{IdleStateHandler}\hlstd{}\hlopt{(}\hlstd{}\hlnum{10}\hlstd{}\hlopt{,\ }\hlstd{}\hlnum{5}\hlstd{}\hlopt{,\ }\hlstd{}\hlnum{0}\hlstd{}\hlopt{))}\\
\hllin{22\ }\hlstd{}\hlstd{\ \ \ \ \ \ \ \ \ \ \ \ \ \ \ \ \ \ \ \ \ \ \ \ \ \ \ \ }\hlstd{p}\hlopt{.}\hlstd{}\hlkwd{addLast}\hlstd{}\hlopt{(}\hlstd{}\hlstr{"decoder"}\hlstd{}\hlopt{,\ }\hlstd{byteToMysqlDecoder}\hlopt{)}\\
\hllin{23\ }\hlstd{}\hlstd{\ \ \ \ \ \ \ \ \ \ \ \ \ \ \ \ \ \ \ \ \ \ \ \ \ \ \ \ }\hlstd{p}\hlopt{.}\hlstd{}\hlkwd{addLast}\hlstd{}\hlopt{(}\hlstd{}\hlstr{"packet"}\hlstd{}\hlopt{,\ }\hlstd{bytetomysql}\hlopt{)}\\
\hllin{24\ }\hlstd{}\hlstd{\ \ \ \ \ \ \ \ \ \ \ \ \ \ \ \ \ \ \ \ \ \ \ \ \ \ \ \ }\hlstd{p}\hlopt{.}\hlstd{}\hlkwd{addLast}\hlstd{}\hlopt{(}\hlstd{group}\hlopt{,\ }\hlstd{}\hlstr{"hander"}\hlstd{}\hlopt{,\ }\hlstd{mysqlPacketHander}\hlopt{)}\\
\hllin{25\ }\hlstd{}\hlstd{\ \ \ \ \ \ \ \ \ \ \ \ \ \ \ \ \ \ \ \ \ \ \ \ }\hlstd{}\hlopt{\symbol{125}}\\
\hllin{26\ }\hlstd{}\hlstd{\ \ \ \ \ \ \ \ \ \ \ \ \ \ \ \ \ \ \ \ }\hlstd{}\hlopt{\symbol{125})}\\
\hllin{27\ }\hlstd{}\hlstd{\ \ \ \ \ \ \ \ \ \ \ \ }\hlstd{}\hlkwa{val\ }\hlstd{f\ }\hlopt{=\ }\hlstd{b}\hlopt{.}\hlstd{}\hlkwd{bind}\hlstd{}\hlopt{(}\hlstd{PORT}\hlopt{).}\hlstd{}\hlkwd{sync}\hlstd{}\hlopt{()}\\
\hllin{28\ }\hlstd{}\hlstd{\ \ \ \ \ \ \ \ \ \ \ \ }\hlstd{}\hlslc{//\ Wait\ until\ the\ server\ socket\ is\ closed.}\\
\hllin{29\ }\hlstd{}\hlstd{\ \ \ \ \ \ \ \ \ \ \ \ }\hlstd{logger}\hlopt{.}\hlstd{}\hlkwd{info}\hlstd{}\hlopt{(}\hlstd{}\hlstr{"server\ start}\hlstd{\ \ }\hlstr{complete....................\ "}\hlstd{}\hlopt{)}\\
\hllin{30\ }\hlstd{}\hlstd{\ \ \ \ \ \ \ \ \ \ \ \ }\hlstd{Minformation\symbol{95}schama}\hlopt{.}\hlstd{}\hlkwd{init\symbol{95}if\symbol{95}notexits}\hlstd{}\hlopt{()}\\
\hllin{31\ }\hlstd{}\hlstd{\ \ \ \ \ \ \ \ \ \ \ \ }\hlstd{}\hlkwa{if\ }\hlstd{}\hlopt{(}\hlstd{config}\hlopt{.}\hlstd{distributed}\hlopt{)\ \symbol{123}}\\
\hllin{32\ }\hlstd{}\hlstd{\ \ \ \ \ \ \ \ \ \ \ \ \ \ \ \ }\hlstd{myHazelcast}\hlopt{.}\hlstd{}\hlkwd{inits}\hlstd{}\hlopt{()}\\
\hllin{33\ }\hlstd{}\hlstd{\ \ \ \ \ \ \ \ \ \ \ \ }\hlstd{}\hlopt{\symbol{125}}\\
\hllin{34\ }\hlstd{}\hlstd{\ \ \ \ \ \ \ \ \ \ \ \ }\hlstd{f}\hlopt{.}\hlstd{}\hlkwd{channel}\hlstd{}\hlopt{().}\hlstd{}\hlkwd{closeFuture}\hlstd{}\hlopt{().}\hlstd{}\hlkwd{sync}\hlstd{}\hlopt{()}\\
\hllin{35\ }\hlstd{}\hlstd{\ \ \ \ \ \ \ \ }\hlstd{}\hlopt{\symbol{125}\ }\hlstd{finally\ }\hlopt{\symbol{123}}\\
\hllin{36\ }\hlstd{}\hlstd{\ \ \ \ \ \ \ \ \ \ \ \ }\hlstd{}\hlslc{//\ Shut\ down\ all\ event\ loops\ to\ terminate\ all\ threads.}\\
\hllin{37\ }\hlstd{}\hlstd{\ \ \ \ \ \ \ \ \ \ \ \ }\hlstd{bossGroup}\hlopt{.}\hlstd{}\hlkwd{shutdownGracefully}\hlstd{}\hlopt{()}\\
\hllin{38\ }\hlstd{}\hlstd{\ \ \ \ \ \ \ \ \ \ \ \ }\hlstd{workerGroup}\hlopt{.}\hlstd{}\hlkwd{shutdownGracefully}\hlstd{}\hlopt{()}\\
\hllin{39\ }\hlstd{}\hlstd{\ \ \ \ \ \ \ \ }\hlstd{}\hlopt{\symbol{125}}\\
\hllin{40\ }\hlstd{}\hlstd{\ \ \ \ }\hlstd{}\hlopt{\symbol{125}}\\
\hllin{41\ }\hlstd{}\hlopt{\symbol{125}}\hlstd{}
\mbox{}
\normalfont
\normalsize


除了实现网络模块以外,我们还要实现通信协议,系统
采用了和mysql一样的通信协议,在mysql包下面实现了mysql的通信协议。
表\ref{codepdf/mysql}给出了实现通信协议所用到的所有的类。
\pictable[htbp]{通信协议实现所用到的类}{}{codepdf/mysql}

网络服务器接受到客服端的连接以后,就要解析mysql的通信协议,
包每一个包封装到具体的对象里面,表\ref{codepdf/mysqlmysql}描述了所有的mysql通信协议的封装对象。
\pictable[htbp]{mysql所有协议包的封装对象}{}{codepdf/mysqlmysql}
mysql里面有很多的包,每个包都需要一个类来实现,下面是命令包的源代码,
其他包的实现大体相同。

\noindent
\ttfamily
\hlstd{}\hllin{01\ }\hlstd{}\hlkwa{class\ }\hlstd{CommandPacket\ }\hlopt{:\ }\hlstd{}\hlkwd{MySQLPacket}\hlstd{}\hlopt{()\ \symbol{123}}\\
\hllin{02\ }\hlstd{}\hlstd{\ \ \ \ }\hlstd{}\hlkwa{var\ }\hlstd{command}\hlopt{:\ }\hlstd{}\hlkwb{Byte\ }\hlstd{}\hlopt{=\ }\hlstd{}\hlnum{0}\\
\hllin{03\ }\hlstd{}\hlstd{\ \ \ \ }\hlstd{}\hlkwa{var\ }\hlstd{arg}\hlopt{:\ }\hlstd{ByteArray?\ }\hlopt{=\ }\hlstd{}\hlkwa{null}\\
\hllin{04\ }\hlstd{}\hlstd{\ \ \ \ }\hlstd{}\hlkwa{override\ fun\ }\hlstd{}\hlkwd{read}\hlstd{}\hlopt{(}\hlstd{data}\hlopt{:\ }\hlstd{ByteArray}\hlopt{)\ \symbol{123}}\\
\hllin{05\ }\hlstd{}\hlstd{\ \ \ \ \ \ \ \ }\hlstd{}\hlkwa{val\ }\hlstd{mm\ }\hlopt{=\ }\hlstd{}\hlkwd{MySQLMessage}\hlstd{}\hlopt{(}\hlstd{data}\hlopt{)}\\
\hllin{06\ }\hlstd{}\hlstd{\ \ \ \ \ \ \ \ }\hlstd{packetLength\ }\hlopt{=\ }\hlstd{mm}\hlopt{.}\hlstd{}\hlkwd{readUB3}\hlstd{}\hlopt{()}\\
\hllin{07\ }\hlstd{}\hlstd{\ \ \ \ \ \ \ \ }\hlstd{packetId\ }\hlopt{=\ }\hlstd{mm}\hlopt{.}\hlstd{}\hlkwd{read}\hlstd{}\hlopt{()}\\
\hllin{08\ }\hlstd{}\hlstd{\ \ \ \ \ \ \ \ }\hlstd{command\ }\hlopt{=\ }\hlstd{mm}\hlopt{.}\hlstd{}\hlkwd{read}\hlstd{}\hlopt{()}\\
\hllin{09\ }\hlstd{}\hlstd{\ \ \ \ \ \ \ \ }\hlstd{arg\ }\hlopt{=\ }\hlstd{mm}\hlopt{.}\hlstd{}\hlkwd{readBytes}\hlstd{}\hlopt{()}\\
\hllin{10\ }\hlstd{}\hlstd{\ \ \ \ }\hlstd{}\hlopt{\symbol{125}}\\
\hllin{11\ }\hlstd{}\hlstd{\ \ \ \ }\hlstd{}\hlkwc{@Throws}\hlstd{}\hlopt{(}\hlstd{IOException}\hlopt{::}\hlstd{}\hlkwa{class}\hlstd{}\hlopt{)}\\
\hllin{12\ }\hlstd{}\hlstd{\ \ \ \ }\hlstd{}\hlkwa{fun\ }\hlstd{}\hlkwd{write}\hlstd{}\hlopt{(}\hlstd{out}\hlopt{:\ }\hlstd{OutputStream}\hlopt{)\ \symbol{123}}\\
\hllin{13\ }\hlstd{}\hlstd{\ \ \ \ \ \ \ \ }\hlstd{StreamUtil}\hlopt{.}\hlstd{}\hlkwd{writeUB3}\hlstd{}\hlopt{(}\hlstd{out}\hlopt{,\ }\hlstd{}\hlkwd{calcPacketSize}\hlstd{}\hlopt{())}\\
\hllin{14\ }\hlstd{}\hlstd{\ \ \ \ \ \ \ \ }\hlstd{StreamUtil}\hlopt{.}\hlstd{}\hlkwd{write}\hlstd{}\hlopt{(}\hlstd{out}\hlopt{,\ }\hlstd{packetId}\hlopt{)}\\
\hllin{15\ }\hlstd{}\hlstd{\ \ \ \ \ \ \ \ }\hlstd{StreamUtil}\hlopt{.}\hlstd{}\hlkwd{write}\hlstd{}\hlopt{(}\hlstd{out}\hlopt{,\ }\hlstd{command}\hlopt{)}\\
\hllin{16\ }\hlstd{}\hlstd{\ \ \ \ \ \ \ \ }\hlstd{out}\hlopt{.}\hlstd{}\hlkwd{write}\hlstd{}\hlopt{(}\hlstd{arg}\hlopt{)}\\
\hllin{17\ }\hlstd{}\hlstd{\ \ \ \ }\hlstd{}\hlopt{\symbol{125}}\\
\hllin{18\ }\hlstd{}\hlopt{\symbol{125}}\hlstd{}\\
\mbox{}
\normalfont
\normalsize


协议主要有2个阶段,一个是认证阶段,一个是命令阶段,认证阶段就是
接受客服端的请求,然后检查用户的认证信息,比如用户名和密码,
下面代码主要用来实现用户认证的功能。

\noindent
\ttfamily
\hlstd{}\hllin{01\ }\hlstd{}\hlcom{/{*}{*}}\\
\hllin{02\ }\hlcom{\ {*}\ 前端认证处理器}\\
\hllin{03\ }\hlcom{\ {*}\ 处理auth包}\\
\hllin{04\ }\hlcom{\ {*}/}\hlstd{}\\
\hllin{05\ }\hlstd{}\hlkwc{@Component}\\
\hllin{06\ }\hlstd{}\hlkwa{open\ class\ }\hlstd{MysqlAuthHander\ }\hlopt{:\ }\hlstd{MysqlPacketHander\ }\hlopt{\symbol{123}}\\
\hllin{07\ }\hlstd{}\hlstd{\ \ \ \ }\hlstd{}\hlkwa{private\ fun\ }\hlstd{}\hlkwd{handle0}\hlstd{}\hlopt{(}\hlstd{auth}\hlopt{:\ }\hlstd{AuthPacket}\hlopt{,\ }\hlstd{source}\hlopt{:\ }\hlstd{OConnection}\hlopt{)\ \symbol{123}}\\
\hllin{08\ }\hlstd{}\hlstd{\ \ \ \ \ \ \ \ }\hlstd{source}\hlopt{.}\hlstd{schema\ }\hlopt{=\ }\hlstd{auth}\hlopt{.}\hlstd{database}\\
\hllin{09\ }\hlstd{}\hlstd{\ \ \ \ \ \ \ \ }\hlstd{}\hlslc{//\ check\ password}\\
\hllin{10\ }\hlstd{}\hlstd{\ \ \ \ \ \ \ \ }\hlstd{}\hlkwa{if\ }\hlstd{}\hlopt{(!}\hlstd{}\hlkwd{checkPassword}\hlstd{}\hlopt{(}\hlstd{auth}\hlopt{.}\hlstd{password}\hlopt{!!,\ }\hlstd{auth}\hlopt{.}\hlstd{user}\hlopt{!!))\ \symbol{123}}\\
\hllin{11\ }\hlstd{}\hlstd{\ \ \ \ \ \ \ \ \ \ \ \ }\hlstd{}\hlkwa{if\ }\hlstd{}\hlopt{(}\hlstd{config}\hlopt{.}\hlstd{audit}\hlopt{)\ \symbol{123}}\\
\hllin{12\ }\hlstd{}\hlstd{\ \ \ \ \ \ \ \ \ \ \ \ }\hlstd{}\hlkwd{LoginLog}\hlstd{}\hlopt{(}\hlstd{auth}\hlopt{.}\hlstd{user\ ?}\hlopt{:\ }\hlstd{}\hlstr{"null"}\hlstd{}\hlopt{,\ }\hlstd{source}\hlopt{.}\hlstd{host}\hlopt{,\ }\hlstd{}\hlkwa{false}\hlstd{}\hlopt{).}\hlstd{}\hlkwd{sendesServer}\hlstd{}\hlopt{()}\\
\hllin{13\ }\hlstd{}\hlstd{\ \ \ \ \ \ \ \ \ \ \ \ }\hlstd{}\hlopt{\symbol{125}}\\
\hllin{14\ }\hlstd{}\hlstd{\ \ \ \ \ \ \ \ \ \ \ \ }\hlstd{}\hlkwd{failure}\hlstd{}\hlopt{(}\hlstd{ErrorCode}\hlopt{.}\hlstd{ER\symbol{95}ACCESS\symbol{95}DENIED\symbol{95}ERROR}\hlopt{,}\\
\hllin{15\ }\hlstd{}\hlstd{\ \ \ \ \ \ }\hlstd{}\hlstr{"Access\ denied\ for\ user\ '"}\hlstd{\ }\hlopt{+\ }\hlstd{auth}\hlopt{.}\hlstd{user\ }\hlopt{+\ }\hlstd{}\hlstr{"',\ }\\
\hllin{16\ }\hlstr{}\hlstd{\ \ \ \ \ \ }\hlstr{because\ password\ is\ error\ "}\hlstd{}\hlopt{,\ }\hlstd{source}\hlopt{)}\\
\hllin{17\ }\hlstd{}\hlstd{\ \ \ \ \ \ \ \ }\hlstd{}\hlopt{\symbol{125}\ }\hlstd{}\hlkwa{else\ }\hlstd{}\hlopt{\symbol{123}}\\
\hllin{18\ }\hlstd{}\hlstd{\ \ \ \ \ \ \ \ \ \ \ \ }\hlstd{}\hlkwd{success}\hlstd{}\hlopt{(}\hlstd{auth}\hlopt{,\ }\hlstd{source}\hlopt{)}\\
\hllin{19\ }\hlstd{}\hlstd{\ \ \ \ \ \ \ \ }\hlstd{}\hlopt{\symbol{125}}\\
\hllin{20\ }\hlstd{}\hlstd{\ \ \ \ }\hlstd{}\hlopt{\symbol{125}}\\
\hllin{21\ }\hlstd{}\hlopt{\symbol{125}}\hlstd{}\\
\mbox{}
\normalfont
\normalsize


认证成功以后就是命令阶段,服务器接受客服端的命令,然后处理,
下面是命令处理的具体代码。

\noindent
\ttfamily
\hlstd{}\hllin{01\ }\hlstd{}\hlcom{/{*}{*}}\\
\hllin{02\ }\hlcom{\ {*}\ 前端命令处理器}\\
\hllin{03\ }\hlcom{\ {*}\ 处理命令包}\\
\hllin{04\ }\hlcom{\ {*}/}\hlstd{}\\
\hllin{05\ }\hlstd{}\hlkwc{@Component}\\
\hllin{06\ }\hlstd{}\hlkwa{class\ }\hlstd{MysqlCommandHandler\ }\hlopt{:\ }\hlstd{MysqlPacketHander\ }\hlopt{\symbol{123}}\\
\hllin{07\ }\hlstd{}\hlstd{\ \ \ \ }\hlstd{}\hlkwa{override\ fun\ }\hlstd{}\hlkwd{hander}\hlstd{}\hlopt{(}\hlstd{mySQLPacket}\hlopt{:\ }\hlstd{MySQLPacket}\hlopt{,\ }\hlstd{source}\hlopt{:\ }\hlstd{OConnection}\hlopt{)\ \symbol{123}}\\
\hllin{08\ }\hlstd{}\hlstd{\ \ \ \ \ \ \ \ }\hlstd{}\hlkwd{handle0}\hlstd{}\hlopt{(}\hlstd{mySQLPacket\ }\hlkwa{as\ }\hlstd{CommandPacket}\hlopt{,\ }\hlstd{source}\hlopt{)}\\
\hllin{09\ }\hlstd{}\hlstd{\ \ \ \ }\hlstd{}\hlopt{\symbol{125}}\\
\hllin{10\ }\hlstd{}\hlstd{\ \ \ \ }\hlstd{}\hlkwa{private\ fun\ }\hlstd{}\hlkwd{handle0}\hlstd{}\hlopt{(}\hlstd{data}\hlopt{:\ }\hlstd{CommandPacket}\hlopt{,\ }\hlstd{source}\hlopt{:\ }\hlstd{OConnection}\hlopt{)\ \symbol{123}}\\
\hllin{11\ }\hlstd{}\hlstd{\ \ \ \ \ \ \ \ }\hlstd{logger}\hlopt{.}\hlstd{}\hlkwd{debug}\hlstd{}\hlopt{(}\hlstd{data}\hlopt{.}\hlstd{}\hlkwd{toString}\hlstd{}\hlopt{())}\\
\hllin{12\ }\hlstd{}\hlstd{\ \ \ \ \ \ \ \ }\hlstd{logger}\hlopt{.}\hlstd{}\hlkwd{info}\hlstd{}\hlopt{(}\hlstd{}\hlstr{"command\ info"}\hlstd{}\hlopt{)}\\
\hllin{13\ }\hlstd{}\hlstd{\ \ \ \ \ \ \ \ }\hlstd{}\hlkwa{when\ }\hlstd{}\hlopt{(}\hlstd{data}\hlopt{.}\hlstd{command}\hlopt{)\ \symbol{123}}\\
\hllin{14\ }\hlstd{}\hlstd{\ \ \ \ \ \ \ \ \ \ \ \ }\hlstd{MySQLPacket}\hlopt{.}\hlstd{COM\symbol{95}INIT\symbol{95}DB\ }\hlopt{{-}\symbol{62}\ }\hlstd{}\hlkwd{initDB}\hlstd{}\hlopt{(}\hlstd{data}\hlopt{,\ }\hlstd{source}\hlopt{)}\\
\hllin{15\ }\hlstd{}\hlstd{\ \ \ \ \ \ \ \ \ \ \ \ }\hlstd{MySQLPacket}\hlopt{.}\hlstd{COM\symbol{95}QUERY\ }\hlopt{{-}\symbol{62}\ }\hlstd{}\hlkwd{query}\hlstd{}\hlopt{(}\hlstd{data}\hlopt{,\ }\hlstd{source}\hlopt{)}\\
\hllin{16\ }\hlstd{}\hlstd{\ \ \ \ \ \ \ \ \ \ \ \ }\hlstd{MySQLPacket}\hlopt{.}\hlstd{COM\symbol{95}PING\ }\hlopt{{-}\symbol{62}\ }\hlstd{}\hlkwd{ping}\hlstd{}\hlopt{(}\hlstd{source}\hlopt{)}\\
\hllin{17\ }\hlstd{}\hlstd{\ \ \ \ \ \ \ \ \ \ \ \ }\hlstd{MySQLPacket}\hlopt{.}\hlstd{COM\symbol{95}QUIT\ }\hlopt{{-}\symbol{62}\ }\hlstd{}\hlkwd{close}\hlstd{}\hlopt{(}\hlstd{}\hlstr{"quit\ cmd"}\hlstd{}\hlopt{,\ }\hlstd{source}\hlopt{)}\\
\hllin{18\ }\hlstd{}\hlstd{\ \ \ \ \ \ \ \ \ \ \ \ }\hlstd{MySQLPacket}\hlopt{.}\hlstd{COM\symbol{95}PROCESS\symbol{95}KILL\ }\hlopt{{-}\symbol{62}\ }\hlstd{}\hlkwd{kill}\hlstd{}\hlopt{(}\hlstd{data}\hlopt{,\ }\hlstd{source}\hlopt{)}\\
\hllin{19\ }\hlstd{}\hlstd{\ \ \ \ \ \ \ \ \ \ \ \ }\hlstd{MySQLPacket}\hlopt{.}\hlstd{COM\symbol{95}STMT\symbol{95}PREPARE\ }\hlopt{{-}\symbol{62}\ }\hlstd{}\hlkwd{stmtPrepare}\hlstd{}\hlopt{(}\hlstd{data}\hlopt{,\ }\hlstd{source}\hlopt{)}\\
\hllin{20\ }\hlstd{}\hlstd{\ \ \ \ \ \ \ \ \ \ \ \ }\hlstd{MySQLPacket}\hlopt{.}\hlstd{COM\symbol{95}STMT\symbol{95}SEND\symbol{95}LONG\symbol{95}DATA\ }\hlopt{{-}\symbol{62}\ }\hlstd{}\hlkwd{stmtSendLongData}\hlstd{}\hlopt{(}\hlstd{data}\hlopt{,\ }\hlstd{source}\hlopt{)}\\
\hllin{21\ }\hlstd{}\hlstd{\ \ \ \ \ \ \ \ \ \ \ \ }\hlstd{MySQLPacket}\hlopt{.}\hlstd{COM\symbol{95}STMT\symbol{95}RESET\ }\hlopt{{-}\symbol{62}\ }\hlstd{}\hlkwd{stmtReset}\hlstd{}\hlopt{(}\hlstd{data}\hlopt{,\ }\hlstd{source}\hlopt{)}\\
\hllin{22\ }\hlstd{}\hlstd{\ \ \ \ \ \ \ \ \ \ \ \ }\hlstd{MySQLPacket}\hlopt{.}\hlstd{COM\symbol{95}STMT\symbol{95}EXECUTE\ }\hlopt{{-}\symbol{62}\ }\hlstd{}\hlkwd{stmtExecute}\hlstd{}\hlopt{(}\hlstd{data}\hlopt{,\ }\hlstd{source}\hlopt{)}\\
\hllin{23\ }\hlstd{}\hlstd{\ \ \ \ \ \ \ \ \ \ \ \ }\hlstd{MySQLPacket}\hlopt{.}\hlstd{COM\symbol{95}STMT\symbol{95}CLOSE\ }\hlopt{{-}\symbol{62}\ }\hlstd{}\hlkwd{stmtClose}\hlstd{}\hlopt{(}\hlstd{data}\hlopt{,\ }\hlstd{source}\hlopt{)}\\
\hllin{24\ }\hlstd{}\hlstd{\ \ \ \ \ \ \ \ \ \ \ \ }\hlstd{MySQLPacket}\hlopt{.}\hlstd{COM\symbol{95}HEARTBEAT\ }\hlopt{{-}\symbol{62}\ }\hlstd{}\hlkwd{heartbeat}\hlstd{}\hlopt{(}\hlstd{data}\hlopt{,\ }\hlstd{source}\hlopt{)}\\
\hllin{25\ }\hlstd{}\hlstd{\ \ \ \ \ \ \ \ \ \ \ \ }\hlstd{}\hlkwa{else\ }\hlstd{}\hlopt{{-}\symbol{62}\ }\hlstd{source}\hlopt{.}\hlstd{}\hlkwd{writeErrMessage}\hlstd{}\hlopt{(}\hlstd{ErrorCode}\hlopt{.}\hlstd{ER\symbol{95}UNKNOWN\symbol{95}COM\symbol{95}ERROR}\hlopt{,}\\
\hllin{26\ }\hlstd{}\hlstd{\ \ \ \ \ \ \ \ \ \ \ \ \ \ \ \ \ \ \ \ }\hlstd{}\hlstr{"Unknown\ command"}\hlstd{}\hlopt{)}\\
\hllin{27\ }\hlstd{}\hlstd{\ \ \ \ \ \ \ \ }\hlstd{}\hlopt{\symbol{125}}\\
\hllin{28\ }\hlstd{}\hlstd{\ \ \ \ }\hlstd{}\hlopt{\symbol{125}}\\
\hllin{29\ }\hlstd{}\hlstd{\ \ \ \ }\hlstd{}\hlkwa{private\ fun\ }\hlstd{}\hlkwd{query}\hlstd{}\hlopt{(}\hlstd{data}\hlopt{:\ }\hlstd{CommandPacket}\hlopt{,\ }\hlstd{source}\hlopt{:\ }\hlstd{OConnection}\hlopt{)\ \symbol{123}}\\
\hllin{30\ }\hlstd{}\hlstd{\ \ \ \ \ \ \ \ }\hlstd{}\hlkwa{val\ }\hlstd{mm\ }\hlopt{=\ }\hlstd{}\hlkwd{MySQLMessage}\hlstd{}\hlopt{(}\hlstd{data}\hlopt{.}\hlstd{arg}\hlopt{!!)}\\
\hllin{31\ }\hlstd{}\hlstd{\ \ \ \ \ \ \ \ }\hlstd{mm}\hlopt{.}\hlstd{}\hlkwd{position}\hlstd{}\hlopt{(}\hlstd{}\hlnum{0}\hlstd{}\hlopt{)}\\
\hllin{32\ }\hlstd{}\hlstd{\ \ \ \ \ \ \ \ }\hlstd{}\hlkwa{try\ }\hlstd{}\hlopt{\symbol{123}}\\
\hllin{33\ }\hlstd{}\hlstd{\ \ \ \ \ \ \ \ \ \ \ \ }\hlstd{}\hlkwa{val\ }\hlstd{sql\ }\hlopt{=\ }\hlstd{mm}\hlopt{.}\hlstd{}\hlkwd{readString}\hlstd{}\hlopt{(}\hlstd{source}\hlopt{.}\hlstd{charset}\hlopt{)}\\
\hllin{34\ }\hlstd{}\hlstd{\ \ \ \ \ \ \ \ \ \ \ \ }\hlstd{source}\hlopt{.}\hlstd{sqlHander}\hlopt{.}\hlstd{}\hlkwd{handle}\hlstd{}\hlopt{(}\hlstd{sql}\hlopt{!!,\ }\hlstd{source}\hlopt{)}\\
\hllin{35\ }\hlstd{}\hlstd{\ \ \ \ \ \ \ \ }\hlstd{}\hlopt{\symbol{125}\ }\hlstd{}\hlkwd{catch\ }\hlstd{}\hlopt{(}\hlstd{e}\hlopt{:\ }\hlstd{UnsupportedEncodingException}\hlopt{)\ \symbol{123}}\\
\hllin{36\ }\hlstd{}\hlstd{\ \ \ \ \ \ \ \ \ \ \ \ }\hlstd{source}\hlopt{.}\hlstd{}\hlkwd{writeErrMessage}\hlstd{}\hlopt{(}\hlstd{ErrorCode}\hlopt{.}\hlstd{ER\symbol{95}UNKNOWN\symbol{95}CHARACTER\symbol{95}SET}\hlopt{,\ }\hlstd{}\hlstr{"Unknown\ charset\ '"}\hlstd{\ }\hlopt{+\ }\hlstd{source}\hlopt{.}\hlstd{charset\ }\hlopt{+\ }\hlstd{}\hlstr{"'"}\hlstd{}\hlopt{)}\\
\hllin{37\ }\hlstd{}\hlstd{\ \ \ \ \ \ \ \ \ \ \ \ }\hlstd{e}\hlopt{.}\hlstd{}\hlkwd{printStackTrace}\hlstd{}\hlopt{()}\\
\hllin{38\ }\hlstd{}\hlstd{\ \ \ \ \ \ \ \ }\hlstd{}\hlopt{\symbol{125}}\\
\hllin{39\ }\hlstd{}\hlstd{\ \ \ \ }\hlstd{}\hlopt{\symbol{125}}\\
\hllin{40\ }\hlstd{}\hlstd{\ \ \ \ }\hlstd{}\hlkwa{private\ fun\ }\hlstd{}\hlkwd{initDB}\hlstd{}\hlopt{(}\hlstd{data}\hlopt{:\ }\hlstd{CommandPacket}\hlopt{,\ }\hlstd{source}\hlopt{:\ }\hlstd{OConnection}\hlopt{)\ \symbol{123}}\\
\hllin{41\ }\hlstd{}\hlstd{\ \ \ \ \ \ \ \ }\hlstd{}\hlkwa{val\ }\hlstd{mm\ }\hlopt{=\ }\hlstd{}\hlkwd{MySQLMessage}\hlstd{}\hlopt{(}\hlstd{data}\hlopt{.}\hlstd{arg}\hlopt{!!)}\\
\hllin{42\ }\hlstd{}\hlstd{\ \ \ \ \ \ \ \ }\hlstd{mm}\hlopt{.}\hlstd{}\hlkwd{position}\hlstd{}\hlopt{(}\hlstd{}\hlnum{0}\hlstd{}\hlopt{)}\\
\hllin{43\ }\hlstd{}\hlstd{\ \ \ \ \ \ \ \ }\hlstd{}\hlkwa{val\ }\hlstd{db\ }\hlopt{=\ }\hlstd{mm}\hlopt{.}\hlstd{}\hlkwd{readString}\hlstd{}\hlopt{()}\\
\hllin{44\ }\hlstd{}\hlstd{\ \ \ \ \ \ \ \ }\hlstd{}\hlslc{//\ 检查schema的有效性}\\
\hllin{45\ }\hlstd{}\hlstd{\ \ \ \ \ \ \ \ }\hlstd{}\hlkwa{try\ }\hlstd{}\hlopt{\symbol{123}}\\
\hllin{46\ }\hlstd{}\hlstd{\ \ \ \ \ \ \ \ \ \ \ \ }\hlstd{}\hlkwa{if\ }\hlstd{}\hlopt{(!}\hlstd{OConnection}\hlopt{.}\hlstd{DB\symbol{95}ADMIN}\hlopt{.}\hlstd{}\hlkwd{getallDBs}\hlstd{}\hlopt{().}\hlstd{}\hlkwd{contains}\hlstd{}\hlopt{(}\hlstd{db}\hlopt{))\ \symbol{123}}\\
\hllin{47\ }\hlstd{}\hlstd{\ \ \ \ \ \ \ \ \ \ \ \ \ \ \ \ }\hlstd{source}\hlopt{.}\hlstd{}\hlkwd{writeErrMessage}\hlstd{}\hlopt{(}\hlstd{ErrorCode}\hlopt{.}\hlstd{ER\symbol{95}BAD\symbol{95}DB\symbol{95}ERROR}\hlopt{,\ }\hlstd{}\hlstr{"Unknown\ database\ '}\hlipl{\$db}\hlstr{'"}\hlstd{}\hlopt{)}\\
\hllin{48\ }\hlstd{}\hlstd{\ \ \ \ \ \ \ \ \ \ \ \ \ \ \ \ }\hlstd{}\hlkwa{return}\\
\hllin{49\ }\hlstd{}\hlstd{\ \ \ \ \ \ \ \ \ \ \ \ }\hlstd{}\hlopt{\symbol{125}}\\
\hllin{50\ }\hlstd{}\hlstd{\ \ \ \ \ \ \ \ }\hlstd{}\hlopt{\symbol{125}\ }\hlstd{}\hlkwd{catch\ }\hlstd{}\hlopt{(}\hlstd{e}\hlopt{:\ }\hlstd{StorageException}\hlopt{)\ \symbol{123}}\\
\hllin{51\ }\hlstd{}\hlstd{\ \ \ \ \ \ \ \ \ \ \ \ }\hlstd{e}\hlopt{.}\hlstd{}\hlkwd{printStackTrace}\hlstd{}\hlopt{()}\\
\hllin{52\ }\hlstd{}\hlstd{\ \ \ \ \ \ \ \ \ \ \ \ }\hlstd{source}\hlopt{.}\hlstd{}\hlkwd{writeErrMessage}\hlstd{}\hlopt{(}\hlstd{e}\hlopt{.}\hlstd{message}\hlopt{!!)}\\
\hllin{53\ }\hlstd{}\hlstd{\ \ \ \ \ \ \ \ \ \ \ \ }\hlstd{}\hlkwa{return}\\
\hllin{54\ }\hlstd{}\hlstd{\ \ \ \ \ \ \ \ }\hlstd{}\hlopt{\symbol{125}}\\
\hllin{55\ }\hlstd{}\\
\hllin{56\ }\hlstd{}\hlstd{\ \ \ \ \ \ \ \ }\hlstd{source}\hlopt{.}\hlstd{schema\ }\hlopt{=\ }\hlstd{db}\\
\hllin{57\ }\hlstd{}\hlstd{\ \ \ \ \ \ \ \ }\hlstd{source}\hlopt{.}\hlstd{}\hlkwd{writeok}\hlstd{}\hlopt{()}\\
\hllin{58\ }\hlstd{}\hlstd{\ \ \ \ }\hlstd{}\hlopt{\symbol{125}}\\
\hllin{59\ }\hlstd{}\hlstd{\ \ \ \ }\hlstd{companion\ }\hlkwa{object\ }\hlstd{}\hlopt{\symbol{123}}\\
\hllin{60\ }\hlstd{}\hlstd{\ \ \ \ \ \ \ \ }\hlstd{}\hlkwa{var\ }\hlstd{logger\ }\hlopt{=\ }\hlstd{LoggerFactory}\hlopt{.}\hlstd{}\hlkwd{getLogger}\hlstd{}\hlopt{(}\hlstd{MysqlCommandHandler}\hlopt{::}\hlstd{}\hlkwa{class}\hlstd{}\hlopt{.}\hlstd{java}\hlopt{.}\hlstd{simpleName}\hlopt{)}\\
\hllin{61\ }\hlstd{}\hlstd{\ \ \ \ }\hlstd{}\hlopt{\symbol{125}}\\
\hllin{62\ }\hlstd{}\\
\hllin{63\ }\hlstd{}\\
\hllin{64\ }\hlstd{}\hlopt{\symbol{125}}\hlstd{}
\mbox{}
\normalfont
\normalsize

\subsection{SQL解析模块}
\pictable[htbp]{SQL前端连接模块相关的类}{}{codepdf/jsqlsql}
sql解析主要用到了druid开源的框架,解析sql以后就要做具体的数据处理,
下面的代码主要用处理用户的sql语句。

\noindent
\ttfamily
\hlstd{}\hllin{01\ }\hlstd{}\hlcom{/{*}{*}}\\
\hllin{02\ }\hlcom{\ {*}\ 处理sql语句的入口}\\
\hllin{03\ }\hlcom{\ {*}/}\hlstd{}\\
\hllin{04\ }\hlstd{}\hlkwc{@Component}\\
\hllin{05\ }\hlstd{}\hlkwa{class\ }\hlstd{MysqlSQLhander\ }\hlopt{:\ }\hlstd{SQLHander\ }\hlopt{\symbol{123}}\\
\hllin{06\ }\hlstd{}\hlstd{\ \ \ \ }\hlstd{}\hlkwc{@Autowired}\\
\hllin{07\ }\hlstd{}\hlstd{\ \ \ \ }\hlstd{lateinit\ }\hlkwa{private\ var\ }\hlstd{allHanders}\hlopt{:\ }\hlstd{AllHanders}\\
\hllin{08\ }\hlstd{}\hlstd{\ \ }\hlstd{}\hlslc{//所有的sql处理器容器}\\
\hllin{09\ }\hlstd{}\hlstd{\ \ \ \ }\hlstd{}\hlkwc{@PostConstruct}\\
\hllin{10\ }\hlstd{}\hlstd{\ \ \ \ }\hlstd{}\hlkwa{override\ fun\ }\hlstd{}\hlkwd{handle}\hlstd{}\hlopt{(}\hlstd{sql}\hlopt{:\ }\hlstd{}\hlkwb{String}\hlstd{}\hlopt{,\ }\hlstd{c}\hlopt{:\ }\hlstd{OConnection}\hlopt{)\symbol{123}}\\
\hllin{11\ }\hlstd{}\hlstd{\ \ }\hlstd{}\hlslc{//处理正常的sql语句,前端连接}\\
\hllin{12\ }\hlstd{}\hlstd{\ \ \ \ \ \ \ \ }\hlstd{}\hlkwa{if\ }\hlstd{}\hlopt{(}\hlstd{config}\hlopt{.}\hlstd{audit}\hlopt{)\ \symbol{123}}\\
\hllin{13\ }\hlstd{}\hlstd{\ \ \ \ \ \ \ \ \ \ \ \ }\hlstd{}\hlkwd{SqlLog}\hlstd{}\hlopt{(}\hlstd{sql}\hlopt{,\ }\hlstd{c}\hlopt{.}\hlstd{user\ ?}\hlopt{:\ }\hlstd{}\hlstr{"null"}\hlstd{}\hlopt{,\ }\hlstd{c}\hlopt{.}\hlstd{host}\hlopt{).}\hlstd{}\hlkwd{sentoELServer}\hlstd{}\hlopt{()}\\
\hllin{14\ }\hlstd{}\hlstd{\ \ \ \ \ \ \ \ }\hlstd{}\hlopt{\symbol{125}}\\
\hllin{15\ }\hlstd{}\hlstd{\ \ \ \ \ \ \ \ }\hlstd{logger}\hlopt{.}\hlstd{}\hlkwd{info}\hlstd{}\hlopt{(}\hlstd{sql}\hlopt{)}\\
\hllin{16\ }\hlstd{}\hlstd{\ \ \ \ \ \ \ \ }\hlstd{}\hlkwa{if\ }\hlstd{}\hlopt{(}\hlstd{logger}\hlopt{.}\hlstd{isDebugEnabled}\hlopt{)\ \symbol{123}}\\
\hllin{17\ }\hlstd{}\hlstd{\ \ \ \ \ \ \ \ \ \ \ \ }\hlstd{logger}\hlopt{.}\hlstd{}\hlkwd{debug}\hlstd{}\hlopt{(}\hlstd{sql}\hlopt{)}\\
\hllin{18\ }\hlstd{}\hlstd{\ \ \ \ \ \ \ \ }\hlstd{}\hlopt{\symbol{125}}\\
\hllin{19\ }\hlstd{}\hlstd{\ \ \ \ \ \ \ \ }\hlstd{}\hlkwa{val\ }\hlstd{sqlStatement}\hlopt{:\ }\hlstd{SQLStatement}\\
\hllin{20\ }\hlstd{}\hlstd{\ \ \ \ \ \ \ \ }\hlstd{}\hlkwa{try\ }\hlstd{}\hlopt{\symbol{123}}\\
\hllin{21\ }\hlstd{}\hlstd{\ \ \ \ \ \ \ \ \ \ \ \ }\hlstd{sqlStatement\ }\hlopt{=\ }\hlstd{sql}\hlopt{.}\hlstd{}\hlkwd{tosql}\hlstd{}\hlopt{()}\\
\hllin{22\ }\hlstd{}\hlstd{\ \ \ \ \ \ \ \ \ \ \ \ }\hlstd{}\hlkwa{val\ }\hlstd{hander\ }\hlopt{=\ }\hlstd{allHanders}\hlopt{.}\hlstd{handerMap}\hlopt{{[}}\hlstd{sqlStatement}\hlopt{.}\hlstd{javaClass}\hlopt{{]}}\\
\hllin{23\ }\hlstd{}\hlstd{\ \ \ \ \ \ \ \ \ \ \ \ }\hlstd{}\hlkwa{if\ }\hlstd{}\hlopt{(}\hlstd{hander\ }\hlopt{!=\ }\hlstd{}\hlkwa{null}\hlstd{}\hlopt{)\ \symbol{123}}\\
\hllin{24\ }\hlstd{}\hlstd{\ \ \ \ \ \ \ \ \ \ \ \ \ \ \ \ }\hlstd{hander}\hlopt{.}\hlstd{}\hlkwd{handle}\hlstd{}\hlopt{(}\hlstd{sqlStatement}\hlopt{,\ }\hlstd{c}\hlopt{)}\\
\hllin{25\ }\hlstd{}\hlstd{\ \ \ \ \ \ \ \ \ \ \ \ }\hlstd{}\hlopt{\symbol{125}\ }\hlstd{}\hlkwa{else\ }\hlstd{}\hlopt{\symbol{123}}\\
\hllin{26\ }\hlstd{}\hlstd{\ \ \ \ \ \ \ \ \ \ \ \ \ \ \ \ }\hlstd{sqlStatement}\hlopt{.}\hlstd{}\hlkwd{accept}\hlstd{}\hlopt{(}\hlstd{}\hlkwd{MSQLvisitor}\hlstd{}\hlopt{(}\hlstd{c}\hlopt{))}\\
\hllin{27\ }\hlstd{}\hlstd{\ \ \ \ \ \ \ \ \ \ \ \ }\hlstd{}\hlopt{\symbol{125}}\\
\hllin{28\ }\hlstd{}\hlstd{\ \ \ \ \ \ \ \ \ \ \ \ }\hlstd{}\hlkwa{if\ }\hlstd{}\hlopt{(}\hlstd{}\hlkwd{isupdatesql}\hlstd{}\hlopt{(}\hlstd{sql}\hlopt{))\ \symbol{123}}\\
\hllin{29\ }\hlstd{}\hlstd{\ \ \ \ \ \ \ \ \ \ \ \ \ \ \ \ }\hlstd{}\hlkwa{if\ }\hlstd{}\hlopt{(}\hlstd{config}\hlopt{.}\hlstd{distributed}\hlopt{)\ \symbol{123}}\\
\hllin{30\ }\hlstd{}\hlstd{\ \ \ \ \ \ \ \ \ \ \ \ \ \ \ \ \ \ \ \ }\hlstd{myHazelcast}\hlopt{.}\hlstd{}\hlkwd{exeSql}\hlstd{}\hlopt{(}\hlstd{sql}\hlopt{,\ }\hlstd{}\hlkwa{if\ }\hlstd{}\hlopt{(}\hlstd{c}\hlopt{.}\hlstd{schema\ }\hlopt{==\ }\hlstd{}\hlkwa{null}\hlstd{}\hlopt{)\ }\hlstd{}\hlstr{""}\hlstd{}\\
\hllin{31\ }\hlstd{}\hlstd{\ \ \ \ \ \ \ \ \ \ }\hlstd{}\hlkwa{else\ }\hlstd{c}\hlopt{.}\hlstd{schema}\hlopt{!!)}\\
\hllin{32\ }\hlstd{}\hlstd{\ \ \ \ \ \ \ \ \ \ \ \ \ \ \ \ }\hlstd{}\hlopt{\symbol{125}\ }\hlstd{}\hlkwa{else\ }\hlstd{}\hlopt{\symbol{123}}\\
\hllin{33\ }\hlstd{}\hlstd{\ \ \ \ \ \ \ \ \ \ \ \ \ \ \ \ \ \ \ \ }\hlstd{myHazelcast}\hlopt{.}\hlstd{}\hlkwd{exesqlLocal}\hlstd{}\hlopt{(}\hlstd{sql}\hlopt{,\ }\hlstd{}\hlkwa{if\ }\hlstd{}\hlopt{(}\hlstd{c}\hlopt{.}\hlstd{schema\ }\hlopt{==\ }\hlstd{}\hlkwa{null}\hlstd{}\hlopt{)}\\
\hllin{34\ }\hlstd{}\hlstd{\ \ \ \ \ \ \ \ \ \ }\hlstd{}\hlstr{""}\hlstd{\ }\hlkwa{else\ }\hlstd{c}\hlopt{.}\hlstd{schema}\hlopt{!!)}\\
\hllin{35\ }\hlstd{}\hlstd{\ \ \ \ \ \ \ \ \ \ \ \ \ \ \ \ }\hlstd{}\hlopt{\symbol{125}}\\
\hllin{36\ }\hlstd{}\hlstd{\ \ \ \ \ \ \ \ \ \ \ \ }\hlstd{}\hlopt{\symbol{125}}\\
\hllin{37\ }\hlstd{}\hlstd{\ \ \ \ \ \ \ \ \ \ \ \ }\hlstd{}\hlkwa{return}\\
\hllin{38\ }\hlstd{}\hlstd{\ \ \ \ \ \ \ \ }\hlstd{}\hlopt{\symbol{125}\ }\hlstd{}\hlkwd{catch\ }\hlstd{}\hlopt{(}\hlstd{e}\hlopt{:\ }\hlstd{Exception}\hlopt{)\ \symbol{123}}\\
\hllin{39\ }\hlstd{}\hlstd{\ \ \ \ }\hlstd{}\hlslc{//如果不是合法的mysql语句,就报错}\\
\hllin{40\ }\hlstd{}\hlstd{\ \ \ \ \ \ \ \ }\hlstd{}\hlslc{//druid支持的语句就用上面的方法语句处理,如果不支持,就会有异常,}\\
\hllin{41\ }\hlstd{}\hlstd{\ \ \ \ \ \ }\hlstd{}\hlslc{//就自己写代码解析sql语句,处理。}\\
\hllin{42\ }\hlstd{}\hlstd{\ \ \ \ \ \ \ \ \ }\hlstd{}\hlslc{//下面是drop\ event语句的例子,这个例子druid不支持,所以自己写}\\
\hllin{43\ }\hlstd{}\hlstd{\ \ \ \ \ \ \ \ \ \ \ \ }\hlstd{}\hlkwd{handleotherStatement}\hlstd{}\hlopt{(}\hlstd{sql}\hlopt{,\ }\hlstd{c}\hlopt{,\ }\hlstd{e}\hlopt{)}\\
\hllin{44\ }\hlstd{}\hlstd{\ \ \ \ \ \ \ \ }\hlstd{}\hlopt{\symbol{125}}\\
\hllin{45\ }\hlstd{}\hlstd{\ \ \ \ }\hlstd{}\hlopt{\symbol{125}}\\
\hllin{46\ }\hlstd{}\hlstd{\ \ \ \ }\hlstd{}\hlkwa{fun\ }\hlstd{}\hlkwd{handle}\hlstd{}\hlopt{(}\hlstd{sql}\hlopt{:\ }\hlstd{SqlUpdateLog}\hlopt{,\ }\hlstd{c}\hlopt{:\ }\hlstd{OConnection}\hlopt{)\ \symbol{123}}\\
\hllin{47\ }\hlstd{}\hlstd{\ \ }\hlstd{}\hlslc{//处理来自其他服务器的sql语句,同步,不需要前端连接}\\
\hllin{48\ }\hlstd{}\hlstd{\ \ \ \ \ \ \ \ }\hlstd{logger}\hlopt{.}\hlstd{}\hlkwd{info}\hlstd{}\hlopt{(}\hlstd{sql}\hlopt{.}\hlstd{}\hlkwd{toString}\hlstd{}\hlopt{())}\\
\hllin{49\ }\hlstd{}\hlstd{\ \ \ \ \ \ \ \ }\hlstd{}\hlkwa{if\ }\hlstd{}\hlopt{(}\hlstd{logger}\hlopt{.}\hlstd{isDebugEnabled}\hlopt{)\ \symbol{123}}\\
\hllin{50\ }\hlstd{}\hlstd{\ \ \ \ \ \ \ \ \ \ \ \ }\hlstd{logger}\hlopt{.}\hlstd{}\hlkwd{debug}\hlstd{}\hlopt{(}\hlstd{sql}\hlopt{.}\hlstd{}\hlkwd{toString}\hlstd{}\hlopt{())}\\
\hllin{51\ }\hlstd{}\hlstd{\ \ \ \ \ \ \ \ }\hlstd{}\hlopt{\symbol{125}}\\
\hllin{52\ }\hlstd{}\hlstd{\ \ \ \ \ \ \ \ }\hlstd{}\hlkwa{val\ }\hlstd{sqlStatement}\hlopt{:\ }\hlstd{SQLStatement}\\
\hllin{53\ }\hlstd{}\hlstd{\ \ \ \ \ \ \ \ }\hlstd{}\hlkwa{try\ }\hlstd{}\hlopt{\symbol{123}}\\
\hllin{54\ }\hlstd{}\hlstd{\ \ \ \ \ \ \ \ \ \ \ \ }\hlstd{}\hlkwa{val\ }\hlstd{parser\ }\hlopt{=\ }\hlstd{}\hlkwd{MySqlStatementParser}\hlstd{}\hlopt{(}\hlstd{sql}\hlopt{.}\hlstd{sql}\hlopt{)}\\
\hllin{55\ }\hlstd{}\hlstd{\ \ \ \ \ \ \ \ \ \ \ \ }\hlstd{sqlStatement\ }\hlopt{=\ }\hlstd{parser}\hlopt{.}\hlstd{}\hlkwd{parseStatement}\hlstd{}\hlopt{()}\\
\hllin{56\ }\hlstd{}\hlstd{\ \ \ \ \ \ \ \ \ \ \ \ }\hlstd{}\hlkwa{val\ }\hlstd{hander\ }\hlopt{=\ }\hlstd{allHanders}\hlopt{.}\hlstd{handerMap}\hlopt{{[}}\hlstd{sqlStatement}\hlopt{.}\hlstd{javaClass}\hlopt{{]}}\\
\hllin{57\ }\hlstd{}\hlstd{\ \ \ \ \ \ \ \ \ \ \ \ }\hlstd{}\hlkwa{if\ }\hlstd{}\hlopt{(}\hlstd{hander\ }\hlopt{!=\ }\hlstd{}\hlkwa{null}\hlstd{}\hlopt{)\ \symbol{123}}\\
\hllin{58\ }\hlstd{}\hlstd{\ \ \ \ \ \ \ \ \ \ \ \ \ \ \ \ }\hlstd{hander}\hlopt{.}\hlstd{}\hlkwd{handle}\hlstd{}\hlopt{(}\hlstd{sqlStatement}\hlopt{,\ }\hlstd{c}\hlopt{)}\\
\hllin{59\ }\hlstd{}\hlstd{\ \ \ \ \ \ \ \ \ \ \ \ }\hlstd{}\hlopt{\symbol{125}\ }\hlstd{}\hlkwa{else\ }\hlstd{}\hlopt{\symbol{123}}\\
\hllin{60\ }\hlstd{}\hlstd{\ \ \ \ \ \ \ \ \ \ \ \ \ \ \ \ }\hlstd{sqlStatement}\hlopt{.}\hlstd{}\hlkwd{accept}\hlstd{}\hlopt{(}\hlstd{}\hlkwd{MSQLvisitor}\hlstd{}\hlopt{(}\hlstd{c}\hlopt{))}\\
\hllin{61\ }\hlstd{}\hlstd{\ \ \ \ \ \ \ \ \ \ \ \ }\hlstd{}\hlopt{\symbol{125}}\\
\hllin{62\ }\hlstd{}\hlstd{\ \ \ \ \ \ \ \ \ \ \ \ }\hlstd{}\hlkwa{return}\\
\hllin{63\ }\hlstd{}\hlstd{\ \ \ \ \ \ \ \ }\hlstd{}\hlopt{\symbol{125}}\\
\hllin{64\ }\hlstd{}\hlstd{\ \ \ \ }\hlstd{}\hlopt{\symbol{125}}\\
\hllin{65\ }\hlstd{}\hlopt{\symbol{125}}\hlstd{}\\
\mbox{}
\normalfont
\normalsize


\noindent
\ttfamily
\hlstd{}\hllin{01\ }\hlstd{}\hlcom{/{*}{*}}\\
\hllin{02\ }\hlcom{\ {*}\ 所有的sql语句处理器必须是这个类的子类.}\\
\hllin{03\ }\hlcom{\ {*}\ 比如Mupdate。Mselect}\\
\hllin{04\ }\hlcom{\ {*}/}\hlstd{}\\
\hllin{05\ }\hlstd{}\hlkwa{abstract\ class\ }\hlstd{SqlStatementHander\ }\hlopt{\symbol{123}}\\
\hllin{06\ }\hlstd{}\\
\hllin{07\ }\hlstd{}\hlstd{\ \ \ \ \ }\hlstd{}\hlopt{{*}\ }\hlstd{}\hlkwc{@param\ }\hlstd{sqlStatement\ the\ sql\ statement}\\
\hllin{08\ }\hlstd{}\hlstd{\ \ \ \ \ }\hlstd{}\hlopt{{*}\ {*}}\\
\hllin{09\ }\hlstd{}\hlstd{\ \ \ \ \ }\hlstd{}\hlopt{{*}\ }\hlstd{}\hlkwc{@return\ }\hlstd{the\ }\hlkwa{object\ }\hlstd{返回值只有}\hlnum{4}\hlstd{种可能,不然报错!!!}\\
\hllin{10\ }\hlstd{}\hlstd{\ \ \ \ \ }\hlstd{}\hlopt{{*}\ {*}\ }\hlstd{一种是long类型,\ 一种是MyResultSet\ 一种是}\hlkwa{null\ }\hlstd{}\hlopt{,}\\
\hllin{11\ }\hlstd{}\hlstd{\ \ \ }\hlstd{一种是string表示错误的消息}\\
\hllin{12\ }\hlstd{}\hlstd{\ \ \ \ \ }\hlstd{}\hlopt{{*}\ {*}\ }\hlstd{返回其他对面都是错误的}\\
\hllin{13\ }\hlstd{}\hlstd{\ \ \ \ \ }\hlstd{}\hlopt{{*}\ {*}}\\
\hllin{14\ }\hlstd{}\hlstd{\ \ \ \ \ }\hlstd{}\hlopt{{*}\ }\hlstd{}\hlkwc{@throws\ }\hlstd{Exception\ the\ exception}\\
\hllin{15\ }\hlstd{}\hlstd{\ \ \ \ \ }\hlstd{{*}/}\\
\hllin{16\ }\hlstd{}\hlstd{\ \ \ \ }\hlstd{}\hlkwc{@Throws}\hlstd{}\hlopt{(}\hlstd{Exception}\hlopt{::}\hlstd{}\hlkwa{class}\hlstd{}\hlopt{)}\\
\hllin{17\ }\hlstd{}\hlstd{\ \ \ \ }\hlstd{}\hlkwa{protected\ abstract\ fun\ }\hlstd{}\hlkwd{handle0}\hlstd{}\hlopt{(}\hlstd{sqlStatement}\hlopt{:\ }\hlstd{SQLStatement}\hlopt{,\ }\hlstd{c}\hlopt{:\ }\hlstd{OConnection}\hlopt{):\ }\hlstd{}\hlkwb{Any}\hlstd{?}\\
\hllin{18\ }\hlstd{}\hlstd{\ \ \ \ }\hlstd{}\hlkwa{fun\ }\hlstd{}\hlkwd{handle}\hlstd{}\hlopt{(}\hlstd{sqlStatement}\hlopt{:\ }\hlstd{SQLStatement}\hlopt{,\ }\hlstd{connection}\hlopt{:\ }\hlstd{OConnection}\hlopt{)\ \symbol{123}}\\
\hllin{19\ }\hlstd{}\hlstd{\ \ \ \ \ \ \ \ }\hlstd{}\hlkwa{try\ }\hlstd{}\hlopt{\symbol{123}}\\
\hllin{20\ }\hlstd{}\hlstd{\ \ \ \ \ \ \ \ \ \ \ \ }\hlstd{}\hlkwa{val\ }\hlstd{result\ }\hlopt{=\ }\hlstd{}\hlkwd{handle0}\hlstd{}\hlopt{(}\hlstd{sqlStatement}\hlopt{,\ }\hlstd{connection}\hlopt{)}\\
\hllin{21\ }\hlstd{}\hlstd{\ \ \ \ \ \ \ \ \ \ \ \ }\hlstd{}\hlkwa{when\ }\hlstd{}\hlopt{(}\hlstd{result}\hlopt{)\ \symbol{123}}\\
\hllin{22\ }\hlstd{}\hlstd{\ \ \ \ \ \ \ \ \ \ \ \ \ \ \ \ }\hlstd{}\hlkwa{null\ }\hlstd{}\hlopt{{-}\symbol{62}\ }\hlstd{connection}\hlopt{.}\hlstd{}\hlkwd{writeok}\hlstd{}\hlopt{()}\\
\hllin{23\ }\hlstd{}\hlstd{\ \ \ \ \ \ \ \ \ \ \ \ \ \ \ \ }\hlstd{}\hlkwa{is\ }\hlstd{MyResultSet\ }\hlopt{{-}\symbol{62}\ }\hlstd{}\hlkwd{onsuccess}\hlstd{}\hlopt{(}\hlstd{result}\hlopt{.}\hlstd{data}\hlopt{,\ }\hlstd{result}\hlopt{.}\hlstd{columns}\hlopt{,\ }\hlstd{connection}\hlopt{)}\\
\hllin{24\ }\hlstd{}\hlstd{\ \ \ \ \ \ \ \ \ \ \ \ \ \ \ \ }\hlstd{}\hlkwa{is\ }\hlstd{}\hlkwb{Long\ }\hlstd{}\hlopt{{-}\symbol{62}\ }\hlstd{}\hlkwd{onsuccess}\hlstd{}\hlopt{(}\hlstd{result}\hlopt{,\ }\hlstd{connection}\hlopt{)}\\
\hllin{25\ }\hlstd{}\hlstd{\ \ \ \ \ \ \ \ \ \ \ \ \ \ \ \ }\hlstd{}\hlkwa{is\ }\hlstd{}\hlkwb{String\ }\hlstd{}\hlopt{{-}\symbol{62}\ }\hlstd{connection}\hlopt{.}\hlstd{}\hlkwd{writeErrMessage}\hlstd{}\hlopt{(}\hlstd{result}\hlopt{)}\\
\hllin{26\ }\hlstd{}\hlstd{\ \ \ \ \ \ \ \ \ \ \ \ \ \ \ \ }\hlstd{}\hlkwa{else\ }\hlstd{}\hlopt{{-}\symbol{62}\ }\hlstd{connection}\hlopt{.}\hlstd{}\hlkwd{writeok}\hlstd{}\hlopt{()}\\
\hllin{27\ }\hlstd{}\hlstd{\ \ \ \ \ \ \ \ \ \ \ \ }\hlstd{}\hlopt{\symbol{125}}\\
\hllin{28\ }\hlstd{}\hlstd{\ \ \ \ \ \ \ \ }\hlstd{}\hlopt{\symbol{125}\ }\hlstd{}\hlkwd{catch\ }\hlstd{}\hlopt{(}\hlstd{e}\hlopt{:\ }\hlstd{Exception}\hlopt{)\ \symbol{123}}\\
\hllin{29\ }\hlstd{}\hlstd{\ \ \ \ \ \ \ \ \ \ \ \ }\hlstd{e}\hlopt{.}\hlstd{}\hlkwd{printStackTrace}\hlstd{}\hlopt{()}\\
\hllin{30\ }\hlstd{}\hlstd{\ \ \ \ \ \ \ \ \ \ \ \ }\hlstd{}\hlkwd{onerror}\hlstd{}\hlopt{(}\hlstd{e}\hlopt{,\ }\hlstd{connection}\hlopt{)}\\
\hllin{31\ }\hlstd{}\hlstd{\ \ \ \ \ \ \ \ }\hlstd{}\hlopt{\symbol{125}}\\
\hllin{32\ }\hlstd{}\\
\hllin{33\ }\hlstd{}\hlstd{\ \ \ \ }\hlstd{}\hlopt{\symbol{125}}\\
\hllin{34\ }\hlstd{}\hlopt{\symbol{125}}\hlstd{}
\mbox{}
\normalfont
\normalsize


mysql当中的sql语句有很多种,每一种语句都要做不同的处理,
\ref{codepdf/sqljiexi}描述了sql模块下实现的各种类型的
语句。
\pictable[htbp]{SQL模块下实现的各种类型的语句}{}{codepdf/sqljiexi}

每一种类型语句下面又回有很多种具体语句的实现,表\ref{codepdf/sqlddm}给出了数据
操纵类型语句实现的所有相关的类和响应的功能。
\pictable[htbp]{数据操纵数据实现所相关的类}{}{codepdf/sqlddm}
\subsection{存储引擎模块}
存储引擎有关的功能主要在storage包下面实现,表\ref{codepdf/storage}给出了
storage包下面各种文件的功能。
\pictable[htbp]{storage包下面各种文件的作用}{}{codepdf/storage}
其中DB类作为底层存储引擎的接口,他的接口如表\ref{codepdf/db}所示。
\pictable[htbp]{数据库存储引擎的函数接口}{}{codepdf/db}
和表有关的接口如表\ref{codepdf/table}所示。
\pictable[htbp]{数据库引擎和表有关功能的接口}{}{codepdf/table}

前面说过,本系统实现的存储引擎用的是哈希树这种数据结构。
哈希树的实现主要有2个类,一个类实现哈希树的节点,一个类实现哈希树
给外部模块提供调用接口。
下面是其中实现的关键代码。

\noindent
\ttfamily
\hlstd{}\hllin{01\ }\hlstd{}\hlkwa{public\ class\ }\hlstd{HtreeNode\ }\hlopt{\symbol{123}}\\
\hllin{02\ }\hlstd{}\hlstd{\ \ \ \ }\hlstd{}\hlkwa{public\ int\ }\hlstd{high}\hlopt{;}\hlstd{}\hlslc{//root\ is\ 0}\\
\hllin{03\ }\hlstd{}\hlstd{\ \ \ \ }\hlstd{}\hlkwa{public\ int\ }\hlstd{hashtable\symbol{95}size}\hlopt{;}\\
\hllin{04\ }\hlstd{}\hlstd{\ \ \ \ }\hlstd{}\hlkwa{public\ }\hlstd{HtreeNode}\hlopt{{[}{]}\ }\hlstd{childs}\hlopt{;}\\
\hllin{05\ }\hlstd{}\hlstd{\ \ \ \ }\hlstd{}\hlkwa{public\ boolean\ }\hlstd{hasV}\hlopt{;}\\
\hllin{06\ }\hlstd{}\hlstd{\ \ \ \ }\hlstd{}\hlkwa{public\ }\hlstd{String\ key}\hlopt{;}\\
\hllin{07\ }\hlstd{}\hlstd{\ \ \ \ }\hlstd{}\hlkwa{public\ }\hlstd{Object\ values}\hlopt{;}\\
\hllin{08\ }\hlstd{}\hlstd{\ \ \ \ }\hlstd{}\hlkwa{public\ }\hlstd{}\hlkwd{HtreeNode}\hlstd{}\hlopt{(}\hlstd{}\hlkwa{int\ }\hlstd{high}\hlopt{,\ }\hlstd{String\ key}\hlopt{,\ }\hlstd{Object\ values}\hlopt{)\ \symbol{123}}\\
\hllin{09\ }\hlstd{}\hlstd{\ \ \ \ \ \ \ \ }\hlstd{}\hlkwa{this}\hlstd{}\hlopt{.}\hlstd{high\ }\hlopt{=\ }\hlstd{high}\hlopt{;}\\
\hllin{10\ }\hlstd{}\hlstd{\ \ \ \ \ \ \ \ }\hlstd{hashtable\symbol{95}size\ }\hlopt{=\ }\hlstd{HashCodes}\hlopt{.}\hlstd{codes}\hlopt{{[}}\hlstd{high}\hlopt{{]};}\\
\hllin{11\ }\hlstd{}\hlstd{\ \ \ \ \ \ \ \ }\hlstd{}\hlkwa{this}\hlstd{}\hlopt{.}\hlstd{key\ }\hlopt{=\ }\hlstd{key}\hlopt{;}\\
\hllin{12\ }\hlstd{}\hlstd{\ \ \ \ \ \ \ \ }\hlstd{}\hlkwa{this}\hlstd{}\hlopt{.}\hlstd{values\ }\hlopt{=\ }\hlstd{values}\hlopt{;}\\
\hllin{13\ }\hlstd{}\hlstd{\ \ \ \ \ \ \ \ }\hlstd{childs\ }\hlopt{=\ }\hlstd{}\hlkwa{new\ }\hlstd{HtreeNode}\hlopt{{[}}\hlstd{hashtable\symbol{95}size}\hlopt{{]};}\\
\hllin{14\ }\hlstd{}\hlstd{\ \ \ \ \ \ \ \ }\hlstd{hasV\ }\hlopt{=\ }\hlstd{key\ }\hlopt{!=\ }\hlstd{}\hlkwb{null}\hlstd{}\hlopt{;}\\
\hllin{15\ }\hlstd{}\hlstd{\ \ \ \ }\hlstd{}\hlopt{\symbol{125}}\\
\hllin{16\ }\hlstd{}\hlopt{\symbol{125}}\hlstd{}
\mbox{}
\normalfont
\normalsize


\noindent
\ttfamily
\hlstd{}\hllin{01\ }\hlstd{}\hlkwa{public\ class\ }\hlstd{HTreeMap\ }\hlkwa{implements\ }\hlstd{Map}\hlopt{\symbol{60}}\hlstd{String}\hlopt{,\ }\hlstd{Object}\hlopt{\symbol{62}\ ,}\hlstd{Serializable}\hlopt{\symbol{123}}\\
\hllin{02\ }\hlstd{}\hlstd{\ \ \ \ }\hlstd{}\hlkwc{@Override}\\
\hllin{03\ }\hlstd{}\hlstd{\ \ \ \ }\hlstd{}\hlkwa{public\ }\hlstd{Object\ }\hlkwd{get}\hlstd{}\hlopt{(}\hlstd{Object\ key}\hlopt{)\ \symbol{123}}\\
\hllin{04\ }\hlstd{}\hlstd{\ \ \ \ \ \ \ \ }\hlstd{}\hlkwa{if\ }\hlstd{}\hlopt{(}\hlstd{key\ }\hlopt{==\ }\hlstd{}\hlkwb{null}\hlstd{}\hlopt{)\ \symbol{123}}\\
\hllin{05\ }\hlstd{}\hlstd{\ \ \ \ \ \ \ \ \ \ \ \ }\hlstd{}\hlkwa{throw\ new\ }\hlstd{}\hlkwd{NullPointerException}\hlstd{}\hlopt{(}\hlstd{}\hlstr{"key\ not\ null"}\hlstd{}\hlopt{);}\\
\hllin{06\ }\hlstd{}\hlstd{\ \ \ \ \ \ \ \ }\hlstd{}\hlopt{\symbol{125}}\\
\hllin{07\ }\hlstd{}\hlstd{\ \ \ \ \ \ \ \ }\hlstd{}\hlkwa{int\ }\hlstd{hashcode\ }\hlopt{=\ }\hlstd{Math}\hlopt{.}\hlstd{}\hlkwd{abs}\hlstd{}\hlopt{(}\hlstd{key}\hlopt{.}\hlstd{}\hlkwd{hashCode}\hlstd{}\hlopt{());}\\
\hllin{08\ }\hlstd{}\hlstd{\ \ \ \ \ \ \ \ }\hlstd{HtreeNode\ htreeNode\ }\hlopt{=\ }\hlstd{root}\hlopt{;}\\
\hllin{09\ }\hlstd{}\hlstd{\ \ \ \ \ \ \ \ }\hlstd{}\hlkwa{while\ }\hlstd{}\hlopt{(}\hlstd{htreeNode\ }\hlopt{!=\ }\hlstd{}\hlkwb{null}\hlstd{}\hlopt{)\ \symbol{123}}\\
\hllin{10\ }\hlstd{}\hlstd{\ \ \ \ \ \ \ \ \ \ \ \ }\hlstd{LOGGER}\hlopt{.}\hlstd{}\hlkwd{debug}\hlstd{}\hlopt{(}\hlstd{}\hlstr{"this\ node\ is:"}\hlstd{\ }\hlopt{+\ }\hlstd{htreeNode}\hlopt{.}\hlstd{key}\hlopt{);}\\
\hllin{11\ }\hlstd{}\hlstd{\ \ \ \ \ \ \ \ \ \ \ \ }\hlstd{}\hlkwa{if\ }\hlstd{}\hlopt{(}\hlstd{htreeNode}\hlopt{.}\hlstd{hasV\ }\hlopt{\&\&\ }\hlstd{key}\hlopt{.}\hlstd{}\hlkwd{equals}\hlstd{}\hlopt{(}\hlstd{htreeNode}\hlopt{.}\hlstd{key}\hlopt{))\ \symbol{123}}\\
\hllin{12\ }\hlstd{}\hlstd{\ \ \ \ \ \ \ \ \ \ \ \ \ \ \ \ }\hlstd{LOGGER}\hlopt{.}\hlstd{}\hlkwd{debug}\hlstd{}\hlopt{(}\hlstd{}\hlstr{"find\ "}\hlstd{\ }\hlopt{+\ }\hlstd{key}\hlopt{);}\\
\hllin{13\ }\hlstd{}\hlstd{\ \ \ \ \ \ \ \ \ \ \ \ \ \ \ \ }\hlstd{}\hlkwa{return\ }\hlstd{htreeNode}\hlopt{.}\hlstd{values}\hlopt{;}\\
\hllin{14\ }\hlstd{}\hlstd{\ \ \ \ \ \ \ \ \ \ \ \ }\hlstd{}\hlopt{\symbol{125}}\\
\hllin{15\ }\hlstd{}\hlstd{\ \ \ \ \ \ \ \ \ \ \ \ }\hlstd{}\hlkwa{int\ }\hlstd{hashindex\ }\hlopt{=\ }\hlstd{hashcode\ }\hlopt{\%\ }\hlstd{htreeNode}\hlopt{.}\hlstd{hashtable\symbol{95}size}\hlopt{;}\\
\hllin{16\ }\hlstd{}\hlstd{\ \ \ \ \ \ \ \ \ \ \ \ }\hlstd{htreeNode\ }\hlopt{=\ }\hlstd{htreeNode}\hlopt{.}\hlstd{childs}\hlopt{{[}}\hlstd{hashindex}\hlopt{{]};}\\
\hllin{17\ }\hlstd{}\hlstd{\ \ \ \ \ \ \ \ }\hlstd{}\hlopt{\symbol{125}}\\
\hllin{18\ }\hlstd{}\hlstd{\ \ \ \ \ \ \ \ }\hlstd{}\hlkwa{return\ }\hlstd{}\hlkwb{null}\hlstd{}\hlopt{;}\\
\hllin{19\ }\hlstd{}\hlstd{\ \ \ \ }\hlstd{}\hlopt{\symbol{125}}\\
\hllin{20\ }\hlstd{}\hlstd{\ \ \ \ }\hlstd{}\hlkwc{@Override}\\
\hllin{21\ }\hlstd{}\hlstd{\ \ \ \ }\hlstd{}\hlkwa{public\ }\hlstd{Object\ }\hlkwd{put}\hlstd{}\hlopt{(}\hlstd{String\ key}\hlopt{,\ }\hlstd{Object\ value}\hlopt{)\ \symbol{123}}\\
\hllin{22\ }\hlstd{}\hlstd{\ \ \ \ \ \ \ \ }\hlstd{}\hlkwa{if\ }\hlstd{}\hlopt{(}\hlstd{key\ }\hlopt{==\ }\hlstd{}\hlkwb{null\ }\hlstd{}\hlopt{\symbol{124}\symbol{124}\ }\hlstd{value\ }\hlopt{==\ }\hlstd{}\hlkwb{null}\hlstd{}\hlopt{)\ \symbol{123}}\\
\hllin{23\ }\hlstd{}\hlstd{\ \ \ \ \ \ \ \ \ \ \ \ }\hlstd{}\hlkwa{throw\ new\ }\hlstd{}\hlkwd{NullPointerException}\hlstd{}\hlopt{(}\hlstd{}\hlstr{"key\ not\ null"}\hlstd{}\hlopt{);}\\
\hllin{24\ }\hlstd{}\hlstd{\ \ \ \ \ \ \ \ }\hlstd{}\hlopt{\symbol{125}}\\
\hllin{25\ }\hlstd{}\hlstd{\ \ \ \ \ \ \ \ }\hlstd{}\hlkwa{int\ }\hlstd{hashcode\ }\hlopt{=\ }\hlstd{Math}\hlopt{.}\hlstd{}\hlkwd{abs}\hlstd{}\hlopt{(}\hlstd{key}\hlopt{.}\hlstd{}\hlkwd{hashCode}\hlstd{}\hlopt{());}\\
\hllin{26\ }\hlstd{}\hlstd{\ \ \ \ \ \ \ \ }\hlstd{HtreeNode\ htreeNode\ }\hlopt{=\ }\hlstd{root}\hlopt{;}\\
\hllin{27\ }\hlstd{}\hlstd{\ \ \ \ \ \ \ \ }\hlstd{HtreeNode\ pre\ }\hlopt{=\ }\hlstd{}\hlkwb{null}\hlstd{}\hlopt{;}\\
\hllin{28\ }\hlstd{}\hlstd{\ \ \ \ \ \ \ \ }\hlstd{}\hlkwa{while\ }\hlstd{}\hlopt{(}\hlstd{htreeNode\ }\hlopt{!=\ }\hlstd{}\hlkwb{null}\hlstd{}\hlopt{)\ \symbol{123}}\\
\hllin{29\ }\hlstd{}\hlstd{\ \ \ \ \ \ \ \ \ \ \ \ }\hlstd{}\hlkwa{if\ }\hlstd{}\hlopt{(}\hlstd{htreeNode}\hlopt{.}\hlstd{hasV\ }\hlopt{\&\&\ }\hlstd{key}\hlopt{.}\hlstd{}\hlkwd{equals}\hlstd{}\hlopt{(}\hlstd{htreeNode}\hlopt{.}\hlstd{key}\hlopt{))\ \symbol{123}}\\
\hllin{30\ }\hlstd{}\hlstd{\ \ \ \ \ \ \ \ \ \ \ \ \ \ \ \ }\hlstd{Object\ stemp\ }\hlopt{=\ }\hlstd{htreeNode}\hlopt{.}\hlstd{values}\hlopt{;}\\
\hllin{31\ }\hlstd{}\hlstd{\ \ \ \ \ \ \ \ \ \ \ \ \ \ \ \ }\hlstd{htreeNode}\hlopt{.}\hlstd{values\ }\hlopt{=\ }\hlstd{value}\hlopt{;}\\
\hllin{32\ }\hlstd{}\hlstd{\ \ \ \ \ \ \ \ \ \ \ \ \ \ \ \ }\hlstd{}\hlkwa{return\ }\hlstd{stemp}\hlopt{;}\\
\hllin{33\ }\hlstd{}\hlstd{\ \ \ \ \ \ \ \ \ \ \ \ }\hlstd{}\hlopt{\symbol{125}\ }\hlstd{}\hlkwa{else\ if\ }\hlstd{}\hlopt{(!}\hlstd{htreeNode}\hlopt{.}\hlstd{hasV}\hlopt{)\ \symbol{123}}\\
\hllin{34\ }\hlstd{}\hlstd{\ \ \ \ \ \ \ \ \ \ \ \ \ \ \ \ }\hlstd{htreeNode}\hlopt{.}\hlstd{hasV\ }\hlopt{=\ }\hlstd{}\hlkwb{true}\hlstd{}\hlopt{;}\\
\hllin{35\ }\hlstd{}\hlstd{\ \ \ \ \ \ \ \ \ \ \ \ \ \ \ \ }\hlstd{htreeNode}\hlopt{.}\hlstd{values\ }\hlopt{=\ }\hlstd{value}\hlopt{;}\\
\hllin{36\ }\hlstd{}\hlstd{\ \ \ \ \ \ \ \ \ \ \ \ \ \ \ \ }\hlstd{htreeNode}\hlopt{.}\hlstd{key\ }\hlopt{=\ }\hlstd{key}\hlopt{;}\\
\hllin{37\ }\hlstd{}\hlstd{\ \ \ \ \ \ \ \ \ \ \ \ \ \ \ \ }\hlstd{}\hlkwa{return\ }\hlstd{}\hlkwb{null}\hlstd{}\hlopt{;}\\
\hllin{38\ }\hlstd{}\hlstd{\ \ \ \ \ \ \ \ \ \ \ \ }\hlstd{}\hlopt{\symbol{125}}\\
\hllin{39\ }\hlstd{}\hlstd{\ \ \ \ \ \ \ \ \ \ \ \ }\hlstd{pre\ }\hlopt{=\ }\hlstd{htreeNode}\hlopt{;}\\
\hllin{40\ }\hlstd{}\hlstd{\ \ \ \ \ \ \ \ \ \ \ \ }\hlstd{}\hlkwa{int\ }\hlstd{hashindex\ }\hlopt{=\ }\hlstd{hashcode\ }\hlopt{\%\ }\hlstd{htreeNode}\hlopt{.}\hlstd{hashtable\symbol{95}size}\hlopt{;}\\
\hllin{41\ }\hlstd{}\hlstd{\ \ \ \ \ \ \ \ \ \ \ \ }\hlstd{htreeNode\ }\hlopt{=\ }\hlstd{htreeNode}\hlopt{.}\hlstd{childs}\hlopt{{[}}\hlstd{hashindex}\hlopt{{]};}\\
\hllin{42\ }\hlstd{}\hlstd{\ \ \ \ \ \ \ \ }\hlstd{}\hlopt{\symbol{125}}\\
\hllin{43\ }\hlstd{}\hlstd{\ \ \ \ \ \ \ \ \ \ \ \ }\hlstd{}\hlkwa{int\ }\hlstd{hashindex\ }\hlopt{=\ }\hlstd{hashcode\ }\hlopt{\%\ }\hlstd{pre}\hlopt{.}\hlstd{hashtable\symbol{95}size}\hlopt{;}\\
\hllin{44\ }\hlstd{}\hlstd{\ \ \ \ \ \ \ \ \ \ \ \ }\hlstd{pre}\hlopt{.}\hlstd{childs}\hlopt{{[}}\hlstd{hashindex}\hlopt{{]}\ =\ }\hlstd{}\hlkwa{new\ }\hlstd{}\hlkwd{HtreeNode}\hlstd{}\hlopt{(}\hlstd{pre}\hlopt{.}\hlstd{high\ }\hlopt{+\ }\hlstd{}\hlnum{1}\hlstd{}\hlopt{,\ }\hlstd{key}\hlopt{,\ }\hlstd{value}\hlopt{);}\\
\hllin{45\ }\hlstd{}\hlstd{\ \ \ \ \ \ \ \ }\hlstd{allnodes}\hlopt{.}\hlstd{}\hlkwd{add}\hlstd{}\hlopt{(}\hlstd{pre}\hlopt{.}\hlstd{childs}\hlopt{{[}}\hlstd{hashindex}\hlopt{{]});}\\
\hllin{46\ }\hlstd{}\hlstd{\ \ \ \ \ \ \ \ \ \ \ \ }\hlstd{}\hlkwa{return\ }\hlstd{}\hlkwb{null}\hlstd{}\hlopt{;}\\
\hllin{47\ }\hlstd{}\hlstd{\ \ \ \ }\hlstd{}\hlopt{\symbol{125}}\\
\hllin{48\ }\hlstd{}\hlstd{\ \ \ \ }\hlstd{}\hlkwc{@Override}\\
\hllin{49\ }\hlstd{}\hlstd{\ \ \ \ }\hlstd{}\hlkwa{public\ }\hlstd{Object\ }\hlkwd{remove}\hlstd{}\hlopt{(}\hlstd{Object\ key}\hlopt{)\ \symbol{123}}\\
\hllin{50\ }\hlstd{}\hlstd{\ \ \ \ \ \ \ \ }\hlstd{}\hlkwa{if\ }\hlstd{}\hlopt{(}\hlstd{key\ }\hlopt{==\ }\hlstd{}\hlkwb{null}\hlstd{}\hlopt{)\ \symbol{123}}\\
\hllin{51\ }\hlstd{}\hlstd{\ \ \ \ \ \ \ \ \ \ \ \ }\hlstd{}\hlkwa{throw\ new\ }\hlstd{}\hlkwd{NullPointerException}\hlstd{}\hlopt{(}\hlstd{}\hlstr{"key\ not\ null"}\hlstd{}\hlopt{);}\\
\hllin{52\ }\hlstd{}\hlstd{\ \ \ \ \ \ \ \ }\hlstd{}\hlopt{\symbol{125}}\\
\hllin{53\ }\hlstd{}\hlstd{\ \ \ \ \ \ \ \ }\hlstd{}\hlkwa{int\ }\hlstd{hashcode\ }\hlopt{=\ }\hlstd{Math}\hlopt{.}\hlstd{}\hlkwd{abs}\hlstd{}\hlopt{(}\hlstd{key}\hlopt{.}\hlstd{}\hlkwd{hashCode}\hlstd{}\hlopt{());}\\
\hllin{54\ }\hlstd{}\hlstd{\ \ \ \ \ \ \ \ }\hlstd{HtreeNode\ htreeNode\ }\hlopt{=\ }\hlstd{root}\hlopt{;}\\
\hllin{55\ }\hlstd{}\hlstd{\ \ \ \ \ \ \ \ }\hlstd{}\hlkwa{while\ }\hlstd{}\hlopt{(}\hlstd{htreeNode\ }\hlopt{!=\ }\hlstd{}\hlkwb{null}\hlstd{}\hlopt{)\ \symbol{123}}\\
\hllin{56\ }\hlstd{}\hlstd{\ \ \ \ \ \ \ \ \ \ \ \ }\hlstd{}\hlkwa{if\ }\hlstd{}\hlopt{(}\hlstd{htreeNode}\hlopt{.}\hlstd{hasV\ }\hlopt{\&\&\ }\hlstd{key}\hlopt{.}\hlstd{}\hlkwd{equals}\hlstd{}\hlopt{(}\hlstd{htreeNode}\hlopt{.}\hlstd{key}\hlopt{))\ \symbol{123}}\\
\hllin{57\ }\hlstd{}\hlstd{\ \ \ \ \ \ \ \ \ \ \ \ \ \ \ \ }\hlstd{htreeNode}\hlopt{.}\hlstd{hasV\ }\hlopt{=\ }\hlstd{}\hlkwb{false}\hlstd{}\hlopt{;}\\
\hllin{58\ }\hlstd{}\hlstd{\ \ \ \ \ \ \ \ \ \ \ \ \ \ \ \ }\hlstd{}\hlkwa{return\ }\hlstd{htreeNode}\hlopt{.}\hlstd{values}\hlopt{;}\\
\hllin{59\ }\hlstd{}\hlstd{\ \ \ \ \ \ \ \ \ \ \ \ }\hlstd{}\hlopt{\symbol{125}}\\
\hllin{60\ }\hlstd{}\hlstd{\ \ \ \ \ \ \ \ \ \ \ \ }\hlstd{}\hlkwa{int\ }\hlstd{hashindex\ }\hlopt{=\ }\hlstd{hashcode\ }\hlopt{\%\ }\hlstd{htreeNode}\hlopt{.}\hlstd{hashtable\symbol{95}size}\hlopt{;}\\
\hllin{61\ }\hlstd{}\hlstd{\ \ \ \ \ \ \ \ \ \ \ \ }\hlstd{htreeNode\ }\hlopt{=\ }\hlstd{htreeNode}\hlopt{.}\hlstd{childs}\hlopt{{[}}\hlstd{hashindex}\hlopt{{]};}\\
\hllin{62\ }\hlstd{}\hlstd{\ \ \ \ \ \ \ \ }\hlstd{}\hlopt{\symbol{125}}\\
\hllin{63\ }\hlstd{}\hlstd{\ \ \ \ \ \ \ \ }\hlstd{}\hlkwa{return\ }\hlstd{}\hlkwb{null}\hlstd{}\hlopt{;}\\
\hllin{64\ }\hlstd{}\hlstd{\ \ \ \ }\hlstd{}\hlopt{\symbol{125}}\\
\hllin{65\ }\hlstd{}\hlopt{\symbol{125}}\hlstd{}
\mbox{}
\normalfont
\normalsize


除了实现数据结构以外还要实现存储引擎,也就是把数据存储在文件系统当中,
其中主要用到内存映射文件,下面是其中的关键代码。

\noindent
\ttfamily
\hlstd{}\hllin{01\ }\hlstd{}\hlcom{/{*}{*}}\\
\hllin{02\ }\hlcom{\ {*}\ The\ type\ M\ storage.}\\
\hllin{03\ }\hlcom{\ {*}\ 文件前面2m保留做头信息1024{*}1024{*}2/4096=512页面}\\
\hllin{04\ }\hlcom{\ {*}/}\hlstd{}\\
\hllin{05\ }\hlstd{}\hlkwa{class\ }\hlstd{MStorage\ }\hlopt{\symbol{123}}\\
\hllin{06\ }\hlstd{}\hlstd{\ \ \ \ }\hlstd{}\hlkwa{public\ void\ }\hlstd{}\hlkwd{write}\hlstd{}\hlopt{(}\hlstd{}\hlkwa{long\ }\hlstd{pageNumber}\hlopt{,\ }\hlstd{ByteBuffer\ data}\hlopt{)\ }\hlstd{}\hlkwa{throws\ }\hlstd{IOException\ }\hlopt{\symbol{123}}\\
\hllin{07\ }\hlstd{}\hlstd{\ \ \ \ \ \ \ \ }\hlstd{FileChannel\ f\ }\hlopt{=\ }\hlstd{}\hlkwd{getChannel}\hlstd{}\hlopt{(}\hlstd{pageNumber}\hlopt{);}\\
\hllin{08\ }\hlstd{}\hlstd{\ \ \ \ \ \ \ \ }\hlstd{}\hlkwa{int\ }\hlstd{offsetInFile\ }\hlopt{=\ (}\hlstd{}\hlkwa{int}\hlstd{}\hlopt{)\ ((}\hlstd{Math}\hlopt{.}\hlstd{}\hlkwd{abs}\hlstd{}\hlopt{(}\hlstd{pageNumber}\hlopt{)\ \%\ }\hlstd{Pagesize}\hlopt{.}\hlstd{max\symbol{95}page\symbol{95}number}\hlopt{)\ {*}\ }\hlstd{Pagesize}\hlopt{.}\hlstd{page\symbol{95}size}\hlopt{);}\\
\hllin{09\ }\hlstd{}\hlstd{\ \ \ \ \ \ \ \ }\hlstd{MappedByteBuffer\ b\ }\hlopt{=\ }\hlstd{buffers}\hlopt{.}\hlstd{}\hlkwd{get}\hlstd{}\hlopt{(}\hlstd{f}\hlopt{);}\\
\hllin{10\ }\hlstd{}\hlstd{\ \ \ \ \ \ \ \ }\hlstd{}\hlkwa{if\ }\hlstd{}\hlopt{(}\hlstd{b}\hlopt{.}\hlstd{}\hlkwd{limit}\hlstd{}\hlopt{()\ \symbol{60}=\ }\hlstd{offsetInFile}\hlopt{)\ \symbol{123}}\\
\hllin{11\ }\hlstd{}\hlstd{\ \ \ \ \ \ \ \ \ \ \ \ }\hlstd{b\ }\hlopt{=\ }\hlstd{}\hlkwd{addfilesize}\hlstd{}\hlopt{(}\hlstd{f}\hlopt{,\ }\hlstd{offsetInFile}\hlopt{,\ }\hlstd{b}\hlopt{);}\\
\hllin{12\ }\hlstd{}\hlstd{\ \ \ \ \ \ \ \ }\hlstd{}\hlopt{\symbol{125}}\\
\hllin{13\ }\hlstd{}\hlstd{\ \ \ \ \ \ \ \ }\hlstd{}\hlslc{//write\ into\ buffer}\\
\hllin{14\ }\hlstd{}\hlstd{\ \ \ \ \ \ \ \ }\hlstd{b}\hlopt{.}\hlstd{}\hlkwd{position}\hlstd{}\hlopt{(}\hlstd{offsetInFile}\hlopt{);}\\
\hllin{15\ }\hlstd{}\hlstd{\ \ \ \ \ \ \ \ }\hlstd{data}\hlopt{.}\hlstd{}\hlkwd{rewind}\hlstd{}\hlopt{();}\\
\hllin{16\ }\hlstd{}\hlstd{\ \ \ \ \ \ \ \ }\hlstd{b}\hlopt{.}\hlstd{}\hlkwd{put}\hlstd{}\hlopt{(}\hlstd{data}\hlopt{);}\\
\hllin{17\ }\hlstd{}\hlstd{\ \ \ \ }\hlstd{}\hlopt{\symbol{125}}\\
\hllin{18\ }\hlstd{}\hlstd{\ \ \ \ }\hlstd{}\hlkwa{public\ }\hlstd{ByteBuffer\ }\hlkwd{read}\hlstd{}\hlopt{(}\hlstd{}\hlkwa{long\ }\hlstd{pageNumber}\hlopt{)\ }\hlstd{}\hlkwa{throws\ }\hlstd{IOException\ }\hlopt{\symbol{123}}\\
\hllin{19\ }\hlstd{}\hlstd{\ \ \ \ \ \ \ \ }\hlstd{FileChannel\ f\ }\hlopt{=\ }\hlstd{}\hlkwd{getChannel}\hlstd{}\hlopt{(}\hlstd{pageNumber}\hlopt{);}\\
\hllin{20\ }\hlstd{}\hlstd{\ \ \ \ \ \ \ \ }\hlstd{}\hlkwa{int\ }\hlstd{offsetInFile\ }\hlopt{=\ (}\hlstd{}\hlkwa{int}\hlstd{}\hlopt{)\ ((}\hlstd{Math}\hlopt{.}\hlstd{}\hlkwd{abs}\hlstd{}\hlopt{(}\hlstd{pageNumber}\hlopt{)\ \%\ }\hlstd{Pagesize}\hlopt{.}\hlstd{max\symbol{95}page\symbol{95}number}\hlopt{)\ {*}\ }\hlstd{Pagesize}\hlopt{.}\hlstd{page\symbol{95}size}\hlopt{);}\\
\hllin{21\ }\hlstd{}\hlstd{\ \ \ \ \ \ \ \ }\hlstd{MappedByteBuffer\ b\ }\hlopt{=\ }\hlstd{buffers}\hlopt{.}\hlstd{}\hlkwd{get}\hlstd{}\hlopt{(}\hlstd{f}\hlopt{);}\\
\hllin{22\ }\hlstd{}\\
\hllin{23\ }\hlstd{}\hlstd{\ \ \ \ \ \ \ \ }\hlstd{}\hlkwa{if\ }\hlstd{}\hlopt{(}\hlstd{b\ }\hlopt{==\ }\hlstd{}\hlkwb{null}\hlstd{}\hlopt{)\ \symbol{123}\ }\hlstd{}\hlslc{//not\ mapped\ yet}\\
\hllin{24\ }\hlstd{}\hlstd{\ \ \ \ \ \ \ \ \ \ \ \ }\hlstd{b\ }\hlopt{=\ }\hlstd{f}\hlopt{.}\hlstd{}\hlkwd{map}\hlstd{}\hlopt{(}\hlstd{FileChannel}\hlopt{.}\hlstd{MapMode}\hlopt{.}\hlstd{READ\symbol{95}WRITE}\hlopt{,\ }\hlstd{}\hlnum{0}\hlstd{}\hlopt{,\ }\hlstd{f}\hlopt{.}\hlstd{}\hlkwd{size}\hlstd{}\hlopt{());}\\
\hllin{25\ }\hlstd{}\hlstd{\ \ \ \ \ \ \ \ }\hlstd{}\hlopt{\symbol{125}}\\
\hllin{26\ }\hlstd{}\hlstd{\ \ \ \ \ \ \ \ }\hlstd{}\hlslc{//增加文件大小,64m为单位}\\
\hllin{27\ }\hlstd{}\hlstd{\ \ \ \ \ \ \ \ }\hlstd{}\hlkwa{if\ }\hlstd{}\hlopt{(}\hlstd{b}\hlopt{.}\hlstd{}\hlkwd{limit}\hlstd{}\hlopt{()\ \symbol{60}=\ }\hlstd{offsetInFile}\hlopt{)\ \symbol{123}}\\
\hllin{28\ }\hlstd{}\hlstd{\ \ \ \ \ \ \ \ \ \ \ \ }\hlstd{b\ }\hlopt{=\ }\hlstd{}\hlkwd{addfilesize}\hlstd{}\hlopt{(}\hlstd{f}\hlopt{,\ }\hlstd{offsetInFile}\hlopt{,\ }\hlstd{b}\hlopt{);}\\
\hllin{29\ }\hlstd{}\hlstd{\ \ \ \ \ \ \ \ }\hlstd{}\hlopt{\symbol{125}}\\
\hllin{30\ }\hlstd{}\hlstd{\ \ \ \ \ \ \ \ }\hlstd{b}\hlopt{.}\hlstd{}\hlkwd{position}\hlstd{}\hlopt{(}\hlstd{offsetInFile}\hlopt{);}\\
\hllin{31\ }\hlstd{}\hlstd{\ \ \ \ \ \ \ \ }\hlstd{ByteBuffer\ ret\ }\hlopt{=\ }\hlstd{b}\hlopt{.}\hlstd{}\hlkwd{slice}\hlstd{}\hlopt{();}\\
\hllin{32\ }\hlstd{}\hlstd{\ \ \ \ \ \ \ \ }\hlstd{ret}\hlopt{.}\hlstd{}\hlkwd{limit}\hlstd{}\hlopt{(}\hlstd{Pagesize}\hlopt{.}\hlstd{page\symbol{95}size}\hlopt{);}\\
\hllin{33\ }\hlstd{}\hlstd{\ \ \ \ \ \ \ \ }\hlstd{}\hlkwa{if\ }\hlstd{}\hlopt{(!}\hlstd{transactionsDisabled\ }\hlopt{\symbol{124}\symbol{124}\ }\hlstd{readonly}\hlopt{)\ \symbol{123}}\\
\hllin{34\ }\hlstd{}\hlstd{\ \ \ \ \ \ \ \ \ \ \ \ }\hlstd{}\hlslc{//\ changes\ written\ into\ buffer\ will\ be\ directly\ written\ into\ file}\\
\hllin{35\ }\hlstd{}\hlstd{\ \ \ \ \ \ \ \ \ \ \ \ }\hlstd{}\hlslc{//\ so\ we\ need\ to\ protect\ buffer\ from\ modifications}\\
\hllin{36\ }\hlstd{}\hlstd{\ \ \ \ \ \ \ \ \ \ \ \ }\hlstd{ret\ }\hlopt{=\ }\hlstd{ret}\hlopt{.}\hlstd{}\hlkwd{asReadOnlyBuffer}\hlstd{}\hlopt{();}\\
\hllin{37\ }\hlstd{}\hlstd{\ \ \ \ \ \ \ \ }\hlstd{}\hlopt{\symbol{125}}\\
\hllin{38\ }\hlstd{}\hlstd{\ \ \ \ \ \ \ \ }\hlstd{}\hlkwa{return\ }\hlstd{ret}\hlopt{;}\\
\hllin{39\ }\hlstd{}\hlstd{\ \ \ \ }\hlstd{}\hlopt{\symbol{125}}\\
\hllin{40\ }\hlstd{}\hlopt{\symbol{125}}\hlstd{}
\mbox{}
\normalfont
\normalsize


\noindent
\ttfamily
\hlstd{}\hllin{01\ }\hlstd{}\hlcom{/{*}{*}}\\
\hllin{02\ }\hlcom{\ {*}\ 和内存指针差不多,new\ 后得到地址,这里地址是page\ index}\\
\hllin{03\ }\hlcom{\ {*}\ 每个页面开始分别是type,记录大小,数据,页面后2个字节用来连接每个页面}\\
\hllin{04\ }\hlcom{\ {*}\ 一个页面pagesize{-}2{-}4{-}4}\\
\hllin{05\ }\hlcom{\ {*}/}\hlstd{}\\
\hllin{06\ }\hlstd{}\hlkwa{public\ class\ }\hlstd{DiscIO\ }\hlkwa{implements\ }\hlstd{MdiscIO\ }\hlopt{\symbol{123}}\\
\hllin{07\ }\hlstd{}\hlstd{\ \ \ \ }\hlstd{}\hlkwc{@Override}\\
\hllin{08\ }\hlstd{}\hlstd{\ \ \ \ }\hlstd{}\hlkwa{public\ int\ }\hlstd{}\hlkwd{write}\hlstd{}\hlopt{(}\hlstd{Object\ o}\hlopt{)\ \symbol{123}}\\
\hllin{09\ }\hlstd{}\hlstd{\ \ \ \ \ \ \ \ }\hlstd{}\hlkwa{byte}\hlstd{}\hlopt{{[}{]}\ }\hlstd{bytes\ }\hlopt{=\ }\hlstd{ObjectSeriaer}\hlopt{.}\hlstd{}\hlkwd{getbytes}\hlstd{}\hlopt{(}\hlstd{o}\hlopt{);}\\
\hllin{10\ }\hlstd{}\hlstd{\ \ \ \ \ \ \ \ }\hlstd{}\hlkwa{int}\hlstd{}\hlopt{{[}{]}\ }\hlstd{pages\ }\hlopt{=\ }\hlstd{pagemanager}\hlopt{.}\hlstd{}\hlkwd{getfreepanages}\hlstd{}\hlopt{(}\hlstd{bytes}\hlopt{.}\hlstd{length}\hlopt{);}\\
\hllin{11\ }\hlstd{}\hlstd{\ \ \ \ \ \ \ \ }\hlstd{}\hlkwa{if\ }\hlstd{}\hlopt{(}\hlstd{pages}\hlopt{.}\hlstd{length\ }\hlopt{==\ }\hlstd{}\hlnum{1}\hlstd{}\hlopt{)\ \symbol{123}}\\
\hllin{12\ }\hlstd{}\hlstd{\ \ \ \ \ \ \ \ \ \ \ \ }\hlstd{}\hlkwa{try\ }\hlstd{}\hlopt{\symbol{123}}\\
\hllin{13\ }\hlstd{}\hlstd{\ \ \ \ \ \ \ \ \ \ \ \ \ \ \ \ }\hlstd{ByteBuffer\ buffer\ }\hlopt{=\ }\hlstd{storage}\hlopt{.}\hlstd{}\hlkwd{read}\hlstd{}\hlopt{(}\hlstd{pages}\hlopt{{[}}\hlstd{}\hlnum{0}\hlstd{}\hlopt{{]});}\\
\hllin{14\ }\hlstd{}\hlstd{\ \ \ \ \ \ \ \ \ \ \ \ \ \ \ \ }\hlstd{}\hlkwa{if\ }\hlstd{}\hlopt{(}\hlstd{o\ }\hlkwa{instanceof\ }\hlstd{DHtree}\hlopt{)}\\
\hllin{15\ }\hlstd{}\hlstd{\ \ \ \ \ \ \ \ \ \ \ \ \ \ \ \ \ \ \ \ }\hlstd{buffer}\hlopt{.}\hlstd{}\hlkwd{putShort}\hlstd{}\hlopt{(}\hlstd{Pagesize}\hlopt{.}\hlstd{pagehead\symbol{95}tree}\hlopt{);}\\
\hllin{16\ }\hlstd{}\hlstd{\ \ \ \ \ \ \ \ \ \ \ \ \ \ \ \ }\hlstd{}\hlkwa{else\ if\ }\hlstd{}\hlopt{(}\hlstd{o\ }\hlkwa{instanceof\ }\hlstd{DHtreeNode}\hlopt{)\ \symbol{123}}\\
\hllin{17\ }\hlstd{}\hlstd{\ \ \ \ \ \ \ \ \ \ \ \ \ \ \ \ \ \ \ \ }\hlstd{buffer}\hlopt{.}\hlstd{}\hlkwd{putShort}\hlstd{}\hlopt{(}\hlstd{Pagesize}\hlopt{.}\hlstd{pagehead\symbol{95}node}\hlopt{);}\\
\hllin{18\ }\hlstd{}\hlstd{\ \ \ \ \ \ \ \ \ \ \ \ \ \ \ \ }\hlstd{}\hlopt{\symbol{125}\ }\hlstd{}\hlkwa{else\ }\hlstd{}\hlopt{\symbol{123}}\\
\hllin{19\ }\hlstd{}\hlstd{\ \ \ \ \ \ \ \ \ \ \ \ \ \ \ \ \ \ \ \ }\hlstd{buffer}\hlopt{.}\hlstd{}\hlkwd{putShort}\hlstd{}\hlopt{(}\hlstd{Pagesize}\hlopt{.}\hlstd{pagehead\symbol{95}other}\hlopt{);}\\
\hllin{20\ }\hlstd{}\hlstd{\ \ \ \ \ \ \ \ \ \ \ \ \ \ \ \ }\hlstd{}\hlopt{\symbol{125}}\\
\hllin{21\ }\hlstd{}\hlstd{\ \ \ \ \ \ \ \ \ \ \ \ \ \ \ \ }\hlstd{buffer}\hlopt{.}\hlstd{}\hlkwd{putInt}\hlstd{}\hlopt{(}\hlstd{bytes}\hlopt{.}\hlstd{length}\hlopt{);}\\
\hllin{22\ }\hlstd{}\hlstd{\ \ \ \ \ \ \ \ \ \ \ \ \ \ \ \ }\hlstd{buffer}\hlopt{.}\hlstd{}\hlkwd{put}\hlstd{}\hlopt{(}\hlstd{bytes}\hlopt{);}\\
\hllin{23\ }\hlstd{}\hlstd{\ \ \ \ \ \ \ \ \ \ \ \ \ \ \ \ }\hlstd{storage}\hlopt{.}\hlstd{}\hlkwd{write}\hlstd{}\hlopt{(}\hlstd{pages}\hlopt{{[}}\hlstd{}\hlnum{0}\hlstd{}\hlopt{{]},\ }\hlstd{buffer}\hlopt{);}\\
\hllin{24\ }\hlstd{}\hlstd{\ \ \ \ \ \ \ \ \ \ \ \ \ \ \ \ }\hlstd{ObjectMap}\hlopt{.}\hlstd{}\hlkwd{putorupdate}\hlstd{}\hlopt{(}\hlstd{o}\hlopt{,\ }\hlstd{pages}\hlopt{{[}}\hlstd{}\hlnum{0}\hlstd{}\hlopt{{]});}\\
\hllin{25\ }\hlstd{}\hlstd{\ \ \ \ \ \ \ \ \ \ \ \ \ \ \ \ }\hlstd{}\hlkwa{return\ }\hlstd{pages}\hlopt{{[}}\hlstd{}\hlnum{0}\hlstd{}\hlopt{{]};}\\
\hllin{26\ }\hlstd{}\hlstd{\ \ \ \ \ \ \ \ \ \ \ \ }\hlstd{}\hlopt{\symbol{125}\ }\hlstd{}\hlkwa{catch\ }\hlstd{}\hlopt{(}\hlstd{IOException\ e}\hlopt{)\ \symbol{123}}\\
\hllin{27\ }\hlstd{}\hlstd{\ \ \ \ \ \ \ \ \ \ \ \ \ \ \ \ }\hlstd{e}\hlopt{.}\hlstd{}\hlkwd{printStackTrace}\hlstd{}\hlopt{();}\\
\hllin{28\ }\hlstd{}\hlstd{\ \ \ \ \ \ \ \ \ \ \ \ }\hlstd{}\hlopt{\symbol{125}}\\
\hllin{29\ }\hlstd{}\hlstd{\ \ \ \ \ \ \ \ }\hlstd{}\hlopt{\symbol{125}\ }\hlstd{}\hlkwa{else\ }\hlstd{}\hlopt{\symbol{123}}\\
\hllin{30\ }\hlstd{}\hlstd{\ \ \ \ \ \ \ \ \ \ \ \ }\hlstd{}\hlkwa{return\ }\hlstd{}\hlkwd{writemorepage}\hlstd{}\hlopt{(}\hlstd{bytes}\hlopt{,\ }\hlstd{pages}\hlopt{,\ }\hlstd{o}\hlopt{);}\\
\hllin{31\ }\hlstd{}\hlstd{\ \ \ \ \ \ \ \ }\hlstd{}\hlopt{\symbol{125}}\\
\hllin{32\ }\hlstd{}\hlstd{\ \ \ \ \ \ \ \ }\hlstd{}\hlkwa{return\ }\hlstd{}\hlnum{0}\hlstd{}\hlopt{;}\\
\hllin{33\ }\hlstd{}\hlstd{\ \ \ \ }\hlstd{}\hlopt{\symbol{125}}\\
\hllin{34\ }\hlstd{}\hlstd{\ \ \ \ }\hlstd{}\hlkwc{@Override}\\
\hllin{35\ }\hlstd{}\hlstd{\ \ \ \ }\hlstd{}\hlkwa{public\ int\ }\hlstd{}\hlkwd{update}\hlstd{}\hlopt{(}\hlstd{Object\ o}\hlopt{,\ }\hlstd{}\hlkwa{int\ }\hlstd{recid}\hlopt{)\ \symbol{123}}\\
\hllin{36\ }\hlstd{}\hlstd{\ \ \ \ \ \ \ \ }\hlstd{}\hlkwa{if\ }\hlstd{}\hlopt{(!}\hlstd{ObjectMap}\hlopt{.}\hlstd{map}\hlopt{.}\hlstd{}\hlkwd{containsKey}\hlstd{}\hlopt{(}\hlstd{recid}\hlopt{))\ \symbol{123}}\\
\hllin{37\ }\hlstd{}\hlstd{\ \ \ \ \ \ \ \ \ \ \ \ }\hlstd{}\hlkwa{return\ }\hlstd{}\hlopt{{-}}\hlstd{}\hlnum{1}\hlstd{}\hlopt{;}\\
\hllin{38\ }\hlstd{}\hlstd{\ \ \ \ \ \ \ \ }\hlstd{}\hlopt{\symbol{125}}\\
\hllin{39\ }\hlstd{}\hlstd{\ \ \ \ \ \ \ \ }\hlstd{}\hlkwa{byte}\hlstd{}\hlopt{{[}{]}\ }\hlstd{bytes\ }\hlopt{=\ }\hlstd{ObjectSeriaer}\hlopt{.}\hlstd{}\hlkwd{getbytes}\hlstd{}\hlopt{(}\hlstd{o}\hlopt{);}\\
\hllin{40\ }\hlstd{}\hlstd{\ \ \ \ \ \ \ \ }\hlstd{}\hlkwa{if\ }\hlstd{}\hlopt{(}\hlstd{bytes}\hlopt{.}\hlstd{length\ }\hlopt{\symbol{60}=\ }\hlstd{Pagesize}\hlopt{.}\hlstd{page\symbol{95}size\symbol{95}for\symbol{95}content}\hlopt{)\ \symbol{123}}\\
\hllin{41\ }\hlstd{}\hlstd{\ \ \ \ \ \ \ \ \ \ \ \ }\hlstd{}\hlkwa{try\ }\hlstd{}\hlopt{\symbol{123}}\\
\hllin{42\ }\hlstd{}\hlstd{\ \ \ \ \ \ \ \ \ \ \ \ \ \ \ \ }\hlstd{ByteBuffer\ buffer\ }\hlopt{=\ }\hlstd{storage}\hlopt{.}\hlstd{}\hlkwd{read}\hlstd{}\hlopt{(}\hlstd{recid}\hlopt{);}\\
\hllin{43\ }\hlstd{}\hlstd{\ \ \ \ \ \ \ \ \ \ \ \ \ \ \ \ }\hlstd{}\hlkwa{if\ }\hlstd{}\hlopt{(}\hlstd{o\ }\hlkwa{instanceof\ }\hlstd{DHtree}\hlopt{)}\\
\hllin{44\ }\hlstd{}\hlstd{\ \ \ \ \ \ \ \ \ \ \ \ \ \ \ \ \ \ \ \ }\hlstd{buffer}\hlopt{.}\hlstd{}\hlkwd{putShort}\hlstd{}\hlopt{(}\hlstd{Pagesize}\hlopt{.}\hlstd{pagehead\symbol{95}tree}\hlopt{);}\\
\hllin{45\ }\hlstd{}\hlstd{\ \ \ \ \ \ \ \ \ \ \ \ \ \ \ \ }\hlstd{}\hlkwa{else\ if\ }\hlstd{}\hlopt{(}\hlstd{o\ }\hlkwa{instanceof\ }\hlstd{DHtreeNode}\hlopt{)\ \symbol{123}}\\
\hllin{46\ }\hlstd{}\hlstd{\ \ \ \ \ \ \ \ \ \ \ \ \ \ \ \ \ \ \ \ }\hlstd{buffer}\hlopt{.}\hlstd{}\hlkwd{putShort}\hlstd{}\hlopt{(}\hlstd{Pagesize}\hlopt{.}\hlstd{pagehead\symbol{95}node}\hlopt{);}\\
\hllin{47\ }\hlstd{}\hlstd{\ \ \ \ \ \ \ \ \ \ \ \ \ \ \ \ }\hlstd{}\hlopt{\symbol{125}\ }\hlstd{}\hlkwa{else\ }\hlstd{}\hlopt{\symbol{123}}\\
\hllin{48\ }\hlstd{}\hlstd{\ \ \ \ \ \ \ \ \ \ \ \ \ \ \ \ \ \ \ \ }\hlstd{buffer}\hlopt{.}\hlstd{}\hlkwd{putShort}\hlstd{}\hlopt{(}\hlstd{Pagesize}\hlopt{.}\hlstd{pagehead\symbol{95}other}\hlopt{);}\\
\hllin{49\ }\hlstd{}\hlstd{\ \ \ \ \ \ \ \ \ \ \ \ \ \ \ \ }\hlstd{}\hlopt{\symbol{125}}\\
\hllin{50\ }\hlstd{}\hlstd{\ \ \ \ \ \ \ \ \ \ \ \ \ \ \ \ }\hlstd{buffer}\hlopt{.}\hlstd{}\hlkwd{putInt}\hlstd{}\hlopt{(}\hlstd{bytes}\hlopt{.}\hlstd{length}\hlopt{);}\\
\hllin{51\ }\hlstd{}\hlstd{\ \ \ \ \ \ \ \ \ \ \ \ \ \ \ \ }\hlstd{buffer}\hlopt{.}\hlstd{}\hlkwd{put}\hlstd{}\hlopt{(}\hlstd{bytes}\hlopt{);}\\
\hllin{52\ }\hlstd{}\hlstd{\ \ \ \ \ \ \ \ \ \ \ \ \ \ \ \ }\hlstd{storage}\hlopt{.}\hlstd{}\hlkwd{write}\hlstd{}\hlopt{(}\hlstd{recid}\hlopt{,\ }\hlstd{buffer}\hlopt{);}\\
\hllin{53\ }\hlstd{}\hlstd{\ \ \ \ \ \ \ \ \ \ \ \ \ \ \ \ }\hlstd{ObjectMap}\hlopt{.}\hlstd{}\hlkwd{putorupdate}\hlstd{}\hlopt{(}\hlstd{o}\hlopt{,\ }\hlstd{recid}\hlopt{);}\\
\hllin{54\ }\hlstd{}\hlstd{\ \ \ \ \ \ \ \ \ \ \ \ \ \ \ \ }\hlstd{}\hlkwa{return\ }\hlstd{recid}\hlopt{;}\\
\hllin{55\ }\hlstd{}\hlstd{\ \ \ \ \ \ \ \ \ \ \ \ }\hlstd{}\hlopt{\symbol{125}\ }\hlstd{}\hlkwa{catch\ }\hlstd{}\hlopt{(}\hlstd{IOException\ e}\hlopt{)\ \symbol{123}}\\
\hllin{56\ }\hlstd{}\hlstd{\ \ \ \ \ \ \ \ \ \ \ \ \ \ \ \ }\hlstd{e}\hlopt{.}\hlstd{}\hlkwd{printStackTrace}\hlstd{}\hlopt{();}\\
\hllin{57\ }\hlstd{}\hlstd{\ \ \ \ \ \ \ \ \ \ \ \ \ \ \ \ }\hlstd{}\hlkwa{return\ }\hlstd{}\hlopt{{-}}\hlstd{}\hlnum{1}\hlstd{}\hlopt{;}\\
\hllin{58\ }\hlstd{}\hlstd{\ \ \ \ \ \ \ \ \ \ \ \ }\hlstd{}\hlopt{\symbol{125}}\\
\hllin{59\ }\hlstd{}\hlstd{\ \ \ \ \ \ \ \ }\hlstd{}\hlopt{\symbol{125}\ }\hlstd{}\hlkwa{else\ }\hlstd{}\hlopt{\symbol{123}}\\
\hllin{60\ }\hlstd{}\hlstd{\ \ \ \ \ \ \ \ \ \ \ \ }\hlstd{}\hlkwd{updatemoredata}\hlstd{}\hlopt{(}\hlstd{bytes}\hlopt{,\ }\hlstd{recid}\hlopt{,\ }\hlstd{o}\hlopt{);}\\
\hllin{61\ }\hlstd{}\hlstd{\ \ \ \ \ \ \ \ \ \ \ \ }\hlstd{}\hlkwa{return\ }\hlstd{recid}\hlopt{;}\\
\hllin{62\ }\hlstd{}\hlstd{\ \ \ \ \ \ \ \ }\hlstd{}\hlopt{\symbol{125}}\\
\hllin{63\ }\hlstd{}\hlstd{\ \ \ \ }\hlstd{}\hlopt{\symbol{125}}\\
\hllin{64\ }\hlstd{}\hlopt{\symbol{125}}\hlstd{}\\
\mbox{}
\normalfont
\normalsize


\section{集群架构的实现}
集群功能主要是利用了开源的hazelcast框架来实现,
其中的功能全在hazelcast包下面实现,表\ref{codepdf/hazelcast}给出了
该包下面每个类的具体的作用。
\pictable[htbp]{集群模块下面各个类的作用}{}{codepdf/hazelcast}
其中最关键的代码如下。

\noindent
\ttfamily
\hlstd{}\hllin{01\ }\hlstd{}\hlcom{/{*}{*}}\\
\hllin{02\ }\hlcom{\ {*}\ Created\ by\ 长宏\ on\ 2017/5/5\ 0005.}\\
\hllin{03\ }\hlcom{\ {*}\ sql队列}\\
\hllin{04\ }\hlcom{\ {*}\ 复制队列}\\
\hllin{05\ }\hlcom{\ {*}\ 命令队列。}\\
\hllin{06\ }\hlcom{\ {*}\ 2个锁。}\\
\hllin{07\ }\hlcom{\ {*}\ 发布}\\
\hllin{08\ }\hlcom{\ {*}/}\hlstd{}\\
\hllin{09\ }\hlstd{}\hlkwc{@Component}\\
\hllin{10\ }\hlstd{}\hlkwa{class\ }\hlstd{MyHazelcast\ }\hlopt{:\ }\hlstd{ItemListener}\hlopt{\symbol{60}}\hlstd{SqlUpdateLog}\hlopt{\symbol{62}\ \symbol{123}}\\
\hllin{11\ }\hlstd{}\hlstd{\ \ \ \ }\hlstd{}\hlkwc{@Volatile\ }\hlstd{}\hlkwa{private\ var\ }\hlstd{isreplicating}\hlopt{:\ }\hlstd{}\hlkwb{Boolean\ }\hlstd{}\hlopt{=\ }\hlstd{}\hlkwa{false}\\
\hllin{12\ }\hlstd{}\hlstd{\ \ \ \ }\hlstd{}\hlkwa{private\ val\ }\hlstd{remotequenelister\ }\hlopt{=\ }\hlstd{}\hlkwd{replicationlister}\hlstd{}\hlopt{()}\\
\hllin{13\ }\hlstd{}\hlstd{\ \ \ \ }\hlstd{}\hlkwa{var\ }\hlstd{hazelcastInstance}\hlopt{:\ }\hlstd{HazelcastInstance?\ }\hlopt{=\ }\hlstd{}\hlkwa{null}\\
\hllin{14\ }\hlstd{}\hlstd{\ \ \ \ }\hlstd{}\hlkwa{private\ val\ }\hlstd{localquene\ }\hlopt{=\ }\hlstd{LinkedList}\hlopt{\symbol{60}}\hlstd{SqlUpdateLog}\hlopt{\symbol{62}()}\\
\hllin{15\ }\hlstd{}\hlstd{\ \ \ \ }\hlstd{}\hlkwa{private\ val\ }\hlstd{localqueneReplication\ }\hlopt{=\ }\hlstd{LinkedList}\hlopt{\symbol{60}}\hlstd{SqlUpdateLog}\hlopt{\symbol{62}()}\\
\hllin{16\ }\hlstd{}\hlstd{\ \ \ \ }\hlstd{}\hlkwa{private\ var\ }\hlstd{remotequene}\hlopt{:\ }\hlstd{IQueue}\hlopt{\symbol{60}}\hlstd{SqlUpdateLog}\hlopt{\symbol{62}}\hlstd{?\ }\hlopt{=\ }\hlstd{}\hlkwa{null}\\
\hllin{17\ }\hlstd{}\hlstd{\ \ \ \ }\hlstd{}\hlkwa{private\ var\ }\hlstd{replicationQuene}\hlopt{:\ }\hlstd{IQueue}\hlopt{\symbol{60}}\hlstd{SqlUpdateLog}\hlopt{\symbol{62}}\hlstd{?\ }\hlopt{=\ }\hlstd{}\hlkwa{null}\\
\hllin{18\ }\hlstd{}\hlstd{\ \ \ \ }\hlstd{}\hlkwa{private\ var\ }\hlstd{cmdiQueue}\hlopt{:\ }\hlstd{IQueue}\hlopt{\symbol{60}}\hlstd{ReplicationCMD}\hlopt{\symbol{62}}\hlstd{?\ }\hlopt{=\ }\hlstd{}\hlkwa{null}\\
\hllin{19\ }\hlstd{}\hlstd{\ \ \ \ }\hlstd{}\hlkwa{private\ var\ }\hlstd{cmdItemListener}\hlopt{:\ }\hlstd{ItemListener}\hlopt{\symbol{60}}\hlstd{ReplicationCMD}\hlopt{\symbol{62}}\hlstd{?\ }\hlopt{=\ }\hlstd{}\hlkwa{null}\\
\hllin{20\ }\hlstd{}\hlstd{\ \ \ \ }\hlstd{}\hlkwa{private\ var\ }\hlstd{idGenerator}\hlopt{:\ }\hlstd{IdGenerator?\ }\hlopt{=\ }\hlstd{}\hlkwa{null}\\
\hllin{21\ }\hlstd{}\hlstd{\ \ \ \ }\hlstd{}\hlkwa{private\ var\ }\hlstd{id\symbol{95}froCMD}\hlopt{:\ }\hlstd{}\hlkwb{Long\ }\hlstd{}\hlopt{=\ {-}}\hlstd{}\hlnum{1}\\
\hllin{22\ }\hlstd{}\hlstd{\ \ \ \ }\hlstd{}\hlkwa{internal\ var\ }\hlstd{iLockwrite}\hlopt{:\ }\hlstd{ILock?\ }\hlopt{=\ }\hlstd{}\hlkwa{null}\\
\hllin{23\ }\hlstd{}\hlstd{\ \ \ \ }\hlstd{}\hlkwa{internal\ var\ }\hlstd{iLockread}\hlopt{:\ }\hlstd{ILock?\ }\hlopt{=\ }\hlstd{}\hlkwa{null}\\
\hllin{24\ }\hlstd{}\hlstd{\ \ \ \ }\hlstd{}\hlkwa{internal\ var\ }\hlstd{ilockcmdquene}\hlopt{:\ }\hlstd{ILock?\ }\hlopt{=\ }\hlstd{}\hlkwa{null}\\
\hllin{25\ }\hlstd{}\hlstd{\ \ \ \ }\hlstd{}\hlkwc{@Autowired}\\
\hllin{26\ }\hlstd{}\hlstd{\ \ \ \ }\hlstd{}\hlkwa{internal\ var\ }\hlstd{logFile}\hlopt{:\ }\hlstd{LogFile?\ }\hlopt{=\ }\hlstd{}\hlkwa{null}\\
\hllin{27\ }\hlstd{}\hlstd{\ \ \ \ }\hlstd{}\hlkwa{internal\ var\ }\hlstd{logger\ }\hlopt{=\ }\hlstd{LoggerFactory}\hlopt{.}\hlstd{}\hlkwd{getLogger}\hlstd{}\hlopt{(}\hlstd{MyHazelcast}\hlopt{::}\hlstd{}\hlkwa{class}\hlstd{}\hlopt{.}\hlstd{java}\hlopt{.}\hlstd{name}\hlopt{)}\\
\hllin{28\ }\hlstd{}\hlstd{\ \ \ \ }\hlstd{}\hlkwa{private\ var\ }\hlstd{locals\symbol{95}maxlsn}\hlopt{:\ }\hlstd{}\hlkwb{Long\ }\hlstd{}\hlopt{=\ }\hlstd{}\hlnum{0}\\
\hllin{29\ }\hlstd{}\hlstd{\ \ \ \ }\hlstd{}\hlkwa{private\ var\ }\hlstd{localmember}\hlopt{:\ }\hlstd{Member?\ }\hlopt{=\ }\hlstd{}\hlkwa{null}\\
\hllin{30\ }\hlstd{}\hlstd{\ \ \ \ }\hlstd{}\hlkwa{private\ var\ }\hlstd{iAtomic\symbol{95}remote\symbol{95}lsn}\hlopt{:\ }\hlstd{IAtomicLong?\ }\hlopt{=\ }\hlstd{}\hlkwa{null}\\
\hllin{31\ }\hlstd{}\hlstd{\ \ \ \ }\hlstd{}\hlkwa{internal\ var\ }\hlstd{oNullConnection}\hlopt{:\ }\hlstd{ONullConnection\ }\hlopt{=\ }\hlstd{}\hlkwd{ONullConnection}\hlstd{}\hlopt{()}\\
\hllin{32\ }\hlstd{}\hlstd{\ \ \ \ }\hlstd{}\hlkwc{@Autowired}\\
\hllin{33\ }\hlstd{}\hlstd{\ \ \ \ }\hlstd{}\hlkwa{internal\ var\ }\hlstd{applicationContext}\hlopt{:\ }\hlstd{ApplicationContext?\ }\hlopt{=\ }\hlstd{}\hlkwa{null}\\
\hllin{34\ }\hlstd{}\hlstd{\ \ \ \ }\hlstd{}\hlkwa{internal\ var\ }\hlstd{sqLhander\ }\hlopt{=\ }\hlstd{}\hlkwd{MysqlSQLhander}\hlstd{}\hlopt{()}\\
\hllin{35\ }\hlstd{}\hlstd{\ \ \ \ }\hlstd{}\hlkwc{@PostConstruct}\\
\hllin{36\ }\hlstd{}\hlstd{\ \ \ \ }\hlstd{}\hlkwa{fun\ }\hlstd{}\hlkwd{myinit}\hlstd{}\hlopt{()\ \symbol{123}}\\
\hllin{37\ }\hlstd{}\hlstd{\ \ \ \ \ \ \ \ }\hlstd{locals\symbol{95}maxlsn\ }\hlopt{=\ }\hlstd{logFile}\hlopt{!!.}\hlstd{}\hlkwd{maxLSN}\hlstd{}\hlopt{()}\\
\hllin{38\ }\hlstd{}\hlstd{\ \ \ \ }\hlstd{}\hlopt{\symbol{125}}\\
\hllin{39\ }\hlstd{}\hlstd{\ \ \ \ }\hlstd{}\hlkwa{fun\ }\hlstd{}\hlkwd{inits}\hlstd{}\hlopt{()\ \symbol{123}}\\
\hllin{40\ }\hlstd{}\hlstd{\ \ \ \ \ \ \ \ }\hlstd{}\hlkwa{val\ }\hlstd{config\ }\hlopt{=\ }\hlstd{}\hlkwd{Config}\hlstd{}\hlopt{()}\\
\hllin{41\ }\hlstd{}\hlstd{\ \ \ \ \ \ \ \ }\hlstd{hazelcastInstance\ }\hlopt{=\ }\hlstd{Hazelcast}\hlopt{.}\hlstd{}\hlkwd{newHazelcastInstance}\hlstd{}\hlopt{(}\hlstd{config}\hlopt{)}\\
\hllin{42\ }\hlstd{}\hlstd{\ \ \ \ \ \ \ \ }\hlstd{localmember\ }\hlopt{=\ }\hlstd{hazelcastInstance}\hlopt{!!.}\hlstd{cluster}\hlopt{.}\hlstd{localMember}\\
\hllin{43\ }\hlstd{}\hlstd{\ \ \ \ \ \ \ \ }\hlstd{remotequene\ }\hlopt{=\ }\hlstd{hazelcastInstance}\hlopt{!!.}\hlstd{getQueue}\hlopt{\symbol{60}}\hlstd{SqlUpdateLog}\hlopt{\symbol{62}(}\hlstd{}\hlstr{"sqls"}\hlstd{}\hlopt{)}\\
\hllin{44\ }\hlstd{}\hlstd{\ \ \ \ \ \ \ \ }\hlstd{remotequene}\hlopt{!!.}\hlstd{}\hlkwd{addItemListener}\hlstd{}\hlopt{(}\hlstd{}\hlkwa{this}\hlstd{}\hlopt{,\ }\hlstd{}\hlkwa{true}\hlstd{}\hlopt{)}\\
\hllin{45\ }\hlstd{}\hlstd{\ \ \ \ \ \ \ \ }\hlstd{replicationQuene\ }\hlopt{=\ }\hlstd{hazelcastInstance}\hlopt{!!.}\hlstd{getQueue}\hlopt{\symbol{60}}\hlstd{SqlUpdateLog}\hlopt{\symbol{62}(}\hlstd{}\hlstr{"sendingsqls"}\hlstd{}\hlopt{)}\\
\hllin{46\ }\hlstd{}\hlstd{\ \ \ \ \ \ \ \ }\hlstd{replicationQuene}\hlopt{!!.}\hlstd{}\hlkwd{addItemListener}\hlstd{}\hlopt{(}\hlstd{remotequenelister}\hlopt{,\ }\hlstd{}\hlkwa{true}\hlstd{}\hlopt{)}\\
\hllin{47\ }\hlstd{}\hlstd{\ \ \ \ \ \ \ \ }\hlstd{iLockwrite\ }\hlopt{=\ }\hlstd{hazelcastInstance}\hlopt{!!.}\hlstd{}\hlkwd{getLock}\hlstd{}\hlopt{(}\hlstd{}\hlstr{"ilocakwrite"}\hlstd{}\hlopt{)}\\
\hllin{48\ }\hlstd{}\hlstd{\ \ \ \ \ \ \ \ }\hlstd{iLockread\ }\hlopt{=\ }\hlstd{hazelcastInstance}\hlopt{!!.}\hlstd{}\hlkwd{getLock}\hlstd{}\hlopt{(}\hlstd{}\hlstr{"ilockread"}\hlstd{}\hlopt{)}\\
\hllin{49\ }\hlstd{}\hlstd{\ \ \ \ \ \ \ \ }\hlstd{iAtomic\symbol{95}remote\symbol{95}lsn\ }\hlopt{=\ }\hlstd{hazelcastInstance}\hlopt{!!.}\hlstd{}\hlkwd{getAtomicLong}\hlstd{}\hlopt{(}\hlstd{}\hlstr{"lsn"}\hlstd{}\hlopt{)}\\
\hllin{50\ }\hlstd{}\hlstd{\ \ \ \ \ \ \ \ }\hlstd{ilockcmdquene\ }\hlopt{=\ }\hlstd{hazelcastInstance}\hlopt{!!.}\hlstd{}\hlkwd{getLock}\hlstd{}\hlopt{(}\hlstd{}\hlstr{"ilockcmdquene"}\hlstd{}\hlopt{)}\\
\hllin{51\ }\hlstd{}\hlstd{\ \ \ \ \ \ \ \ }\hlstd{cmdiQueue\ }\hlopt{=\ }\hlstd{hazelcastInstance}\hlopt{!!.}\hlstd{getQueue}\hlopt{\symbol{60}}\hlstd{ReplicationCMD}\hlopt{\symbol{62}(}\hlstd{}\hlstr{"cmds"}\hlstd{}\hlopt{)}\\
\hllin{52\ }\hlstd{}\hlstd{\ \ \ \ \ \ \ \ }\hlstd{cmdItemListener\ }\hlopt{=\ }\hlstd{}\hlkwd{cmdlister}\hlstd{}\hlopt{()}\\
\hllin{53\ }\hlstd{}\hlstd{\ \ \ \ \ \ \ \ }\hlstd{idGenerator\ }\hlopt{=\ }\hlstd{hazelcastInstance}\hlopt{!!.}\hlstd{}\hlkwd{getIdGenerator}\hlstd{}\hlopt{(}\hlstd{}\hlstr{"idgenerator"}\hlstd{}\hlopt{)}\\
\hllin{54\ }\hlstd{}\hlstd{\ \ \ \ \ \ \ \ }\hlstd{cmdiQueue}\hlopt{!!.}\hlstd{}\hlkwd{addItemListener}\hlstd{}\hlopt{(}\hlstd{cmdItemListener}\hlopt{,\ }\hlstd{}\hlkwa{true}\hlstd{}\hlopt{)}\\
\hllin{55\ }\hlstd{}\hlstd{\ \ \ \ \ \ \ \ }\hlstd{sqLhander\ }\hlopt{=\ }\hlstd{applicationContext}\hlopt{!!.}\hlstd{}\hlkwd{getBean}\hlstd{}\hlopt{(}\hlstd{MysqlSQLhander}\hlopt{::}\hlstd{}\hlkwa{class}\hlstd{}\hlopt{.}\hlstd{java}\hlopt{)}\\
\hllin{56\ }\hlstd{}\hlstd{\ \ \ \ \ \ \ \ }\hlstd{}\hlkwd{init}\hlstd{}\hlopt{()}\\
\hllin{57\ }\hlstd{}\hlstd{\ \ \ \ }\hlstd{}\hlopt{\symbol{125}}\\
\hllin{58\ }\hlstd{}\hlstd{\ \ \ \ }\hlstd{}\hlkwa{internal\ }\hlstd{inner\ }\hlkwa{class\ }\hlstd{cmdlister\ }\hlopt{:\ }\hlstd{ItemListener}\hlopt{\symbol{60}}\hlstd{ReplicationCMD}\hlopt{\symbol{62}\ \symbol{123}}\\
\hllin{59\ }\hlstd{}\hlstd{\ \ \ \ \ \ \ \ }\hlstd{}\hlkwa{override\ fun\ }\hlstd{}\hlkwd{itemAdded}\hlstd{}\hlopt{(}\hlstd{item}\hlopt{:\ }\hlstd{ItemEvent}\hlopt{\symbol{60}}\hlstd{ReplicationCMD}\hlopt{\symbol{62})\ \symbol{123}}\\
\hllin{60\ }\hlstd{}\hlstd{\ \ \ \ \ \ \ \ \ \ \ \ }\hlstd{}\hlkwa{var\ }\hlstd{replicationCMD\ }\hlopt{=\ }\hlstd{item}\hlopt{.}\hlstd{item}\\
\hllin{61\ }\hlstd{}\hlstd{\ \ \ \ \ \ \ \ \ \ \ \ }\hlstd{}\hlkwa{if\ }\hlstd{}\hlopt{(}\hlstd{replicationCMD}\hlopt{.}\hlstd{id\ }\hlopt{==\ }\hlstd{id\symbol{95}froCMD}\hlopt{)\ \symbol{123}}\\
\hllin{62\ }\hlstd{}\hlstd{\ \ \ \ \ \ \ \ \ \ \ \ \ \ \ \ }\hlstd{}\hlkwa{return}\\
\hllin{63\ }\hlstd{}\hlstd{\ \ \ \ \ \ \ \ \ \ \ \ }\hlstd{}\hlopt{\symbol{125}}\\
\hllin{64\ }\hlstd{}\hlstd{\ \ \ \ \ \ \ \ \ \ \ \ }\hlstd{}\hlkwa{if\ }\hlstd{}\hlopt{(}\hlstd{locals\symbol{95}maxlsn\ }\hlopt{\symbol{60}\ }\hlstd{replicationCMD}\hlopt{.}\hlstd{tolsn}\hlopt{)\ \symbol{123}}\\
\hllin{65\ }\hlstd{}\hlstd{\ \ \ \ \ \ \ \ \ \ \ \ \ \ \ \ }\hlstd{}\hlkwa{return}\\
\hllin{66\ }\hlstd{}\hlstd{\ \ \ \ \ \ \ \ \ \ \ \ }\hlstd{}\hlopt{\symbol{125}}\\
\hllin{67\ }\hlstd{}\hlstd{\ \ \ \ \ \ \ \ \ \ \ \ }\hlstd{}\hlkwa{if\ }\hlstd{}\hlopt{(}\hlstd{ilockcmdquene}\hlopt{!!.}\hlstd{}\hlkwd{tryLock}\hlstd{}\hlopt{())\ \symbol{123}}\\
\hllin{68\ }\hlstd{}\hlstd{\ \ \ \ \ \ \ \ \ \ \ \ \ \ \ \ }\hlstd{replicationCMD\ }\hlopt{=\ }\hlstd{cmdiQueue}\hlopt{!!.}\hlstd{}\hlkwd{poll}\hlstd{}\hlopt{()}\\
\hllin{69\ }\hlstd{}\hlstd{\ \ \ \ \ \ \ \ \ \ \ \ \ \ \ \ }\hlstd{ilockcmdquene}\hlopt{!!.}\hlstd{}\hlkwd{unlock}\hlstd{}\hlopt{()}\\
\hllin{70\ }\hlstd{}\hlstd{\ \ \ \ \ \ \ \ \ \ \ \ \ \ \ \ }\hlstd{}\hlkwd{sendlogdata}\hlstd{}\hlopt{(}\hlstd{replicationCMD}\hlopt{.}\hlstd{fromlsn}\hlopt{,\ }\hlstd{replicationCMD}\hlopt{.}\hlstd{tolsn}\hlopt{)}\\
\hllin{71\ }\hlstd{}\hlstd{\ \ \ \ \ \ \ \ \ \ \ \ }\hlstd{}\hlopt{\symbol{125}}\\
\hllin{72\ }\hlstd{}\\
\hllin{73\ }\hlstd{}\hlstd{\ \ \ \ \ \ \ \ }\hlstd{}\hlopt{\symbol{125}}\\
\hllin{74\ }\hlstd{}\hlstd{\ \ \ \ \ \ \ \ }\hlstd{}\hlkwa{override\ fun\ }\hlstd{}\hlkwd{itemRemoved}\hlstd{}\hlopt{(}\hlstd{item}\hlopt{:\ }\hlstd{ItemEvent}\hlopt{\symbol{60}}\hlstd{ReplicationCMD}\hlopt{\symbol{62})\ \symbol{123}}\\
\hllin{75\ }\hlstd{}\\
\hllin{76\ }\hlstd{}\hlstd{\ \ \ \ \ \ \ \ }\hlstd{}\hlopt{\symbol{125}}\\
\hllin{77\ }\hlstd{}\hlstd{\ \ \ \ }\hlstd{}\hlopt{\symbol{125}}\\
\hllin{78\ }\hlstd{}\hlstd{\ \ \ \ }\hlstd{}\hlkwa{internal\ }\hlstd{inner\ }\hlkwa{class\ }\hlstd{replicationlister\ }\hlopt{:\ }\hlstd{ItemListener}\hlopt{\symbol{60}}\hlstd{SqlUpdateLog}\hlopt{\symbol{62}\ \symbol{123}}\\
\hllin{79\ }\hlstd{}\hlstd{\ \ \ \ \ \ \ \ }\hlstd{}\hlkwa{override\ fun\ }\hlstd{}\hlkwd{itemAdded}\hlstd{}\hlopt{(}\hlstd{item}\hlopt{:\ }\hlstd{ItemEvent}\hlopt{\symbol{60}}\hlstd{SqlUpdateLog}\hlopt{\symbol{62})\ \symbol{123}}\\
\hllin{80\ }\hlstd{}\hlstd{\ \ \ \ \ \ \ \ \ \ \ \ }\hlstd{}\hlkwa{if\ }\hlstd{}\hlopt{(!}\hlstd{isreplicating}\hlopt{)\ \symbol{123}}\\
\hllin{81\ }\hlstd{}\hlstd{\ \ \ \ \ \ \ \ \ \ \ \ \ \ \ \ }\hlstd{}\hlkwa{return}\\
\hllin{82\ }\hlstd{}\hlstd{\ \ \ \ \ \ \ \ \ \ \ \ }\hlstd{}\hlopt{\symbol{125}}\\
\hllin{83\ }\hlstd{}\hlstd{\ \ \ \ \ \ \ \ \ \ \ \ }\hlstd{}\hlkwa{val\ }\hlstd{max\ }\hlopt{=\ }\hlstd{iAtomic\symbol{95}remote\symbol{95}lsn}\hlopt{!!.}\hlstd{}\hlkwd{get}\hlstd{}\hlopt{()}\\
\hllin{84\ }\hlstd{}\hlstd{\ \ \ \ \ \ \ \ \ \ \ \ }\hlstd{}\hlkwa{if\ }\hlstd{}\hlopt{(}\hlstd{item}\hlopt{.}\hlstd{item}\hlopt{.}\hlstd{LSN\ }\hlopt{==\ }\hlstd{max}\hlopt{)\ \symbol{123}}\\
\hllin{85\ }\hlstd{}\hlstd{\ \ \ \ \ \ \ \ \ \ \ \ \ \ \ \ }\hlstd{localqueneReplication}\hlopt{.}\hlstd{}\hlkwd{addLast}\hlstd{}\hlopt{(}\hlstd{item}\hlopt{.}\hlstd{item}\hlopt{)}\\
\hllin{86\ }\hlstd{}\hlstd{\ \ \ \ \ \ \ \ \ \ \ \ \ \ \ \ }\hlstd{}\hlkwd{exeSqlforReplication}\hlstd{}\hlopt{()}\\
\hllin{87\ }\hlstd{}\hlstd{\ \ \ \ \ \ \ \ \ \ \ \ }\hlstd{}\hlopt{\symbol{125}\ }\hlstd{}\hlkwa{else\ }\hlstd{}\hlopt{\symbol{123}}\\
\hllin{88\ }\hlstd{}\hlstd{\ \ \ \ \ \ \ \ \ \ \ \ \ \ \ \ }\hlstd{}\hlkwa{if\ }\hlstd{}\hlopt{(}\hlstd{item}\hlopt{.}\hlstd{item}\hlopt{.}\hlstd{LSN\ }\hlopt{\symbol{62}\ }\hlstd{locals\symbol{95}maxlsn}\hlopt{)\ \symbol{123}}\\
\hllin{89\ }\hlstd{}\hlstd{\ \ \ \ \ \ \ \ \ \ \ \ \ \ \ \ \ \ \ \ }\hlstd{localqueneReplication}\hlopt{.}\hlstd{}\hlkwd{addLast}\hlstd{}\hlopt{(}\hlstd{item}\hlopt{.}\hlstd{item}\hlopt{)}\\
\hllin{90\ }\hlstd{}\hlstd{\ \ \ \ \ \ \ \ \ \ \ \ \ \ \ \ }\hlstd{}\hlopt{\symbol{125}}\\
\hllin{91\ }\hlstd{}\hlstd{\ \ \ \ \ \ \ \ \ \ \ \ }\hlstd{}\hlopt{\symbol{125}}\\
\hllin{92\ }\hlstd{}\\
\hllin{93\ }\hlstd{}\hlstd{\ \ \ \ \ \ \ \ }\hlstd{}\hlopt{\symbol{125}}\\
\hllin{94\ }\hlstd{}\hlstd{\ \ \ \ \ \ \ \ }\hlstd{}\hlkwa{override\ fun\ }\hlstd{}\hlkwd{itemRemoved}\hlstd{}\hlopt{(}\hlstd{item}\hlopt{:\ }\hlstd{ItemEvent}\hlopt{\symbol{60}}\hlstd{SqlUpdateLog}\hlopt{\symbol{62})\ \symbol{123}}\\
\hllin{95\ }\hlstd{}\hlstd{\ \ \ \ \ \ \ \ }\hlstd{}\hlopt{\symbol{125}}\\
\hllin{96\ }\hlstd{}\hlstd{\ \ \ \ }\hlstd{}\hlopt{\symbol{125}}\\
\hllin{97\ }\hlstd{}\hlstd{\ \ \ \ }\hlstd{}\hlkwa{private\ fun\ }\hlstd{}\hlkwd{exeSqlforReplication}\hlstd{}\hlopt{()\ \symbol{123}}\\
\hllin{98\ }\hlstd{}\hlstd{\ \ \ \ \ \ \ \ }\hlstd{Collections}\hlopt{.}\hlstd{}\hlkwd{sort}\hlstd{}\hlopt{(}\hlstd{localqueneReplication}\hlopt{)}\\
\hllin{99\ }\hlstd{}\hlstd{\ \ \ \ \ \ \ \ }\hlstd{}\hlkwa{var\ }\hlstd{log}\hlopt{:\ }\hlstd{SqlUpdateLog?\ }\hlopt{=\ }\hlstd{localqueneReplication}\hlopt{.}\hlstd{}\hlkwd{poll}\hlstd{}\hlopt{()}\\
\hllin{100\ }\hlstd{}\hlstd{\ \ \ \ \ \ \ \ }\hlstd{}\hlkwa{var\ }\hlstd{lastlsn}\hlopt{:\ }\hlstd{}\hlkwb{Long\ }\hlstd{}\hlopt{=\ }\hlstd{}\hlnum{0}\\
\hllin{101\ }\hlstd{}\hlstd{\ \ \ \ \ \ \ \ }\hlstd{}\hlkwa{while\ }\hlstd{}\hlopt{(}\hlstd{log\ }\hlopt{!=\ }\hlstd{}\hlkwa{null}\hlstd{}\hlopt{)\ \symbol{123}}\\
\hllin{102\ }\hlstd{}\hlstd{\ \ \ \ \ \ \ \ \ \ \ \ }\hlstd{}\hlslc{//}\hlstd{\ \ \ \ \ \ \ \ \ \ \ \ }\hlslc{{-}{-}{-}{-}{-}{-}{-}{-}{-}{-}{-}{-}{-}{-}{-}{-}{-}{-}{-}{-}{-}{-}{-}{-}{-}{-}{-}{-}{-}{-}{-}{-}{-}{-}{-}{-}{-}{-}{-}{-}{-}{-}{-}{-}{-}{-}{-}{-}{-}{-}{-}{-}{-}{-}{-}{-}{-}{-}{-}{-}}\\
\hllin{103\ }\hlstd{}\hlstd{\ \ \ \ \ \ \ \ \ \ \ \ }\hlstd{locals\symbol{95}maxlsn\ }\hlopt{=\ }\hlstd{log}\hlopt{.}\hlstd{LSN}\\
\hllin{104\ }\hlstd{}\hlstd{\ \ \ \ \ \ \ \ \ \ \ \ }\hlstd{logger}\hlopt{.}\hlstd{}\hlkwd{info}\hlstd{}\hlopt{(}\hlstd{}\hlstr{"exeSqlforReplication\ exe\ sql\ "}\hlstd{\ }\hlopt{+\ }\hlstd{log}\hlopt{)}\\
\hllin{105\ }\hlstd{}\hlstd{\ \ \ \ \ \ \ \ \ \ \ \ }\hlstd{oNullConnection}\hlopt{!!.}\hlstd{schema\ }\hlopt{=\ }\hlstd{log}\hlopt{.}\hlstd{db}\\
\hllin{106\ }\hlstd{}\hlstd{\ \ \ \ \ \ \ \ \ \ \ \ }\hlstd{sqLhander}\hlopt{.}\hlstd{}\hlkwd{handle}\hlstd{}\hlopt{(}\hlstd{log}\hlopt{,\ }\hlstd{oNullConnection}\hlopt{)}\\
\hllin{107\ }\hlstd{}\\
\hllin{108\ }\hlstd{}\hlstd{\ \ \ \ \ \ \ \ \ \ \ \ }\hlstd{}\hlslc{//}\hlstd{\ \ \ \ \ \ \ \ \ \ \ \ }\hlslc{logfiletest.addLast(log);}\\
\hllin{109\ }\hlstd{}\hlstd{\ \ \ \ \ \ \ \ \ \ \ \ }\hlstd{logFile}\hlopt{!!.}\hlstd{}\hlkwd{write}\hlstd{}\hlopt{(}\hlstd{log}\hlopt{)}\\
\hllin{110\ }\hlstd{}\hlstd{\ \ \ \ \ \ \ \ \ \ \ \ }\hlstd{lastlsn\ }\hlopt{=\ }\hlstd{log}\hlopt{.}\hlstd{LSN}\\
\hllin{111\ }\hlstd{}\hlstd{\ \ \ \ \ \ \ \ \ \ \ \ }\hlstd{log\ }\hlopt{=\ }\hlstd{localqueneReplication}\hlopt{.}\hlstd{}\hlkwd{poll}\hlstd{}\hlopt{()}\\
\hllin{112\ }\hlstd{}\hlstd{\ \ \ \ \ \ \ \ }\hlstd{}\hlopt{\symbol{125}}\\
\hllin{113\ }\hlstd{}\hlstd{\ \ \ \ \ \ \ \ }\hlstd{isreplicating\ }\hlopt{=\ }\hlstd{}\hlkwa{false}\\
\hllin{114\ }\hlstd{}\hlstd{\ \ \ \ \ \ \ \ }\hlstd{log\ }\hlopt{=\ }\hlstd{localquene}\hlopt{.}\hlstd{}\hlkwd{poll}\hlstd{}\hlopt{()}\\
\hllin{115\ }\hlstd{}\hlstd{\ \ \ \ \ \ \ \ }\hlstd{}\hlkwa{val\ }\hlstd{remotel\ }\hlopt{=\ }\hlstd{iAtomic\symbol{95}remote\symbol{95}lsn}\hlopt{!!.}\hlstd{}\hlkwd{get}\hlstd{}\hlopt{()}\\
\hllin{116\ }\hlstd{}\hlstd{\ \ \ \ \ \ \ \ }\hlstd{}\hlkwa{while\ }\hlstd{}\hlopt{(}\hlstd{log\ }\hlopt{!=\ }\hlstd{}\hlkwa{null}\hlstd{}\hlopt{)\ \symbol{123}}\\
\hllin{117\ }\hlstd{}\hlstd{\ \ \ \ \ \ \ \ \ \ \ \ }\hlstd{}\hlslc{//}\hlstd{\ \ \ \ \ \ \ \ \ \ \ \ }\hlslc{{-}{-}{-}{-}{-}{-}{-}{-}{-}{-}{-}{-}{-}{-}{-}{-}{-}{-}{-}{-}{-}{-}{-}{-}{-}{-}{-}{-}{-}{-}{-}{-}{-}{-}{-}{-}{-}{-}{-}{-}{-}{-}{-}{-}{-}{-}{-}{-}{-}{-}{-}{-}{-}{-}{-}{-}{-}{-}{-}{-}}\\
\hllin{118\ }\hlstd{}\hlstd{\ \ \ \ \ \ \ \ \ \ \ \ }\hlstd{locals\symbol{95}maxlsn\ }\hlopt{=\ }\hlstd{log}\hlopt{.}\hlstd{LSN}\\
\hllin{119\ }\hlstd{}\hlstd{\ \ \ \ \ \ \ \ \ \ \ \ }\hlstd{logger}\hlopt{.}\hlstd{}\hlkwd{info}\hlstd{}\hlopt{(}\hlstd{}\hlstr{"exe}\hlstd{\ \ }\hlstr{Sql\ :\ "}\hlstd{\ }\hlopt{+\ }\hlstd{log}\hlopt{)}\\
\hllin{120\ }\hlstd{}\hlstd{\ \ \ \ \ \ \ \ \ \ \ \ }\hlstd{oNullConnection}\hlopt{!!.}\hlstd{schema\ }\hlopt{=\ }\hlstd{log}\hlopt{.}\hlstd{db}\\
\hllin{121\ }\hlstd{}\hlstd{\ \ \ \ \ \ \ \ \ \ \ \ }\hlstd{sqLhander}\hlopt{.}\hlstd{}\hlkwd{handle}\hlstd{}\hlopt{(}\hlstd{log}\hlopt{,\ }\hlstd{oNullConnection}\hlopt{)}\\
\hllin{122\ }\hlstd{}\\
\hllin{123\ }\hlstd{}\hlstd{\ \ \ \ \ \ \ \ \ \ \ \ }\hlstd{}\hlslc{//}\hlstd{\ \ \ \ \ \ \ \ \ \ \ \ }\hlslc{logfiletest.addLast(log);}\\
\hllin{124\ }\hlstd{}\hlstd{\ \ \ \ \ \ \ \ \ \ \ \ }\hlstd{logFile}\hlopt{!!.}\hlstd{}\hlkwd{write}\hlstd{}\hlopt{(}\hlstd{log}\hlopt{)}\\
\hllin{125\ }\hlstd{}\hlstd{\ \ \ \ \ \ \ \ \ \ \ \ }\hlstd{lastlsn\ }\hlopt{=\ }\hlstd{log}\hlopt{.}\hlstd{LSN}\\
\hllin{126\ }\hlstd{}\hlstd{\ \ \ \ \ \ \ \ \ \ \ \ }\hlstd{}\hlkwa{if\ }\hlstd{}\hlopt{(}\hlstd{lastlsn\ }\hlopt{\symbol{62}\ }\hlstd{remotel}\hlopt{)\ \symbol{123}}\\
\hllin{127\ }\hlstd{}\hlstd{\ \ \ \ \ \ \ \ \ \ \ \ \ \ \ \ }\hlstd{remotequene}\hlopt{!!.}\hlstd{}\hlkwd{offer}\hlstd{}\hlopt{(}\hlstd{log}\hlopt{)}\\
\hllin{128\ }\hlstd{}\hlstd{\ \ \ \ \ \ \ \ \ \ \ \ }\hlstd{}\hlopt{\symbol{125}}\\
\hllin{129\ }\hlstd{}\hlstd{\ \ \ \ \ \ \ \ \ \ \ \ }\hlstd{log\ }\hlopt{=\ }\hlstd{localqueneReplication}\hlopt{.}\hlstd{}\hlkwd{poll}\hlstd{}\hlopt{()}\\
\hllin{130\ }\hlstd{}\hlstd{\ \ \ \ \ \ \ \ }\hlstd{}\hlopt{\symbol{125}}\\
\hllin{131\ }\hlstd{}\hlstd{\ \ \ \ \ \ \ \ }\hlstd{}\hlkwa{if\ }\hlstd{}\hlopt{(}\hlstd{lastlsn\ }\hlopt{\symbol{62}\ }\hlstd{remotel}\hlopt{)\ \symbol{123}}\\
\hllin{132\ }\hlstd{}\hlstd{\ \ \ \ \ \ \ \ \ \ \ \ }\hlstd{iAtomic\symbol{95}remote\symbol{95}lsn}\hlopt{!!.}\hlstd{}\hlkwd{set}\hlstd{}\hlopt{(}\hlstd{lastlsn}\hlopt{)}\\
\hllin{133\ }\hlstd{}\hlstd{\ \ \ \ \ \ \ \ }\hlstd{}\hlopt{\symbol{125}}\\
\hllin{134\ }\hlstd{}\hlstd{\ \ \ \ }\hlstd{}\hlopt{\symbol{125}}\\
\hllin{135\ }\hlstd{}\\
\hllin{136\ }\hlstd{}\hlstd{\ \ \ \ }\hlstd{}\hlkwa{private\ fun\ }\hlstd{}\hlkwd{sendlogdata}\hlstd{}\hlopt{(}\hlstd{fromlsn}\hlopt{:\ }\hlstd{}\hlkwb{Long}\hlstd{}\hlopt{,\ }\hlstd{tolsn}\hlopt{:\ }\hlstd{}\hlkwb{Long}\hlstd{}\hlopt{)\ \symbol{123}}\\
\hllin{137\ }\hlstd{}\hlstd{\ \ \ \ \ \ \ \ }\hlstd{logFile}\hlopt{!!.}\hlstd{}\hlkwd{getall}\hlstd{}\hlopt{().}\hlstd{forEach\ }\hlopt{\symbol{123}\ }\hlstd{a\ }\hlopt{{-}\symbol{62}}\\
\hllin{138\ }\hlstd{}\hlstd{\ \ \ \ \ \ \ \ \ \ \ \ }\hlstd{}\hlkwa{if\ }\hlstd{}\hlopt{(}\hlstd{a}\hlopt{.}\hlstd{LSN\ }\hlopt{\symbol{62}=\ }\hlstd{fromlsn\ }\hlopt{\&\&\ }\hlstd{a}\hlopt{.}\hlstd{LSN\ }\hlopt{\symbol{60}=\ }\hlstd{tolsn}\hlopt{)\ \symbol{123}}\\
\hllin{139\ }\hlstd{}\hlstd{\ \ \ \ \ \ \ \ \ \ \ \ \ \ \ \ }\hlstd{replicationQuene}\hlopt{!!.}\hlstd{}\hlkwd{offer}\hlstd{}\hlopt{(}\hlstd{a}\hlopt{)}\\
\hllin{140\ }\hlstd{}\hlstd{\ \ \ \ \ \ \ \ \ \ \ \ \ \ \ \ }\hlstd{logger}\hlopt{.}\hlstd{}\hlkwd{info}\hlstd{}\hlopt{(}\hlstd{}\hlstr{"send\ log:"}\hlstd{\ }\hlopt{+\ }\hlstd{a}\hlopt{)}\\
\hllin{141\ }\hlstd{}\hlstd{\ \ \ \ \ \ \ \ \ \ \ \ }\hlstd{}\hlopt{\symbol{125}}\\
\hllin{142\ }\hlstd{}\hlstd{\ \ \ \ \ \ \ \ }\hlstd{}\hlopt{\symbol{125}}\\
\hllin{143\ }\hlstd{}\hlstd{\ \ \ \ \ \ \ \ }\hlstd{logger}\hlopt{.}\hlstd{}\hlkwd{info}\hlstd{}\hlopt{(}\hlstd{}\hlstr{"send\ log\ data\ 从}\hlipl{\$fromlsn\ }\hlstr{to\ }\hlipl{\$tolsn}\hlstr{"}\hlstd{}\hlopt{)}\\
\hllin{144\ }\hlstd{}\hlstd{\ \ \ \ }\hlstd{}\hlopt{\symbol{125}}\\
\hllin{145\ }\hlstd{}\\
\hllin{146\ }\hlstd{}\hlstd{\ \ \ \ }\hlstd{}\hlkwa{private\ fun\ }\hlstd{}\hlkwd{init}\hlstd{}\hlopt{()\ \symbol{123}}\\
\hllin{147\ }\hlstd{}\hlstd{\ \ \ \ \ \ \ \ }\hlstd{}\hlkwa{val\ }\hlstd{l\ }\hlopt{=\ }\hlstd{iAtomic\symbol{95}remote\symbol{95}lsn}\hlopt{!!.}\hlstd{}\hlkwd{get}\hlstd{}\hlopt{()}\\
\hllin{148\ }\hlstd{}\hlstd{\ \ \ \ \ \ \ \ }\hlstd{logger}\hlopt{.}\hlstd{}\hlkwd{info}\hlstd{}\hlopt{(}\hlstd{}\hlstr{"remote\ lsn\ is"}\hlstd{\ }\hlopt{+\ }\hlstd{l}\hlopt{)}\\
\hllin{149\ }\hlstd{}\hlstd{\ \ \ \ \ \ \ \ }\hlstd{}\hlkwa{if\ }\hlstd{}\hlopt{(}\hlstd{locals\symbol{95}maxlsn\ }\hlopt{\symbol{60}\ }\hlstd{l}\hlopt{)\ \symbol{123}}\\
\hllin{150\ }\hlstd{}\hlstd{\ \ \ \ \ \ \ \ \ \ \ \ }\hlstd{isreplicating\ }\hlopt{=\ }\hlstd{}\hlkwa{true}\\
\hllin{151\ }\hlstd{}\hlstd{\ \ \ \ \ \ \ \ \ \ \ \ }\hlstd{id\symbol{95}froCMD\ }\hlopt{=\ }\hlstd{idGenerator}\hlopt{!!.}\hlstd{}\hlkwd{newId}\hlstd{}\hlopt{()}\\
\hllin{152\ }\hlstd{}\hlstd{\ \ \ \ \ \ \ \ \ \ \ \ }\hlstd{cmdiQueue}\hlopt{!!.}\hlstd{}\hlkwd{add}\hlstd{}\hlopt{(}\hlstd{}\hlkwd{ReplicationCMD}\hlstd{}\hlopt{(}\hlstd{locals\symbol{95}maxlsn\ }\hlopt{+\ }\hlstd{}\hlnum{1}\hlstd{}\hlopt{,\ }\hlstd{l}\hlopt{,\ }\hlstd{id\symbol{95}froCMD}\hlopt{))}\\
\hllin{153\ }\hlstd{}\hlstd{\ \ \ \ \ \ \ \ }\hlstd{}\hlopt{\symbol{125}\ }\hlstd{}\hlkwa{else\ }\hlstd{}\hlopt{\symbol{123}}\\
\hllin{154\ }\hlstd{}\hlstd{\ \ \ \ \ \ \ \ \ \ \ \ }\hlstd{iAtomic\symbol{95}remote\symbol{95}lsn}\hlopt{!!.}\hlstd{}\hlkwd{set}\hlstd{}\hlopt{(}\hlstd{locals\symbol{95}maxlsn}\hlopt{)}\\
\hllin{155\ }\hlstd{}\hlstd{\ \ \ \ \ \ \ \ }\hlstd{}\hlopt{\symbol{125}}\\
\hllin{156\ }\hlstd{}\hlstd{\ \ \ \ }\hlstd{}\hlopt{\symbol{125}}\\
\hllin{157\ }\hlstd{}\hlstd{\ \ \ \ }\hlstd{}\hlcom{/{*}{*}}\\
\hllin{158\ }\hlcom{}\hlstd{\ \ \ \ \ }\hlcom{{*}}\hlstd{\ \ }\hlcom{本机发出的sql语句.记录到本地logfile。同时发布到其他服务器{*}/}\hlstd{}\\
\hllin{159\ }\hlstd{}\hlstd{\ \ \ \ }\hlstd{}\hlkwa{fun\ }\hlstd{}\hlkwd{exeSql}\hlstd{}\hlopt{(}\hlstd{sql}\hlopt{:\ }\hlstd{}\hlkwb{String}\hlstd{}\hlopt{,\ }\hlstd{db}\hlopt{:\ }\hlstd{}\hlkwb{String}\hlstd{}\hlopt{)\ \symbol{123}}\\
\hllin{160\ }\hlstd{}\hlstd{\ \ \ \ \ \ \ \ }\hlstd{}\hlslc{//}\hlstd{\ \ \ \ \ \ \ \ }\hlslc{if\ (isupdatesql(sql))\ \symbol{123}}\\
\hllin{161\ }\hlstd{}\hlstd{\ \ \ \ \ \ \ \ }\hlstd{}\hlkwa{if\ }\hlstd{}\hlopt{(}\hlstd{isreplicating}\hlopt{)\ \symbol{123}}\\
\hllin{162\ }\hlstd{}\hlstd{\ \ \ \ \ \ \ \ \ \ \ \ }\hlstd{}\hlkwa{val\ }\hlstd{l\ }\hlopt{=\ }\hlstd{iAtomic\symbol{95}remote\symbol{95}lsn}\hlopt{!!.}\hlstd{}\hlkwd{get}\hlstd{}\hlopt{()}\\
\hllin{163\ }\hlstd{}\hlstd{\ \ \ \ \ \ \ \ \ \ \ \ }\hlstd{}\hlkwa{val\ }\hlstd{log\ }\hlopt{=\ }\hlstd{}\hlkwd{SqlUpdateLog}\hlstd{}\hlopt{(}\hlstd{l\ }\hlopt{+\ }\hlstd{}\hlnum{1}\hlstd{}\hlopt{,\ }\hlstd{sql}\hlopt{,\ }\hlstd{db}\hlopt{)}\\
\hllin{164\ }\hlstd{}\hlstd{\ \ \ \ \ \ \ \ \ \ \ \ }\hlstd{localquene}\hlopt{.}\hlstd{}\hlkwd{addLast}\hlstd{}\hlopt{(}\hlstd{log}\hlopt{)}\\
\hllin{165\ }\hlstd{}\hlstd{\ \ \ \ \ \ \ \ }\hlstd{}\hlopt{\symbol{125}\ }\hlstd{}\hlkwa{else\ }\hlstd{}\hlopt{\symbol{123}}\\
\hllin{166\ }\hlstd{}\hlstd{\ \ \ \ \ \ \ \ \ \ \ \ }\hlstd{locals\symbol{95}maxlsn}\hlopt{++}\\
\hllin{167\ }\hlstd{}\hlstd{\ \ \ \ \ \ \ \ \ \ \ \ }\hlstd{}\hlkwa{val\ }\hlstd{l\ }\hlopt{=\ }\hlstd{iAtomic\symbol{95}remote\symbol{95}lsn}\hlopt{!!.}\hlstd{}\hlkwd{addAndGet}\hlstd{}\hlopt{(}\hlstd{}\hlnum{1}\hlstd{}\hlopt{)}\\
\hllin{168\ }\hlstd{}\hlstd{\ \ \ \ \ \ \ \ \ \ \ \ }\hlstd{}\hlkwa{val\ }\hlstd{log\ }\hlopt{=\ }\hlstd{}\hlkwd{SqlUpdateLog}\hlstd{}\hlopt{(}\hlstd{l}\hlopt{,\ }\hlstd{sql}\hlopt{,\ }\hlstd{db}\hlopt{)}\\
\hllin{169\ }\hlstd{}\hlstd{\ \ \ \ \ \ \ \ \ \ \ \ }\hlstd{logger}\hlopt{.}\hlstd{}\hlkwd{info}\hlstd{}\hlopt{(}\hlstd{}\hlstr{"exe\ sql\ "}\hlstd{\ }\hlopt{+\ }\hlstd{log}\hlopt{)}\\
\hllin{170\ }\hlstd{}\hlstd{\ \ \ \ \ \ \ \ \ \ \ \ }\hlstd{}\hlslc{//}\hlstd{\ \ \ \ \ \ \ \ \ \ \ \ \ \ \ \ }\hlslc{logfiletest.addLast(log);}\\
\hllin{171\ }\hlstd{}\hlstd{\ \ \ \ \ \ \ \ \ \ \ \ }\hlstd{logFile}\hlopt{!!.}\hlstd{}\hlkwd{write}\hlstd{}\hlopt{(}\hlstd{log}\hlopt{)}\\
\hllin{172\ }\hlstd{}\hlstd{\ \ \ \ \ \ \ \ \ \ \ \ }\hlstd{remotequene}\hlopt{!!.}\hlstd{}\hlkwd{offer}\hlstd{}\hlopt{(}\hlstd{log}\hlopt{)}\\
\hllin{173\ }\hlstd{}\hlstd{\ \ \ \ \ \ \ \ }\hlstd{}\hlopt{\symbol{125}}\\
\hllin{174\ }\hlstd{}\hlstd{\ \ \ \ \ \ \ \ }\hlstd{}\hlslc{//}\hlstd{\ \ \ \ \ \ \ \ }\hlslc{\symbol{125}}\\
\hllin{175\ }\hlstd{}\hlstd{\ \ \ \ }\hlstd{}\hlopt{\symbol{125}}\\
\hllin{176\ }\hlstd{}\hlstd{\ \ \ \ }\hlstd{}\hlcom{/{*}{*}}\\
\hllin{177\ }\hlcom{}\hlstd{\ \ \ \ \ }\hlcom{{*}\ 只记录到本地logfile。不发布到其他服务器}\\
\hllin{178\ }\hlcom{}\hlstd{\ \ \ \ \ }\hlcom{{*}\ {*}/}\hlstd{}\\
\hllin{179\ }\hlstd{}\hlstd{\ \ \ \ }\hlstd{}\hlkwa{fun\ }\hlstd{}\hlkwd{exesqlLocal}\hlstd{}\hlopt{(}\hlstd{sql}\hlopt{:\ }\hlstd{}\hlkwb{String}\hlstd{}\hlopt{,\ }\hlstd{db}\hlopt{:\ }\hlstd{}\hlkwb{String}\hlstd{}\hlopt{)\ \symbol{123}}\\
\hllin{180\ }\hlstd{}\hlstd{\ \ \ \ \ \ \ \ }\hlstd{}\hlkwa{var\ }\hlstd{l\ }\hlopt{=\ }\hlstd{locals\symbol{95}maxlsn}\\
\hllin{181\ }\hlstd{}\hlstd{\ \ \ \ \ \ \ \ }\hlstd{}\hlkwa{val\ }\hlstd{log\ }\hlopt{=\ }\hlstd{}\hlkwd{SqlUpdateLog}\hlstd{}\hlopt{(}\hlstd{l\ }\hlopt{+\ }\hlstd{}\hlnum{1}\hlstd{}\hlopt{,\ }\hlstd{sql}\hlopt{,\ }\hlstd{db}\hlopt{)}\\
\hllin{182\ }\hlstd{}\hlstd{\ \ \ \ \ \ \ \ }\hlstd{logFile}\hlopt{!!.}\hlstd{}\hlkwd{write}\hlstd{}\hlopt{(}\hlstd{log}\hlopt{)}\\
\hllin{183\ }\hlstd{}\hlstd{\ \ \ \ }\hlstd{}\hlopt{\symbol{125}}\\
\hllin{184\ }\hlstd{}\hlstd{\ \ \ \ }\hlstd{}\hlkwa{private\ fun\ }\hlstd{}\hlkwd{isupdatesql}\hlstd{}\hlopt{(}\hlstd{sql}\hlopt{:\ }\hlstd{}\hlkwb{String}\hlstd{}\hlopt{):\ }\hlstd{}\hlkwb{Boolean\ }\hlstd{}\hlopt{\symbol{123}}\\
\hllin{185\ }\hlstd{}\hlstd{\ \ \ \ \ \ \ \ }\hlstd{}\hlkwa{val\ }\hlstd{sqll\ }\hlopt{=\ }\hlstd{sql}\hlopt{.}\hlstd{}\hlkwd{toLowerCase}\hlstd{}\hlopt{()}\\
\hllin{186\ }\hlstd{}\hlstd{\ \ \ \ \ \ \ \ }\hlstd{}\hlkwa{val\ }\hlstd{list\ }\hlopt{=\ }\hlstd{Lists}\hlopt{.}\hlstd{}\hlkwd{newArrayList}\hlstd{}\hlopt{(}\hlstd{}\hlstr{"delete"}\hlstd{}\hlopt{,\ }\hlstd{}\hlstr{"drop"}\hlstd{}\hlopt{,\ }\hlstd{}\hlstr{"create"}\hlstd{}\hlopt{,\ }\hlstd{}\hlstr{"insert"}\hlstd{}\hlopt{,\ }\hlstd{}\hlstr{"update"}\hlstd{}\hlopt{)}\\
\hllin{187\ }\hlstd{}\hlstd{\ \ \ \ \ \ \ \ }\hlstd{}\hlkwa{for\ }\hlstd{}\hlopt{(}\hlstd{s\ }\hlkwa{in\ }\hlstd{list}\hlopt{)\ \symbol{123}}\\
\hllin{188\ }\hlstd{}\hlstd{\ \ \ \ \ \ \ \ \ \ \ \ }\hlstd{}\hlkwa{if\ }\hlstd{}\hlopt{(}\hlstd{sqll}\hlopt{.}\hlstd{}\hlkwd{contains}\hlstd{}\hlopt{(}\hlstd{s}\hlopt{))\ \symbol{123}}\\
\hllin{189\ }\hlstd{}\hlstd{\ \ \ \ \ \ \ \ \ \ \ \ \ \ \ \ }\hlstd{}\hlkwa{return\ true}\\
\hllin{190\ }\hlstd{}\hlstd{\ \ \ \ \ \ \ \ \ \ \ \ }\hlstd{}\hlopt{\symbol{125}}\\
\hllin{191\ }\hlstd{}\hlstd{\ \ \ \ \ \ \ \ }\hlstd{}\hlopt{\symbol{125}}\\
\hllin{192\ }\hlstd{}\hlstd{\ \ \ \ \ \ \ \ }\hlstd{}\hlkwa{return\ false}\\
\hllin{193\ }\hlstd{}\hlstd{\ \ \ \ }\hlstd{}\hlopt{\symbol{125}}\\
\hllin{194\ }\hlstd{}\hlstd{\ \ \ \ }\hlstd{}\hlkwa{override\ fun\ }\hlstd{}\hlkwd{itemAdded}\hlstd{}\hlopt{(}\hlstd{item}\hlopt{:\ }\hlstd{ItemEvent}\hlopt{\symbol{60}}\hlstd{SqlUpdateLog}\hlopt{\symbol{62})\ \symbol{123}}\\
\hllin{195\ }\hlstd{}\hlstd{\ \ \ \ \ \ \ \ }\hlstd{}\hlkwa{if\ }\hlstd{}\hlopt{(}\hlstd{item}\hlopt{.}\hlstd{item}\hlopt{.}\hlstd{LSN\ }\hlopt{\symbol{62}\ }\hlstd{locals\symbol{95}maxlsn}\hlopt{)\ \symbol{123}}\\
\hllin{196\ }\hlstd{}\hlstd{\ \ \ \ \ \ \ \ \ \ \ \ }\hlstd{localquene}\hlopt{.}\hlstd{}\hlkwd{offer}\hlstd{}\hlopt{(}\hlstd{item}\hlopt{.}\hlstd{item}\hlopt{)}\\
\hllin{197\ }\hlstd{}\hlstd{\ \ \ \ \ \ \ \ \ \ \ \ }\hlstd{}\hlkwa{if\ }\hlstd{}\hlopt{(!}\hlstd{isreplicating}\hlopt{)\ \symbol{123}}\\
\hllin{198\ }\hlstd{}\hlstd{\ \ \ \ \ \ \ \ \ \ \ \ \ \ \ \ }\hlstd{}\hlkwd{exesqlforremoteData}\hlstd{}\hlopt{()}\\
\hllin{199\ }\hlstd{}\hlstd{\ \ \ \ \ \ \ \ \ \ \ \ }\hlstd{}\hlopt{\symbol{125}}\\
\hllin{200\ }\hlstd{}\hlstd{\ \ \ \ \ \ \ \ }\hlstd{}\hlopt{\symbol{125}}\\
\hllin{201\ }\hlstd{}\hlstd{\ \ \ \ }\hlstd{}\hlopt{\symbol{125}}\\
\hllin{202\ }\hlstd{}\hlstd{\ \ \ \ }\hlstd{}\hlkwa{private\ fun\ }\hlstd{}\hlkwd{exesqlforremoteData}\hlstd{}\hlopt{()\ \symbol{123}}\\
\hllin{203\ }\hlstd{}\hlstd{\ \ \ \ \ \ \ \ }\hlstd{Collections}\hlopt{.}\hlstd{}\hlkwd{sort}\hlstd{}\hlopt{(}\hlstd{localquene}\hlopt{)}\\
\hllin{204\ }\hlstd{}\hlstd{\ \ \ \ \ \ \ \ }\hlstd{}\hlkwa{var\ }\hlstd{log}\hlopt{:\ }\hlstd{SqlUpdateLog?\ }\hlopt{=\ }\hlstd{localquene}\hlopt{.}\hlstd{}\hlkwd{poll}\hlstd{}\hlopt{()}\\
\hllin{205\ }\hlstd{}\hlstd{\ \ \ \ \ \ \ \ }\hlstd{}\hlkwa{var\ }\hlstd{last}\hlopt{:\ }\hlstd{}\hlkwb{Long\ }\hlstd{}\hlopt{=\ }\hlstd{}\hlnum{0}\\
\hllin{206\ }\hlstd{}\hlstd{\ \ \ \ \ \ \ \ }\hlstd{}\hlkwa{while\ }\hlstd{}\hlopt{(}\hlstd{log\ }\hlopt{!=\ }\hlstd{}\hlkwa{null}\hlstd{}\hlopt{)\ \symbol{123}}\\
\hllin{207\ }\hlstd{}\hlstd{\ \ \ \ \ \ \ \ \ \ \ \ }\hlstd{}\hlslc{//}\hlstd{\ \ \ \ \ \ \ \ \ \ \ \ }\hlslc{{-}{-}{-}{-}{-}{-}{-}{-}{-}{-}{-}{-}{-}{-}{-}{-}{-}{-}{-}{-}{-}{-}{-}{-}{-}{-}{-}{-}{-}{-}{-}{-}{-}{-}{-}{-}{-}{-}{-}{-}{-}{-}{-}{-}{-}{-}{-}{-}{-}{-}{-}{-}{-}{-}{-}{-}{-}{-}{-}{-}}\\
\hllin{208\ }\hlstd{}\hlstd{\ \ \ \ \ \ \ \ \ \ \ \ }\hlstd{locals\symbol{95}maxlsn\ }\hlopt{=\ }\hlstd{log}\hlopt{.}\hlstd{LSN}\\
\hllin{209\ }\hlstd{}\hlstd{\ \ \ \ \ \ \ \ \ \ \ \ }\hlstd{logger}\hlopt{.}\hlstd{}\hlkwd{info}\hlstd{}\hlopt{(}\hlstd{}\hlstr{"exesqlforremoteData()}\hlstd{\ \ }\hlstr{"}\hlstd{\ }\hlopt{+\ }\hlstd{log}\hlopt{)}\\
\hllin{210\ }\hlstd{}\hlstd{\ \ \ \ \ \ \ \ \ \ \ \ }\hlstd{oNullConnection}\hlopt{!!.}\hlstd{schema\ }\hlopt{=\ }\hlstd{log}\hlopt{.}\hlstd{db}\\
\hllin{211\ }\hlstd{}\hlstd{\ \ \ \ \ \ \ \ \ \ \ \ }\hlstd{sqLhander}\hlopt{.}\hlstd{}\hlkwd{handle}\hlstd{}\hlopt{(}\hlstd{log}\hlopt{,\ }\hlstd{oNullConnection}\hlopt{)}\\
\hllin{212\ }\hlstd{}\\
\hllin{213\ }\hlstd{}\hlstd{\ \ \ \ \ \ \ \ \ \ \ \ }\hlstd{}\hlslc{//}\hlstd{\ \ \ \ \ \ \ \ \ \ \ \ }\hlslc{logfiletest.offer(log);}\\
\hllin{214\ }\hlstd{}\hlstd{\ \ \ \ \ \ \ \ \ \ \ \ }\hlstd{logFile}\hlopt{!!.}\hlstd{}\hlkwd{write}\hlstd{}\hlopt{(}\hlstd{log}\hlopt{)}\\
\hllin{215\ }\hlstd{}\\
\hllin{216\ }\hlstd{}\hlstd{\ \ \ \ \ \ \ \ \ \ \ \ }\hlstd{last\ }\hlopt{=\ }\hlstd{log}\hlopt{.}\hlstd{LSN}\\
\hllin{217\ }\hlstd{}\hlstd{\ \ \ \ \ \ \ \ \ \ \ \ }\hlstd{log\ }\hlopt{=\ }\hlstd{localquene}\hlopt{.}\hlstd{}\hlkwd{poll}\hlstd{}\hlopt{()}\\
\hllin{218\ }\hlstd{}\hlstd{\ \ \ \ \ \ \ \ }\hlstd{}\hlopt{\symbol{125}}\\
\hllin{219\ }\hlstd{}\hlstd{\ \ \ \ \ \ \ \ }\hlstd{}\hlkwa{val\ }\hlstd{l\ }\hlopt{=\ }\hlstd{iAtomic\symbol{95}remote\symbol{95}lsn}\hlopt{!!.}\hlstd{}\hlkwd{get}\hlstd{}\hlopt{()}\\
\hllin{220\ }\hlstd{}\hlstd{\ \ \ \ \ \ \ \ }\hlstd{}\hlkwa{if\ }\hlstd{}\hlopt{(}\hlstd{last\ }\hlopt{\symbol{62}\ }\hlstd{l}\hlopt{)\ \symbol{123}}\\
\hllin{221\ }\hlstd{}\hlstd{\ \ \ \ \ \ \ \ \ \ \ \ }\hlstd{iAtomic\symbol{95}remote\symbol{95}lsn}\hlopt{!!.}\hlstd{}\hlkwd{set}\hlstd{}\hlopt{(}\hlstd{last}\hlopt{)}\\
\hllin{222\ }\hlstd{}\hlstd{\ \ \ \ \ \ \ \ }\hlstd{}\hlopt{\symbol{125}}\\
\hllin{223\ }\hlstd{}\\
\hllin{224\ }\hlstd{}\hlstd{\ \ \ \ }\hlstd{}\hlopt{\symbol{125}}\\
\hllin{225\ }\hlstd{}\\
\hllin{226\ }\hlstd{}\hlopt{\symbol{125}}\hlstd{}
\mbox{}
\normalfont
\normalsize


\section{数据审计模块的实现}
审计模块有关的功能全部在audit包下面实现,
表\ref{codepdf/audit}给出了audit包下面
每个类的具体的作用。
\pictable[htbp]{audit包下面每个类的作用}{}{codepdf/audit}
\subsection{审计数据库}
审计数据库用Elasticsearch来实现,
作为审计数据库,不能让用户随意的更改,
所以本系统更改了它的源代码,使得它只能增加数据和查找数据
不能更改和删除数据。其中主要存储的对象如图\ref{codepdf/audit}所示。
\subsection{审计管理器}
审计管理功能全部在audit下面实现,主要是存储本地的日志文件,
然后发布到审计数据库。
提供给前端可视化的接口。
\subsection{审计可视化模块的实现}
审计可视化模块的实现用到了grafana,
主要用来监控ELK中的数据。
\section{本章小结}
本文前面一章设计了本系统的架构图和各个模块的详细功能,本章给出
每个功能模块的具体实现。。