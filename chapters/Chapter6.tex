% !Mode:: "TeX:UTF-8"

\chapter{系统测试}
\section{测试环境}
本文的测试环境为个人的计算机。它的硬件和软件参与如下
\begin{enumerate}
	\item 硬件参数:
	\begin{enumerate}
		\item CPU:intel(R) Xeon(R) E5620
		\item 内存:8G
		\item 磁盘:500G
		\item 网络:100Mbit/s
	\end{enumerate}
\item 软件参数:
\begin{enumerate}
	\item 操作系统:windows10
	\item java版本:8.0
\end{enumerate}
\end{enumerate}
\section{功能测试}
\subsection{数据库功能测试}
功能测试设计的细节非常多,项目繁杂。这里将给出几个关键的功能测试的
条件、目的、步骤和结果。其他更为细节的,比如对失败的查询的测试,或其他
类似的,比如与插入、更新类似的删除数据,删除表,由于篇幅原因,就不再细
述。
\begin{enumerate}

	\item 系统启动测试
	\begin{enumerate}
		\item 测试条件:代码编码完成,系统功能正常
		\item 测试步骤:
		\begin{enumerate}
			\item 启动系统
			\item 用mysql客服端连接系统
		\end{enumerate}
	\item 	测试结果:系统能正常启动,而且mysql客服端也可以连接上系统
	\end{enumerate}



\item 工具语句测试
\begin{enumerate}
	\item 测试条件:系统启动成功,客服端连接上系统
	\item 测试步骤:
	\begin{enumerate}
		\item 打开命令行客服端
		\item 输入语句show databases,发送给服务器
	\end{enumerate}
	\item 	测试结果:系统返回当前所有的数据库。
\end{enumerate}	


\item 建表语句测试
\begin{enumerate}
	\item 测试条件:系统启动成功,客服端连接上系统
	\item 测试步骤:
	\begin{enumerate}
		\item 打开命令行客服端
		\item 输入语句create table test(id int,name	varchar(100))
		\item 回头,发送命令给服务器
	\end{enumerate}
	\item 	测试结果:系统返回正确,而且文件系统里面多了一个文件。
\end{enumerate}	


\item 插入语句测试
\begin{enumerate}
	\item 测试条件:系统启动成功,客服端连接上系统
	\item 测试步骤:
	\begin{enumerate}
		\item 打开命令行客服端
		\item 输入语句insert into test(id,name) values(1,'changhong');
		\item 回头,发送命令给服务器
	\end{enumerate}
	\item 	测试结果:系统返回正确
\end{enumerate}	


\item 查询语句测试
\begin{enumerate}
	\item 测试条件:系统启动成功,客服端连接上系统
	\item 测试步骤:
	\begin{enumerate}
		\item 打开命令行客服端
		\item 输入语句select * from test;
		\item 回头,发送命令给服务器
	\end{enumerate}
	\item 	测试结果:系统返回一行数据。
\end{enumerate}	



\item 更新语句测试
\begin{enumerate}
	\item 测试条件:系统启动成功,客服端连接上系统
	\item 测试步骤:
	\begin{enumerate}
		\item 打开命令行客服端
		\item 输入语句update test set name='changhong1'
		\item 回头,发送命令给服务器
		\item 输入语句select * from test
		\item 回头,发送命令给服务器
	\end{enumerate}
	\item 	测试结果:系统执行正确,返回一条数据,名字字段weichangohng1
\end{enumerate}	


\item 更新语句测试
\begin{enumerate}
	\item 测试条件:系统启动成功,客服端连接上系统
	\item 测试步骤:
	\begin{enumerate}
		\item 打开命令行客服端
		\item 输入语句delete from test
		\item 回头,发送命令给服务器
		\item 输入语句select * from test
		\item 回头,发送命令给服务器
	\end{enumerate}
	\item 	测试结果:系统执行正确,没有返回数据
\end{enumerate}	


\end{enumerate}	
\subsection{集群功能测试}
集群的功能测试和单机版一样,考虑到不可能每条语句的测试。
本次测试只测试最常用的insert语句:
\begin{enumerate}
	\item 测试条件:
	\begin{enumerate}
		\item 数据库1启动成功,没有任何数据
		\item 数据库2启动成功,没有任何数据
		\item 数据库1和数据库2都有一个test表
	\end{enumerate}
	\item 测试步骤:
	\begin{enumerate}
		\item 打开命令行客服端,连接数据库1
		\item 输入语句insert into test(id,name) values(1,'changhong')并且执行;
		\item 断开连接,连接数据库2
		\item 输入语句select * from test;
		\item 回头,发送命令给服务器
	\end{enumerate}
	\item 	测试结果
	\begin{enumerate}
		\item 数据库1执行语句正确
		\item 语句2执行正确,并且返回一条数据
	\end{enumerate}
\end{enumerate}	
\subsection{审计功能测试}
审计功能的测试主要是测试可视化的显示功能。
\begin{enumerate}
	\item 测试条件:
	\begin{enumerate}
		\item 数据库1启动成功,没有任何数据
		\item 数据库1有一个test表
	\end{enumerate}
	\item 测试步骤:
	\begin{enumerate}
		\item 打开命令行客服端,连接数据库1
		\item 输入语句insert into test(id,name) values(1,'changhong')并且执行;
		\item 打开grafana web界面
	\end{enumerate}
	\item 	测试结果
	\begin{enumerate}
		\item 数据库1执行语句正确
		\item 通过web界面可以观察到插入的数据和执行者的信息
		\item 图\ref{testpdf/jietu1.pdf},图
		\ref{testpdf/jietu2.pdf}和图
		\ref{testpdf/jietu3.pdf}显示了监控的可视化结果
	\end{enumerate}
\end{enumerate}	

   \pic[htbp]{监控可视化结果}{}{testpdf/jietu1.pdf}
      \pic[htbp]{监控可视化结果}{}{testpdf/jietu2.pdf}
         \pic[htbp]{监控可视化结果}{}{testpdf/jietu3.pdf}
\section{性能测试}
数据库的更删改查是最常用的操作,其中又以查找使用的场景更多,
所以本节主要测试数据库的查询性能。
jsql和mysql都可以通过JDBC来连接,下面的源代码是测试代码,
用来测试查询次数和执行时间的关系。执行结果如图\ref{testpdf/mysqltime.pdf}所示。

\noindent
\ttfamily
\hlstd{}\hllin{01\ }\hlstd{}\hlstd{\ \ \ }\hlstd{}\hlkwa{fun\ }\hlstd{}\hlkwd{test}\hlstd{}\hlopt{()\ \symbol{123}}\\
\hllin{02\ }\hlstd{}\hlstd{\ \ \ \ \ \ \ \ }\hlstd{}\hlslc{//sql执行次数}\\
\hllin{03\ }\hlstd{}\hlstd{\ \ \ \ \ \ \ \ }\hlstd{}\hlkwa{val\ }\hlstd{sqlnumber\ }\hlopt{=\ }\hlstd{}\hlkwd{intArrayOf}\hlstd{}\hlopt{(}\hlstd{}\hlnum{10}\hlstd{}\hlopt{,\ }\hlstd{}\hlnum{100}\hlstd{}\hlopt{,\ }\hlstd{}\hlnum{1000}\hlstd{}\hlopt{,\ }\hlstd{}\hlnum{3000}\hlstd{}\hlopt{,\ }\hlstd{}\hlnum{5000}\hlstd{}\hlopt{,\ }\hlstd{}\hlnum{10000}\hlstd{}\hlopt{)}\\
\hllin{04\ }\hlstd{}\hlstd{\ \ \ \ \ \ \ \ }\hlstd{}\hlkwa{val\ }\hlstd{mysql\ }\hlopt{=\ }\hlstd{}\hlstr{"jdbc:mysql://localhost:3306/changhong?"}\hlstd{\ }\hlopt{+\ }\hlstd{}\hlstr{"user=root\&"}\hlstd{\ }\hlopt{+}\\
\hllin{05\ }\hlstd{}\hlstd{\ \ \ \ \ \ \ \ \ \ \ \ \ \ \ \ }\hlstd{}\hlstr{"password=0000\&useUnicode="}\hlstd{\ }\hlopt{+\ }\hlstd{}\hlstr{"true\&characterEncoding=UTF8"}\hlstd{}\\
\hllin{06\ }\hlstd{}\hlstd{\ \ \ \ \ \ \ \ }\hlstd{}\hlkwa{var\ }\hlstd{connection\ }\hlopt{=\ }\hlstd{DriverManager}\hlopt{.}\hlstd{}\hlkwd{getConnection}\hlstd{}\hlopt{(}\hlstd{mysql}\hlopt{)}\\
\hllin{07\ }\hlstd{}\hlstd{\ \ \ \ \ \ \ \ }\hlstd{}\hlkwa{var\ }\hlstd{statement\ }\hlopt{=\ }\hlstd{connection}\hlopt{.}\hlstd{}\hlkwd{createStatement}\hlstd{}\hlopt{()}\\
\hllin{08\ }\hlstd{}\hlstd{\ \ \ \ \ \ \ \ }\hlstd{}\hlkwd{println}\hlstd{}\hlopt{(}\hlstd{}\hlstr{"mysql\ sqlnumber\ to\ times:"}\hlstd{}\hlopt{)}\\
\hllin{09\ }\hlstd{}\hlstd{\ \ \ \ \ \ \ \ }\hlstd{}\hlkwa{for\ }\hlstd{}\hlopt{(}\hlstd{number\ }\hlkwa{in\ }\hlstd{sqlnumber}\hlopt{)\ \symbol{123}}\\
\hllin{10\ }\hlstd{}\hlstd{\ \ \ \ \ \ \ \ \ \ \ \ }\hlstd{}\hlkwa{val\ }\hlstd{start\ }\hlopt{=\ }\hlstd{System}\hlopt{.}\hlstd{}\hlkwd{currentTimeMillis}\hlstd{}\hlopt{()}\\
\hllin{11\ }\hlstd{}\hlstd{\ \ \ \ \ \ \ \ \ \ \ \ }\hlstd{}\hlkwa{for\ }\hlstd{}\hlopt{(}\hlstd{i\ }\hlkwa{in\ }\hlstd{}\hlnum{1}\hlstd{}\hlopt{..}\hlstd{number}\hlopt{)\ \symbol{123}}\\
\hllin{12\ }\hlstd{}\hlstd{\ \ \ \ \ \ \ \ \ \ \ \ \ \ \ \ }\hlstd{statement}\hlopt{.}\hlstd{}\hlkwd{executeQuery}\hlstd{}\hlopt{(}\hlstd{}\hlstr{"select\ {*}\ from\ test"}\hlstd{}\hlopt{)}\\
\hllin{13\ }\hlstd{}\hlstd{\ \ \ \ \ \ \ \ \ \ \ \ }\hlstd{}\hlopt{\symbol{125}}\\
\hllin{14\ }\hlstd{}\hlstd{\ \ \ \ \ \ \ \ \ \ \ \ }\hlstd{}\hlkwa{val\ }\hlstd{end\ }\hlopt{=\ }\hlstd{System}\hlopt{.}\hlstd{}\hlkwd{currentTimeMillis}\hlstd{}\hlopt{()}\\
\hllin{15\ }\hlstd{}\hlstd{\ \ \ \ \ \ \ \ \ \ \ \ }\hlstd{}\hlkwd{println}\hlstd{}\hlopt{(}\hlstd{}\hlstr{"}\hlipl{\$number}\hlstd{\ \ \ \ \ \ }\hlipl{\$\symbol{123}end\ {-}\ start\symbol{125}}\hlstr{"}\hlstd{}\hlopt{)}\\
\hllin{16\ }\hlstd{}\hlstd{\ \ \ \ \ \ \ \ }\hlstd{}\hlopt{\symbol{125}}\\
\hllin{17\ }\hlstd{}\hlstd{\ \ \ \ \ \ \ \ }\hlstd{connection}\hlopt{.}\hlstd{}\hlkwd{close}\hlstd{}\hlopt{()}\\
\hllin{18\ }\hlstd{}\hlstd{\ \ \ \ \ \ \ \ }\hlstd{}\hlkwa{val\ }\hlstd{jsql\ }\hlopt{=\ }\hlstd{}\hlstr{"jdbc:mysql://localhost:9999/changhong?"}\hlstd{\ }\hlopt{+\ }\hlstd{}\hlstr{"user=root\&"}\hlstd{\ }\hlopt{+}\\
\hllin{19\ }\hlstd{}\hlstd{\ \ \ \ \ \ \ \ \ \ \ \ \ \ \ \ }\hlstd{}\hlstr{"password=0000\&useUnicode="}\hlstd{\ }\hlopt{+\ }\hlstd{}\hlstr{"true\&characterEncoding=UTF8"}\hlstd{}\\
\hllin{20\ }\hlstd{}\hlstd{\ \ \ \ \ \ \ \ }\hlstd{connection\ }\hlopt{=\ }\hlstd{DriverManager}\hlopt{.}\hlstd{}\hlkwd{getConnection}\hlstd{}\hlopt{(}\hlstd{jsql}\hlopt{)}\\
\hllin{21\ }\hlstd{}\hlstd{\ \ \ \ \ \ \ \ }\hlstd{statement\ }\hlopt{=\ }\hlstd{connection}\hlopt{.}\hlstd{}\hlkwd{createStatement}\hlstd{}\hlopt{()}\\
\hllin{22\ }\hlstd{}\hlstd{\ \ \ \ \ \ \ \ }\hlstd{}\hlkwd{println}\hlstd{}\hlopt{(}\hlstd{}\hlstr{"jsql\ sqlnumber\ to\ times:"}\hlstd{}\hlopt{)}\\
\hllin{23\ }\hlstd{}\hlstd{\ \ \ \ \ \ \ \ }\hlstd{}\hlkwa{for\ }\hlstd{}\hlopt{(}\hlstd{number\ }\hlkwa{in\ }\hlstd{sqlnumber}\hlopt{)\ \symbol{123}}\\
\hllin{24\ }\hlstd{}\hlstd{\ \ \ \ \ \ \ \ \ \ \ \ }\hlstd{}\hlkwa{val\ }\hlstd{start\ }\hlopt{=\ }\hlstd{System}\hlopt{.}\hlstd{}\hlkwd{currentTimeMillis}\hlstd{}\hlopt{()}\\
\hllin{25\ }\hlstd{}\hlstd{\ \ \ \ \ \ \ \ \ \ \ \ }\hlstd{}\hlkwa{for\ }\hlstd{}\hlopt{(}\hlstd{i\ }\hlkwa{in\ }\hlstd{}\hlnum{1}\hlstd{}\hlopt{..}\hlstd{number}\hlopt{)\ \symbol{123}}\\
\hllin{26\ }\hlstd{}\hlstd{\ \ \ \ \ \ \ \ \ \ \ \ \ \ \ \ }\hlstd{statement}\hlopt{.}\hlstd{}\hlkwd{executeQuery}\hlstd{}\hlopt{(}\hlstd{}\hlstr{"select\ {*}\ from\ test"}\hlstd{}\hlopt{)}\\
\hllin{27\ }\hlstd{}\hlstd{\ \ \ \ \ \ \ \ \ \ \ \ }\hlstd{}\hlopt{\symbol{125}}\\
\hllin{28\ }\hlstd{}\hlstd{\ \ \ \ \ \ \ \ \ \ \ \ }\hlstd{}\hlkwa{val\ }\hlstd{end\ }\hlopt{=\ }\hlstd{System}\hlopt{.}\hlstd{}\hlkwd{currentTimeMillis}\hlstd{}\hlopt{()}\\
\hllin{29\ }\hlstd{}\hlstd{\ \ \ \ \ \ \ \ \ \ \ \ }\hlstd{}\hlkwd{println}\hlstd{}\hlopt{(}\hlstd{}\hlstr{"}\hlipl{\$number}\hlstd{\ \ \ \ \ \ }\hlipl{\$\symbol{123}end\ {-}\ start\symbol{125}}\hlstr{"}\hlstd{}\hlopt{)}\\
\hllin{30\ }\hlstd{}\hlstd{\ \ \ \ \ \ \ \ }\hlstd{}\hlopt{\symbol{125}}\\
\hllin{31\ }\hlstd{}\hlstd{\ \ \ \ \ \ \ \ }\hlstd{connection}\hlopt{.}\hlstd{}\hlkwd{close}\hlstd{}\hlopt{()}\\
\hllin{32\ }\hlstd{}\hlstd{\ \ \ \ }\hlstd{}\hlopt{\symbol{125}}\\
\hllin{33\ }\hlstd{}\hlopt{\symbol{125}}\hlstd{}
\mbox{}
\normalfont
\normalsize


\pic[htbp]{mysql和jsql的性能比较}{}{testpdf/mysqltime.pdf}
从图\ref{testpdf/mysqltime.pdf}可以看出,随着SQL执行次数的增加,mysql和jsql的执行时
一般都响应的增加,在3000执行次数为3000以下时,jsql和mysql的性能相当,
当执行次数大于5000时,mysql的性能小幅度的超过jsql。
再从变化幅度来看,当执行语句是5000次数的时候,mysql的执行时候审计比执行3000次数的时候
还要少,这点可能是因为mysql本身的缓存或者优化有关。
而jsql就相对来说随着执行次数的增加,时间就随着增加,符合预期。

除了jdbc的测试以外,本次测试还用了一个流行的测试框架JMeter。
Apache JMeter是Apache组织开发的基于Java的压力测试工具。用于对软件做压力测试,
它最初被设计用于Web应用测试,但后来扩展到其他测试领域。
 它可以用于测试静态和动态资源,例如静态文件、Java
  小服务程序、CGI 脚本、Java 对象、数据库、FTP 服务器,
   等等。JMeter 可以用于对服务器、网络或对象模拟巨大的负载
   ,来自不同压力类别下测试它们的强度和分析整体性能
   另外,JMeter能够对应用程序做功能/回归测试
   ,通过创建带有断言的脚本来验证你的程序返回了你期望的结果。
   为了最大限度的灵活性,JMeter允许使用正则表达式创建断言。
   用JMeter测试jsql的结果如图\ref{testpdf/jmetjsql.pdf}所示。
    用JMeter测试jsql的结果如图\ref{testpdf/jmetmysql.pdf}所示。
   \pic[htbp]{jsql的JMeter性能测试}{}{testpdf/jmetjsql.pdf}
   \pic[htbp]{mysql的JMeter性能测试}{}{testpdf/jmetmysql.pdf}
   
   从测试结果来看,jsql相对mysql的性能有一定的差距。其中一个原因是jsql是用
   纯java语言开发的,而mysql用c语法开发,其中自然有一定的性能损耗。
\section{本章小结}
本章对JSQL数据库系统进行测试,主要分为功能测试和性能测试,功能测试
包括数据库功能测试,集群功能测试以及审计功能的测试。性能测试分为JDBC和JMeter测试
两部分。