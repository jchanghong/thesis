% !Mode:: "TeX:UTF-8"

\chapter{系统设计}
\section{总体架构设计}
jsql是一个可以替代mysql的分布式java数据库

...
\section{系统模块划分}
下面是分布式环境下的部署图:

\pic[htbp]{分布式环境下架构}{}{djsql}

图6-2是jsql的架构图


\subsection{数据库模块设计}
数据库主要是sqk到粗存

...
\subsection{集群架构设计}
主要用hazelcast

...
\subsection{数据审计模块设计}
数据库审计的

...
\subsubsection*{elasticsearch简介}
...
\subsubsection*{grafana简介}
...
\subsubsection*{审计的架构}
...

...
\section{各模块详细设计}
本文主要是

...
\subsection{数据库模块设计}
主要分为3个部分

...
\subsubsection*{网络模块的设计}
数据库主要是sqk到粗存

...
\subsubsection*{sql解析模块的设计}
主要用hazelcast

...
\subsubsection*{存储引擎模块的设计}
数据库审计的

...
\subsection{集群架构的设计}
主要用hazelcast


数据库审计的

...
\subsection{数据审计架构的设计}
数据库是任何商业和公共安全中最具有战略性的资产,通常都保存着重要的
商业伙伴和客户信息,这些信息需要被保护起来,以防止竞争者和其他非法者获
酬12】。互联网的急速发展使得企业数据库信息的价值及可访问性得到了提升,
同时,也致使数据库信息资产面临严峻的挑战,概括起来主要表现在以下三个层
面:

1.管理风险:主要表现为人员的职责、流程有待完善,内部员工的日常操作
有待规范,第三方维护人员的操作监控失效等等,离职员工的后门,致使安全事
件发生时,无法追溯并定位真实的操作者。其中典型的例子,曾为西藏移动进行
设备安装工作的原深圳某高科技公司的工程师利用它为西藏移动做技术时使用
的密码(此密码自工程师离开后一直没有更改),轻松进入了西藏移动的服务器,
再跳转到北京移动的服务器,从而通过修改系统的数据库轻松获得了14000个充
值卡密码并获得380万元的利益。

2.技术风险:Oracle,SQL Server是一个庞大而复杂的系统,安全漏洞如溢
出, 注入层出不穷,每一次的CPUIl4J(Critical Patch Update)都疲丁奔命,
 而
企业和政府处于稳定性考虑,往往对补丁的跟进非常延后,更何况通过应用层的
注入攻击使得数据库处于一个无辜受害的状态。

3.审计层面:现有的依赖于数据库日志文件的审计方法,存在诸多的弊端,
比如:数据库审计功能的开启会影响数据库本身的性能、数据库日志文件本身存
在被篡改的风险,难于体现审计信息的有效性和公正性。此外,对于海量数据的
挖掘和迅速定位也是任何审计系统必须面对和解决的一个核心问题之一。
伴随着数据库信息价值以及可访问性提升,使得数据库面对来自内部和外部
的安全风险大大增加,如违规越权操作、恶意入侵导致机密信息窃取泄漏,但事
后却无法有效追溯和审计。

系统审计实现的工作主要分三项:一是对数据库源代码进行必要修改,管理子系统可以存储
监控信息;二是审计监听器的编写;三是
审计规则和记录管理器的编写,这部分需要实现图形用户界面。
\subsubsection*{审计数据库设计}
通过改写数据库源代码,使得数据库所有的更改都记录下来。
任何人都没有权限删除记录。保证了记录的可靠性和安全性。
\subsubsection*{审计管理器设计}
分为安全规则的设计和实现;
安全审计的分析;
安全审计的查询。
安全审计的报警功能
\subsubsection*{审计可视化模块的实现}
主要是图形化的界面。实时的查询审计信息。
实时的查看数据库的状态和实时的报警功能。
\section{本章小结}
分布式数据库

...