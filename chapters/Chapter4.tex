% !Mode:: "TeX:UTF-8"

\chapter{系统设计}
在软件工程中,需求分析之后就是设计阶段。首先,开发者需要对
软件系统进行概要设计,即总体和框架设计。概要设计需要对软件系统的总体功能进行设计,包括系统的基本处理流程、系统的组织结构和功能模块划分。,为软件的详细设计提供基础。
在概要设计的基础上,然后需要进行软件系统的详细设计。
在系统的详细设计中,描述实现具体模块所涉及到的主要算法、数据结构和功能流程,需要说明软件系统各个层次中的每一个
程序的设计考
虑,方便以后的编码和测试工作。应当保证软件的需求完全分配给整个软
件。详细设计应当足够详细,应该能够根据详细设计进行具体的编码。
只有经过仔细的系统设计,在编码阶段才能提前减少不必要的错误。
\section{系统总体设计}
\subsection{系统架构设计}
在系统设计中,先进行原型化设计可以对系统整体功能进行大概的说明,有利于对整体的功能把握。
原型化设计就是先构建一个系统的小版本,通常只包括整体功能中有限的功能,突出整体中的关键功能,其可用于:
\begin{enumerate}
	\item 帮助用户和客户标识系统的关键功能需求,忽视具体的细节,从而利于把握整体。
	\item 证明设计或方法的可行性,方便对系统做进一步的总体设计。
\end{enumerate}
通常,原型化过程是迭代的过程:首先构建一个原型,然后对这个原型进行评估,考虑如何对原型设计进行改进,之后再构建另外一个原
型,如此循环迭代。当客户认为原型解决方案满意时,迭代过程就终止了,可以进一步完成系统的设计工作。
由于本文所涉及的分布式数据库系统是作为实验室数据库,并没有做实际上线的考虑,
因此采用了原型化模型的软件开发方式开发一个数
据库原型。重点针对分布式数据库系统的大容量可扩展性进行设
计,主要完成数据库的增加、查询、修改、删除等功能,所以对系统的架构作了
简化调整。系统的网络拓扑结构如图\ref{pic4/bushu}所示。
\pic[htbp]{系统网络拓扑结构图}{}{pic4/bushu}
在系统网络拓扑图中,主要有下面几种节点:


	\begin{enumerate}
		\item 客服端节点
	\end{enumerate}

	客服端节点作为连接分布式数据库服务器的代理,发送SQL命令请求到数据库服务器,请求数据,客服端节点可以是原生的Mysql节点,也可以是自己开发的具有负载均衡的JDBC客服端。如果是用Mysql客服端,就可以直接连接分布式数据库节点,可以任意选择一台分布式数据库节点执行数据库操作命令,其中数据库会自动同步到其他数据库节点。当用自己开发的JDBC客服端的时候,客服端就会首先连接分布式管理节点,得到具体的分布式数据库节点后,才能连接数据库节点执行具体的数据库命令。

	\begin{enumerate}[resume]
		\item 分布式管理节点
	\end{enumerate}

	分布式管理节点管理分布式数据库节点的元数据和实现负载均衡算法。管理计算机通过连接分布式管理节点,可以对分布式节点进行管理,对系统进行监控。这样当客服端请求到来的适合,分布式管理节点就可以返回适合的分布式数据库节点,从而达到负载均衡的效果。

	\begin{enumerate}[resume]
		\item 分布式数据库节点
	\end{enumerate}

	分布式节点存储具体的数据,数据库系统功能主要在这个节点上面执行,响应客服端的增删改查的命令请求。当客服端连接分布式数据库节点以后,会检查客服端的权限,只有有权限的用户才能执行命令。

	\begin{enumerate}[resume]
		\item 管理计算机
	\end{enumerate}

	管理员通过管理计算机管理分布式数据库几点和负载均衡算法。也可以通过管理计算机来查看审计信息,设置监控报警功能。


在系统网络拓扑结构图中,客户端模拟负责发送命令的终端设备,各个
分布式数据库节点和分布式管理节点构成了系统的服务器端。其中,分布
式管理节点作为中心节点接收所有的控制消息,
通过分布式管理节点才能找到分布式数据库节点的IP地址
。得到地址以后,客服端再连接数据库节点
,得到具体的数据,
分布式管理节点通过选举参数,也就是第一次启动的服务器节点,同时也能存储数据库数据。
根据原型化模型和系统的网络拓扑结构图,本文所涉及的分布式数据库
系统的整体结构是:有两种计算机,一台作为客户端使用,
另一台作为服务器使
用,根据需求设定服务器端的数据库节点数个数N。这N存数据库之
间相互独立、存储各自的数据信息。它们的数据结构、存储方式都是相同的,由
服务器端的分布式管理节点负责把客户端发来的数据分发到对应的数据库节点,
因此不同的数据库节点中存放着不同的数据。每一个数据库节点构成了自己的
数据库单元,每个数据库单元具有独立处理的能力,它可以执行局部应用。
同时,如果这 N 个数据库物理上应用于多台计算机时,每个数据库节
点也能通过网络通信子系统执行全局应用。本系统的架构图如图\ref{pic4/jiagou}所示。
\pic[htbp]{系统总体架构图}{}{pic4/jiagou}
本系统中,客户端模拟真实终端设备。
分布式管理节点收到客服端请求以后,
返回给客服端数据库节点的地址,然后客服端再连接数据库节点,
数据库节点负责对请求消息进行处
理,包括增加、删除、修改、查询等操作,并将响应消息返回给客户端。分布式
管理节点与各个数据库模块在同一台机器中运行,也可以分布到不同的计算机上面。
在系统可靠性保护方面,当出现某个数据库模块发生故障时,分布式管理节点
会实时的更新管理信息,同时其他数据库节点也会同步这台计算机的信息。
数据因为在不同的计算机上面都有备份,所以不存储丢失的危险。
\subsection{系统功能设计}
根据系统的需求分析,采用模块化分解结构划分本系统,使系统能够进行增
加、删除、查询、修改等数据库基本操作。系统采用传统的C/S架构,分布式管
理节点和数据库节点共同组成了服务器端。分布式数据库系统的功能模
块分解图如图\ref{pic4/mokuai}所示。
\pic[htbp]{系统总体架构图}{}{pic4/mokuai}

以下对图\ref{pic4/mokuai}中服务器端各个模块的功能进行简要说明。
分布式管理节点按逻辑功能划分为两个子模块,元数据管理模块、审计界面模块。
下面分别对两个子模块的功能作介绍。
\begin{enumerate}[fullwidth,itemindent=2em,listparindent=2em]
	\item 元数据管理模块管理服务器的所有元数据,包括每个数据库节点的
	IP地址和端口号。分布式管理节点利用这些元数据来完成分布式的负载均衡算法,为客户端选择合适的
	分布式数据库节点。
	\item 审计界面管理模块对管理员使用,它可以让管理员管理所有的数据节点。
	审计功能除了审计界面以外,还包括分布式数据库节点的审计数据库。
	只有结合了分布式数据库的审计模块和分布式管理节点的审计模块,管理员才能对分布式系统进行审计管理。
\end{enumerate}

分布式数据库节点由审计数据库模块、数据库功能模块、集群功能模块组成,
 下面分别对 3 个模块的功能进行介绍。
 \begin{enumerate}[fullwidth,itemindent=2em,listparindent=2em]
 	\item 审计数据库模块用来收集所有的日志信息,作为审计使用。
 	所有的数据库节点在接受到数据更改请求以后,就会把这些日志信息发送到审计数据库。
 	然后分布式管理节点就能利用这个数据为管理员提供审计功能。
 	\item 数据库功能模块完成具体的数据库功能。为用户提供操作接口,数据库功能模块可以分为网络模块,协议模块,解析模块和存储模块。在收到用户的请求命令以后。数据库功能模块就会对用户返回正确的信息。
 	\item 集群功能实现每个数据库节点的通信。本系统利用Hazelcast实现了数据库的集群。每个分布式数据库节点的数据都能自动同步。同时,利用分布式多版本并发控制技术解决了数据的一致性问题。
 \end{enumerate}
\section{客服端模块设计}
分布式系统需要解决的一个重要问题便是决定数据在集群中的分布策略,好
的分布策略应该能将数据均衡地分布到所有节点上,并且还应该能适应集群节点
的变化,本文采用的分布式管理节点较好地满足了这两点。分布式管理节点存储
所有数据库节点的元数据信息,实现了分布式负载均衡算法。
对于分布式系统,系统中每个节点的负载均衡是非常重要的,分布式算法必
须具有良好的分散性,使消息能够均匀的转发到各个数据库节点。分布式管理节点实现了负载均衡功能,
所以当我们想要使用负载均衡的时候我们必须要先连接管理节点,再连接数据库节点。
客服端连接的流程图如图\ref{pic4/lianjielc}所示。
\pic[htbp]{客服端连接流程图}{}{pic4/lianjielc}
每个流程解释如下:
\begin{enumerate}[fullwidth,itemindent=2em,listparindent=2em]
	\item 首先客服端连接到分布式管理节点,管理节点返回给客服端
	正确的数据库节点的地址。
	\item 然后客服端连接到正确的数据库节点。
	\item 最后数据库节点返回给客服端具体的数据。
\end{enumerate}
当然我们也可以直接连接分布式数据库节点,其中的数据会自动同步到其他的分布式数据库节点,这样可以直接用Mysql的客服端,减少用户的使用成本。
使用Mysql客服端,可以执行几乎所有的Mysql命令。在用户看来,分布式数据库节点就像一个Mysql数据库服务器一样。只是本系统能自动同步每个分布式数据库节点的数据。
\section{分布式管理模块}
要实现负载均衡和其他管理功能,只实现客服端是不够的。还需要一个分布式管理节点。
分布式管理模块存储所有数据库节点的元数据和实现负载均衡算法,
当启动分布式管理节点的时候,它会去查找所有的数据库节点,
然后存储到数据库里面保存。
分布式管理节点的启动流程图如图\ref{pic4/fenbushiguanli}所示。
\pic[htbp]{分布式管理节点初始化过程}{}{pic4/fenbushiguanli}
分布式管理节点和分布式数据库节点一样,都是服务器端的节点,都实现了同一个分布式功能。他们之间可以通过多播通信来交换信息。分布式管理节点得到这些信息以后。就能实现正确的负载均衡算法来选择合适的分布式数据库节点。
分布式管理节点初始化成功以后,客服端就可以通过自己开发的基于JDBC的客服端来连接后端的分布式数据库节点,发送数据查询命令。
\section{数据库功能模块详细设计}
数据库功能模块主要负责根据命令消息类型来完成对应的业务逻辑
处理。本模块利用资源管理模块提供的接口,完成数据的增加、删除、修改、
查询等业务功能,并将响应消息返回给客户端。下面设计出每个操作业务的流程。
\subsection{数据库功能操作流程}
\subsubsection{新增数据过程}
第一步:客户端发送新增数据请求消息,消息包到达服务器端的分布式管理
节点,由分布式管理节点返回客服端的连接会话信息,客服端得到服务器的地址。
第二步:客服端连接具体的数据库节点,发送新增数据请求,请求发送到具体的数据库节点。
数据库节点经过网络模块和通信模块,得到请求信息以后,调用数据库引擎的接口,
得到具体的数据,返回给客服。
\subsubsection{数据查询和修改过程}
第一步:客户端发送查询和修改数据请求消息,消息包到达服务器端的分布式管理
节点,由分布式管理节点返回客服端的连接会话信息,客服端得到服务器的地址。
第二步:客服端连接具体的数据库节点,发送新增数据请求,请求发送到具体的数据库节点。
数据库节点经过网络模块和通信模块,得到请求信息以后,调用数据库引擎的接口,
得到具体的数据,返回给客服。
\subsubsection{数据删除过程}
第一步:客户端发送删除数据请求消息,消息包到达服务器端的分布式管理
节点,由分布式管理节点返回客服端的连接会话信息,客服端得到服务器的地址。
第二步:客服端连接具体的数据库节点,发送新增数据请求,请求发送到具体的数据库节点。
数据库节点经过网络模块和通信模块,得到请求信息以后,调用数据库引擎的接口,
得到具体的数据,返回给客服。
数据库模块是本系统的主要开发模块,其详细模块图如图\ref{pic4/shujuku}所示。
\pic[htbp]{数据库功能模块图}{}{pic4/shujuku}
下面对每个功能模块进行详细的说明。
\subsection{网络模块设计}
所有的服务软件都需要实现网络模块,这样才能连接客服端的请求。
JSQL用java语言开发,主要用到的是java的网络开发模块。
但是直接用java语言开发网络功能非常的麻烦,
所以本文基于Netty来开发直接的网络模块。具体的网络实现,包括
线程池的规划都按照Netty的建议来开发。Netty的线程模型如图\ref{netty}所示。
\pic[htbp]{Netty线程模型}{}{netty}
根据图\ref{netty}所示,netty主要有三种线程:
\begin{enumerate}[fullwidth,itemindent=2em,listparindent=2em]
	\item 主线程池主要接受客服端的网络连接请求,当主线程接受到客服端的请求以后,就可以把客服端的请求交个从线程池来处理。这样主线程就可以接受更多的其他客服端的连接请求。使得服务器的吞吐量大大的提高。
	\item 从线程从主线程得到客服端的连接以后,执行具体的读取操作。如果需要执行其他的事务逻辑,比如说读取磁盘,那么就需要把这个请求发送给工作线程来处理,这样客服端才不会因为从线程的忙碌而堵塞。
	\item 工作线程主要用来执行时间消耗比较大的任何,比如任何和磁盘文件有关的操作都应该放在工作线程来处理。
\end{enumerate}
通过使用合适的线程模型,服务器具有很高的吞吐量,能接受成千上万的客服端连接。
\subsection{通信协议设计}
每个服务器和客服端通信都要实现自己的通信协议,考虑到mysql使用的广泛性
。jsql采用mysql的通信协议。这样就不用自己开发协议了。
在MySQL数据库通信过程开始时,服务器会使用TCP监听一个本地socket
端口或本地socket链接。当一个客户端的连接请求到达,就会执行握手和权限验
证。如果验证成功,会话开始。客户端发送消息,服务器会以一个适合该发送命
令的数据类型的数据集或一条消息进行回复。当客户端发送完成后,会发送一个
特殊的命令,告诉服务器己发送,然后会话结束。通信的基本单位是应用程序包。
多个指令可以合成一个包;答复可以包含多个包。
MySQL客户端与服务器的交互主要分为两个阶段:握手认证阶段和命令执
行阶段。

\begin{enumerate}[fullwidth,itemindent=2em,listparindent=2em]
	\item 握手认证阶段\\
	握手认证阶段为客户端与服务器建立连接后进行,交互过程如下:
	\begin{enumerate}
		\item 服务器发送给客户端握手初始化报文。
		\item 客服端回复服务器端登陆认证报文。
		\item 服务器发送给客户端认证结果报文。
	\end{enumerate}
	\item 命令执行阶段\\
	客户端认证成功后,会进入命令执行阶段,交互过程如下:
	\begin{enumerate}
		\item 	客户端发送服务器执行命令报文。
		\item 服务器发送给客服端命令执行结果。
	\end{enumerate}
\end{enumerate}

MySQL客户端与服务器之间的完整交互过程如图\ref{mysqljh}所示。
\pic[htbp]{协议交互流程图}{}{mysqljh}

每个报文分为报文头和报文数据两部分,其中报文头占用固定的4个字节,报文
数据长度由报文头中的长度字段决定。报文头后面三个字节
用于标记当前请求报文的实际数据长度值,以字节为单位,
最大值为0xFFFFFF,即接近16 MB大小(比16MB少1个字节)。
除了报文长度还有报文序列号,在一次完整的请求和响应交互过程中,用于保证报文顺序的正确,每次客户
端发起请求时,序号值都会从0开始计算。
每个报文后面的字节就是报文数据,
报文数据用于存放请求的内容及响应的数据,长度由报文头中的长度值决定。
\subsection{SQL引擎设计}
本文基于成熟的开源框架Druid
来实现自己的解析模块,解析流程如图\ref{pic4/sqljx}所示。
\pic[htbp]{SQL解析流程图}{}{pic4/sqljx}
SQL引擎层从协议层解析到语句以后,就要对这个语句进行解析,才能交给下面的存储引擎处理具体的请求。解析模块首先对语句进行词法分析和语法分析,得到抽象语法树,然后用访问者模式去遍历抽象语法树,得到我们需要的数据结构。这个结构被封装成类,每种语句都有不同的类对应。得到语句类型以后,我们就可以对他进行具体的分析和处理。最后交个存储引擎层。
\subsection{存储引擎设计}
存储引擎是存储系统的发动机,直接决定了存储系统能够提供的性能和功能。存储
系统的基本功能包括:增、删、读、改,其中,读取操作又分为随机读取和顺序扫描。
每种存储引擎底层都基于一种数据结构。比如常用的哈希表结构和B+树结构。
本系统采用了一种NoSQL数据库引擎,这样可以结合关系数据库和非关系型数据库的优点。在OrientDB这个存储引擎基础之上,封装成关系型的操作。完成关系型表和NoSQL数据库的转化。对上一层来说,我们提供了关系型存储引擎的接口调用。在存储引擎的底层,我们调用OrientDB的接口来完成具体的功能。这样能结合关系型操作的便利和非关系型数据库的优点。
\section{集群架构详细设计}
要实现集群功能,肯定要用到网络功能,
还要考虑各种不可靠的因素。
为了系统的稳定性和可靠性,
本文用到了一个开源的分布式的开发框架hazelcast,
基于hazelcast的应用程序的结构图如图\ref{hazelcast}所示。
\pic[htbp]{集群架构图}{}{hazelcast}
分布式数据库节点和分布式管理节点都嵌入了Hazelcast模块,
这样分布式管理节点就能很容易的得到每个数据库节点的详细信息,
直接调用框架的接口就可以得到这些信息。
hazelcast分布式原理如图\ref{pic4/fenbusyl}所示。
\pic[htbp]{集群架构图}{}{pic4/fenbusyl}
在分布式数据库集群中,不用设计主节点,集群会自己发现和设置主节点。
在一个集群中,最先启动的计算机就是主节点,主节点拥有其他所有节点的信息。
当其他节点启动的时候,通过多播技术,加入多播网络。主节点发现以后,就会把自己的信息和集群所有其他计算机的信息发送给这个新的计算机,同时更新集群的元数据。当第一个主节点关掉的时候,再一次启动的计算机就自动成为主节点。
这样就不需要用户频繁的设置和更新元数据。分布式数据库集群在增加和删除节点的时候,不需要任何人为的干预。
\section{审计模块详细设计}
审计模块部署在管理计算机和分布式管理节点上面。管理员通过管理计算机连接分布式管理节点就可以管理所有的数据。监控所有数据的更改情况。
图\ref{jiankong}显示了审计模块的实现结构图。从上大下一共可以分为下面4个模块:
\begin{enumerate}[fullwidth,itemindent=2em,listparindent=2em]
	\item grafana可视化和报警模块,主要用到了开源的图形框架,来显示底层的审计数据,
	同时也可以让用户设定预警数据,
	这样以后就可以在一定的情况下向用户报警。
	\item elasticsearch模块用来存储监控数据,监控数据为了登陆日志和sql执行日志
	我们需要更改它的源代码才能用来作为审计数据库,
	因为审计数据库不能随意的更改和删除,只能查询和增加。
	\item JSQL接口层,主要是为上面和下面的各层提供连接功能,
	为上面层提供数据接口,为下面层提供显示接口,使得更加容易使用。
	\item 本地日志层,主要是在文件系统中存储所有需要的日志记录,这样可以随时和审计数据库的数据来
	对比。审计数据的安全和一致性。
\end{enumerate}
\pic[htbp]{监控模块}{}{jiankong}
\section{本章小结}
任何系统只有经过仔细的系统设计,在编码阶段才能提前减少不必要的错误。
本章从总体上设计了系统的部署模式和模块图,对每个模块进行了详细了的
模块设计,便于后面的具体实现。