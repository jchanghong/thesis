% !Mode:: "TeX:UTF-8"

\chapter{系统设计}
在软件工程中,需求分析之后就是设计阶段。首先,开发者需要对
软件系统进行概要设计,即系统设计。概要设计需要对软件系统的设计进行
考虑,包括系统的基本处理流程、系统的组织结构、模块划分、功
能分配、接口设计、运行设计、数据结构设计和出错处理设计等,
为软件的详细设计提供基础。
在概要设计的基础上,开发者需要进行软件系统的详细设计。
在详细设计中,描述实现具体模块所涉及到的主要算法、数据结构、类的层
次结构及调用关系,需要说明软件系统各个层次中的每一个
程序(每个模块或子程序)的设计考
虑,以便进行编码和测试。应当保证软件的需求完全分配给整个软
件。详细设计应当足够详细,能够根据详细设计报告进行编码。
只有经过仔细的系统设计,在编码阶段才能提前减少不必要的错误。
\section{系统总体设计}
\subsection{系统架构设计}
原型化设计意味着构建一个系统的小版本,通常只有有限的功能,它可用于:
\begin{enumerate}
	\item 帮助用户和客户标识系统的关键需求。
	\item 证明设计或方法的可行性。
\end{enumerate}
通常,原型化过程是迭代的:首先构建原型,然后对原型进行评估(利用用
户和客户的反馈),考虑如何改变可以改进产品或设计,之后再构建另外一个原
型。当开发团队和客户认为解决方案满意时,迭代过程就终止了。
由于本文所涉及的分布式数据库系统是作为预研项目的前期工作,不能
确定是否一定能够构造系统,因此采用了原型化模型的软件开发方式开发一个数
据库原型。重点针对分布式数据库系统的大容量可扩展性进行设
计,主要完成数据库的增加、查询、修改、删除等功能,所以对系统的架构作了
简化调整。系统的网络拓扑结构如图\ref{pic4/bushu}所示。
\pic[htbp]{系统网络拓扑结构图}{}{pic4/bushu}

在系统网络拓扑结构图中,客户端模拟负责发送命令的终端设备,各个
分布式数据库节点和分布式管理节点构成了系统的服务器端。其中,分布
式管理节点作为中心节点接收所有的控制消息,
通过分布式管理节点才能找到分布式数据库节点的IP地址
。得到地址以后,客服端再连接数据库节点
,得到具体的数据,
分布式管理节点并不存储数据信息,数据只存储在分布式存储节点中。
根据原型化模型和系统的网络拓扑结构图,本文所涉及的分布式数据库
系统的整体结构是:有两种计算机,一台作为客户端使用,
另一台作为服务器使
用,根据需求设定服务器端的数据库节点数个数N。这N存数据库之
间相互独立、存储各自的数据信息。它们的数据结构、存储方式都是相同的,由
服务器端的分布式管理节点负责把客户端发来的数据分发到对应的数据库节点,
因此不同的数据库节点中存放着不同的数据。每一个数据库节点构成了自己的
数据库单元,每个数据库单元具有独立处理的能力,它可以执行局部应用。
同时,如果这 N 个数据库物理上应用于多台计算机时,每个数据库节
点也能通过网络通信子系统执行全局应用。本系统的架构图如图\ref{pic4/jiagou}所示。
\pic[htbp]{系统总体架构图}{}{pic4/jiagou}

本系统中,客户端模拟真实终端设备。
分布式管理节点收到客服端请求以后,
返回给客服端数据库节点的地址,然后客服端再连接数据库节点,
数据库节点负责对请求消息进行处
理,包括增加、删除、修改、查询等操作,并将响应消息返回给客户端。分布式
管理节点与各个数据库模块在同一台机器中运行,也可以分布到不同的计算机上面。
在系统可靠性保护方面,当出现某个数据库模块发生故障时,分布式管理节点
会实时的更新管理信息,同时其他数据库节点也会同步这台计算机的信息。
数据因为在不同的计算机上面都有备份,所有不存储丢失的危险。
\subsection{系统功能设计}
根据系统的需求分析,采用模块化分解结构划分本系统,使系统能够进行增
加、删除、查询、修改等数据库基本操作。系统采用传统的C/S架构,分布式管
理节点和数据库节点共同组成了服务器端。分布式数据库系统的功能模
块分解图如图\ref{pic4/mokuai}所示。
\pic[htbp]{系统总体架构图}{}{pic4/mokuai}

以下对图\ref{pic4/mokuai}中服务器端各个模块的功能进行简要说明。
分布式管理节点按逻辑功能划分为两个子模块,元数据管理模块、审计界面模块。
下面分别对两个子模块的功能作介绍。
\begin{enumerate}
	\item 元数据管理模块管理服务器的所有元数据,包括每个数据库节点的
	IP地址和端口号。
	\item 审计界面管理模块对管理员使用,它可以让管理员管理所有的数据节点。
\end{enumerate}

分布式数据库节点由审计数据库模块、数据库功能模块、集群功能模块组成,
 下面分别对 4 个模块的功能进行介绍。
 \begin{enumerate}
 	\item 审计数据库模块用来收集所有的日志信息,作为审计使用。
 	\item 数据库功能模块完成具体的数据库功能,对所有的数据操作
 	都做了实现。
 	\item 集群功能实现每个数据库节点的通信。
 \end{enumerate}
\section{客服端模块设计}
分布式系统需要解决的一个重要问题便是决定数据在集群中的分布策略,好
的分布策略应该能将数据均衡地分布到所有节点上,并且还应该能适应集群节点
的变化,本文采用的分布式管理节点较好地满足了这两点。分布式管理节点存储
所有数据库节点的元数据信息,同时也能存储负载均衡的算法信息,管理员
可以决定采用什么算法那分配服务器资源。
对于分布式系统,系统中每个节点的负载均衡是非常重要的,分布式算法必
须具有良好的分散性,使消息能够均匀的转发到各个数据库节点。哈希算法
对分散性的支持很好的满足了系统对分散性的要求,并且利用哈希算法能够保证
同一号码经过哈希算法计算后,每次的哈希值都是相同的,这可以保证分布式管
理节点对同一数据标识的消息多次转发时,每次都能将消息正确转发到同一
数据库中。但是随着数据库节点数量的变化,客服端必须要重新计算服务器节点地址,
所以我们必须要先连接管理节点,再连接数据库节点。
客服端连接的流程图如图\ref{pic4/lianjielc}所示。
\pic[htbp]{客服端连接流程图}{}{pic4/lianjielc}
每个流程解释如下:
\begin{enumerate}
	\item 首先客服端连接到分布式管理节点,管理节点返回给客服端
	正确的数据库节点的地址。
	\item 然后客服端连接到正确的数据库节点。
	\item 最后数据库节点返回给客服端具体的数据。
\end{enumerate}
\section{分布式管理模块}
分布式管理模块存储所有数据库节点的元数据和负载均衡的算法,
当启动分布式管理节点的时候,它会去查找所有的数据库节点,
然后存储到数据库里面保存。同时管理员也可以对负载均的算法进行
管理。
分布式管理节点的启动流程图如图\ref{pic4/fenbushiguanli}所示。
\pic[htbp]{分布式管理节点初始化过程}{}{pic4/fenbushiguanli}
\section{数据库功能模块详细设计}
数据库功能模块主要负责根据命令消息类型来完成对应的业务逻辑
处理。本模块利用资源管理模块提供的接口,完成数据的增加、删除、修改、
查询等业务功能,并将响应消息返回给客户端。下面设计出每个操作业务的流程。
\subsection{业务流程设计}
\subsubsection{新增数据过程}
第一步:客户端发送新增数据请求消息,消息包到达服务器端的分布式管理
节点,由分布式管理节点返回客服端的连接会话信息,客服端得到服务器的地址。
第二步:客服端连接具体的数据库节点,发送新增数据请求,请求发送到具体的数据库节点。
数据库节点经过网络模块和通信模块,得到请求信息以后,调用数据库引擎的接口,
得到具体的数据,返回给客服。
\subsubsection{数据查询和修改过程}
第一步:客户端发送查询和修改数据请求消息,消息包到达服务器端的分布式管理
节点,由分布式管理节点返回客服端的连接会话信息,客服端得到服务器的地址。
第二步:客服端连接具体的数据库节点,发送新增数据请求,请求发送到具体的数据库节点。
数据库节点经过网络模块和通信模块,得到请求信息以后,调用数据库引擎的接口,
得到具体的数据,返回给客服。
\subsubsection{数据删除过程}
第一步:客户端发送删除数据请求消息,消息包到达服务器端的分布式管理
节点,由分布式管理节点返回客服端的连接会话信息,客服端得到服务器的地址。
第二步:客服端连接具体的数据库节点,发送新增数据请求,请求发送到具体的数据库节点。
数据库节点经过网络模块和通信模块,得到请求信息以后,调用数据库引擎的接口,
得到具体的数据,返回给客服。
数据库模块是本系统的主要开发模块,其详细模块图如图\ref{pic4/shujuku}所示。
\pic[htbp]{数据库功能模块图}{}{pic4/shujuku}
下面对每个功能模块进行详细的说明。
\subsection{网络模块设计}
所有的服务软件都需要实现网络模块,这样才能连接客服端的请求。
JSQL用java语言开发,主要用到的是java的网络开发模块。
但是直接用java语言开发网络功能非常的麻烦,
所以本文基于Netty来开发直接的网络模块。具体的网络实现,包括
线程池的规划都按照Netty的建议来开发。
\subsection{通信协议设计}
每个服务器和客服端通信都要实现自己的通信协议,考虑到mysql使用的广泛性
。jsql采用mysql的通信协议。这样就不用自己开发协议了。图\ref{mysqljh}显示了
mysql协议中客服端和服务器之间的交互过程。
本文基于这个交互过程来实现自己的通信协议。
\pic[htbp]{协议交互流程图}{}{mysqljh}
\subsection{SQL引擎设计}
mysql的SQL语句非常的复杂,
自己解析很浪费时间,本文基于成熟的开源框架Druid
来实现自己的解析模块,解析流程如图\ref{sqljx}所示。
\pic[htbp]{SQL解析流程图}{}{sqljx}
\subsection{存储引擎设计}
存储引擎是存储系统的发动机,直接决定了存储系统能够提供的性能和功能。存储
系统的基本功能包括:增、删、读、改,其中,读取操作又分为随机读取和顺序扫描。
每种存储引擎底层都基于一种数据结构。比如常用的哈希表结构和B+树结构。
本系统采用了一种NoSQL数据库引擎,这样可以结合关系数据库和非关系型数据库的优点。
\section{集群架构详细设计}
要实现集群功能,肯定要用到网络功能,
还要考虑各种不可靠的因素。
为了系统的稳定性和可靠性,
本文用到了一个开源的分布式的开发框架hazelcast,
基于hazelcast的应用程序的结构图如图\ref{hazelcast}所示。
\pic[htbp]{集群架构图}{}{hazelcast}
分布式数据库节点和分布式管理节点都嵌入了Hazelcast模块,
这样分布式管理节点就能很容易的得到每个数据库节点的详细信息,
直接调用框架的接口就可以得到这些信息。
\section{审计模块详细设计}
图\ref{jiankong}显示了审计模块的实现结构图。从上大下一共可以分为下面4个模块:
\begin{enumerate}
	\item grafana可视化和报警模块,主要用到了开源的图形框架,来显示底层的审计数据,
	同时也可以让用户设定预警数据,
	这样以后就可以在一定的情况下向用户报警。
	\item elasticsearch模块用来存储监控数据,监控数据为了登陆日志和sql执行日志
	我们需要更改它的源代码才能用来作为审计数据库,
	因为审计数据库不能随意的更改和删除,只能查询和增加。
	\item JSQL接口层,主要是为上面和下面的各层提供连接功能,
	为上面层提供数据接口,为下面层提供显示接口,使得更加容易使用。
	\item 本地日志层,主要是在文件系统中存储所有需要的日志记录,这样可以随时和审计数据库的数据来
	对比。审计数据的安全和一致性。
\end{enumerate}
\pic[htbp]{监控模块}{}{jiankong}
\section{本章小结}
本章从总体上设计了系统的部署模式和模块图,对每个模块进行了详细了的
模块设计,便于后面的具体实现。