% !Mode:: "TeX:UTF-8"

\chapter{系统设计}
\section{总体架构设计}
jsql是一个可以替代mysql的分布式java数据库。
它的部署场景如题\ref{jgbst}所示。
\pic[htbp]{系统部署图}{}{jgbst}
每个数据库都可以接受客服端的命令请求,
底层数据会负责数据一致性的实现。

图\ref{jgztmkt}显示了系统的模块图。
一共分为下面几个模块:
\begin{enumerate}
	\item 数据库模块,实现单机版数据库
	\begin{enumerate}
		\item 网络层,实现前端网络层,接受客服端的连接请求
		\item 协议层,实现mysql的通信协议
		\item SQL层,解析和优化sql的执行
		\item 存储引擎层,基于哈希树的存储引擎
		\item 硬件层,基于内存映射文件的存储系统
	\end{enumerate}
	\item 分布式模块,实现分布式功能
	\item 监控模块,监控,报警功能
\end{enumerate}
其中每一个模块的具体功能不一样,需要分别设计,下面是每个功能
模块的详细设计。
\pic[htbp]{系统总体架构图}{}{jgztmkt}
\section{数据库模块详细设计}
\subsection{网络模块设计}
所有的服务软件都需要实现网络模块,这样才能连接客服端的请求。
JSQL用java语言开发,主要用到的是java的网络开发模块。
但是直接用java语言开发网络功能非常的麻烦,
所以本文基于Netty来开发直接的网络模块。具体的网络实现,包括
线程池的规划都按照Netty的建议来开发。
\subsection{通信协议设计}
每个服务器和客服端通信都要实现自己的通信协议,考虑到mysql使用的广泛性
。jsql采用mysql的通信协议。这样就不用自己开发协议了。图\ref{mysqljh}显示了
mysql协议中客服端和服务器之间的交互过程。
本文基于这个交互过程来实现自己的通信协议。
\pic[htbp]{协议交互流程图}{}{mysqljh}
\subsection{SQL引擎设计}
mysql的SQL语句非常的复杂,
自己解析很浪费时间,本文基于成熟的开源框架Druid
来实现自己的解析模块,解析流程如图\ref{sqljx}所示。
\pic[htbp]{SQL解析流程图}{}{sqljx}
\subsection{存储引擎设计}
存储引擎是存储系统的发动机,直接决定了存储系统能够提供的性能和功能。存储
系统的基本功能包括:增、删、读、改,其中,读取操作又分为随机读取和顺序扫描。
每种存储引擎底层都基于一种数据结构。比如常用的哈希表结构和B+树结构。
本系统用的是一种叫HashTree的数据结构,它的结果如图\ref{hashtree}所示。
\section{集群架构详细设计}
要实现集群功能,肯定要用到网络功能,
还要考虑各种不可靠的因素。
为了系统的稳定性和可靠性,
本文用到了一个开源的分布式的开发框架hazelcast,
基于hazelcast的应用程序的结构图如图\ref{hazelcast}所示。
\pic[htbp]{集群架构图}{}{hazelcast}
\section{审计模块详细设计}
图\ref{jiankong}显示了审计模块的实现结构图。从上大下一共可以分为下面4个模块:
\begin{enumerate}
	\item grafana可视化和报警模块,主要用到了开源的图形框架,来显示底层的审计数据,
	同时也可以让用户设定预警数据,
	这样以后就可以在一定的情况下向用户报警。
	\item elasticsearch模块用来存储监控数据,监控数据为了登陆日志和sql执行日志
	我们需要更改它的源代码才能用来作为审计数据库,
	因为审计数据库不能随意的更改和删除,只能查询和增加。
	\item JSQL接口层,主要是为上面和下面的各层提供连接功能,
	为上面层提供数据接口,为下面层提供显示接口,使得更加容易使用。
	\item 本地日志层,主要是在文件系统中存储所有需要的日志记录,这样可以随时和审计数据库的数据来
	对比。审计数据的安全和一致性。
\end{enumerate}
\pic[htbp]{监控模块}{}{jiankong}
\section{本章小结}
本章从总体上设计了系统的部署模式和模块图,对每个模块进行了详细了的
模块设计,便于后面的具体实现。