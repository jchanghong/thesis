% !Mode:: "TeX:UTF-8"
\chapter{总结与展望}
\section{本文工作的总结}
现在社会的发展,离不开各种各样的数据库系统,关系数据库系统在很多关键场合无法被替代,非关系数据库系统各有各的优点,如今企业中常常出现的情况是,用不同的数据库系统来存储不同数据类型的数据,这显然浪费了资源,增加了人力维护成本。
另一方面,单机数据库已经不能满足现在数据存储和处理的要求。最后,在数据越来越重要的今天,从底层加入审计功能对一个数据库系统来说非常有必要,所以研究设计和实现一个分布式多模型安全数据库系统非常有意义。
因此,在导师的
指导下,在研究生阶段作者主要做数据库有关的开发工作。
最终,将理论知识和实践技能相结合,
开发出了这套分布式多模型安全数据库系统。

分布式系统的研发是一个不小的挑战。作为这个系统的开发者,我首
先在理论方面做了很多学习和研究,包括硬件软件有关的知识,和数据库事务
相关的理论
,以及各种分布式相关的协议。理论只能作为指导,而不能成为一个系统。
在实现方面,我主要深入学习了JAVA语言,对多线程和高性能网络开发技能也进行了
深入的学习。最终实现的这套数据库系统作者认为有如下优点:
\begin{enumerate}[fullwidth,itemindent=2em]
	\item 论文所述系统完全采用JAVA语言编写,具有跨平台和安全的特点。
	\item 结合了非关系型数据库和关系型数据库的优点。
	\item 在系统的架构设计上,采用了面向对象的思想,对各个模块进行了很好的
	划分,有利于对系统进行进一步的完善。
	\item 从数据库系统本身加入了审计功能。
	\item 采用分布式负载均衡算法,实现多主分布式架构,能够动态增加数据库节点。
\end{enumerate}

在设计和实现这套系统的过程中,我的工作主要包括以下几点:

\begin{enumerate}[fullwidth,itemindent=2em]
	\item 利用OrientDB等非关系型存储引擎设计和实现了分布式数据库的本地存储系统。
	\item 实现了Mysql通信协议,使得可以通过Mysql客服端来连接JSQL分布式数据库,
	方便Mysql用户迁移到本数据库系统。
	\item 实现了对SQL语句的解析和执行,完成对关系数据库接口的支持。
	\item 实现了数据库的分布式架构,使得数据可以存储在多台计算机上面,支持数据复制和负载均衡。
	\item 实现了系统的审计系统,使得系统的安全性得到增强。
	\item 设计和实现了分布式负载均衡算法,完成了分布式数据库集群节点的动态负载均衡功能。
	\item   在设计和开发完成系统以后,论文对系统进行了性能和功能测试,并将本系统和MYSQL集群进行了对比,验证了本系统在实践中的可行性。
\end{enumerate}
\section{未来的工作方向}
作者设计和开发的分布式数据库系统,从数据存储的基本功能来看,其已基本达到要求。但是这个初步开发的系统还不能和生产环境下那些非常成熟的系统相比较;
还有,作者一个人水平也有限,本系统难免还有很多需要完善的地方,总结起来有下面几
个方面:

\begin{enumerate}[fullwidth,itemindent=2em]
	\item 系统的数据恢复机制不够完善。虽然现在系统能够对数据进行存储,但是其不能很好的应对很多突发情况。在出现这些情况的时候,其数据有可能会丢失。在这样的问题下,论文在以后的工作中还要对其更好的设计,加入日志恢复功能。
	\item 分布式数据库中数据迁移目前还没有实现。现在有非常多的数据库系统,对于新的数据库系统来说,要想用户使用其系统,就必须要提供一个安全的迁移工具,使得其能将老的数据迁移到新的数据库系统上。论文后期工作可以加入对常用数据库系统的迁移工具。
	\item 关系型数据库大都有存储过程和触发器这些功能,本系统作为一个实验室产品,目前还没有完全的实现这些功能。虽然作为一个能存储结构化数据的关系型数据系统而言,这些都是应该有的功能。由于暂时时间有限,而且这些功能在一般情况下用的不多,所以系统目前就没有实现这些功能。在未来的工作中,我会把jsql当作一个关系型分布式数据库系统,继续对其关系型数据库功能进行完善。
	\item 缺少分区分片功能。对于现在的应用程序来说,很少有需要分区分片功能的场景。大多数应用程序只需要对读写性能进行扩展就可以满足要求。但是对于超大应用程序来说。还是要在分布式环境下实现分区分片。所以这个功能也是需要未来来完成的。未来需要完成本系统的分区分片功能,使得系统的存储容量可以随着机器数的增加而增加。
\end{enumerate}
\section{数据库系统的未来}
数据库系统出现之前是用文件来存储数据的,其缺点是不能在不同的应用程序中共享数据,不同的应用程序存储文件的格式不同,就很难与其他应用程序分享数据。
在数据库系统的发展过程中,出现了非常多的数据库系统,其中上世界发明的关系型数据库系统因为很多优点而一直沿用至今。在当时网络不是很快,移动设备不多的时代,关系型数据库系统能很好的满足企业的存储和处理要求。

但是随着移动互联网的发展,出现了越来越多的移动设备,数据的增长数据已经远远超过我们的想象。在数据量增加的同时,数据类型也在变多。以前我们主要关注结构化数据的存储。现在出现了非常多的数据类型,就比如推荐引擎中的图数据类型、日志分析系统中的非结构化数据,而关系数据库系统不能很好的存储和处理这类数据,所以就需要不同的数据库系统。

企业需要对不同的数据进行存储和分析处理,就需要不同的数据库系统,这对系统资源是严重的浪费。新的数据库系统需要维护人员,这也增加了企业的人力成本。如果有一个数据库系统能够存储所有数据类型,而且其部署在分布式环境下,能够线性扩展其存储能力和读写能力,就能解决当前企业数据存储的所有问题。

把所有系统都存储在一个系统中也带来了非常大的危险,如果这个系统被攻击了,那么企业的所有数据就会收到损害或者丢失。这就是为什么本论文所设计的分布式系统实现了安全监控功能。

我认为,未来的数据库系统应该会朝着jsql的理想发展。理想的数据库系统应该是一个分布式的数据库系统,其能部署在全球所有地区,能够动态增加数据库节点,自动部署,最理想的情况下还应该能线性扩展其存储能力,能动态扩展其读写能力;理想的数据库系统也系统是一个多模型的数据库系统,其能存储所有数据类型的数据,包括结构化的订单数据,图片和视频等二进制数据,还能对其进行分析处理,产生价值,数据本身没有价值,只有对其进行分析和挖掘,才能找出价值;最后,理想的数据库系统应该是一个安全的数据库系统,其能自动备份,能够抵御自然灾害,能够抵御人为攻击。我们希望出现一个这样的理想的数据库系统,这需要所有工作者的艰苦努力。