% !Mode:: "TeX:UTF-8"
\chapter{总结与展望}
现在社会的发展,离不开各种各样的数据库系统。
关系型数据库在很多关键场合有不可替代的优势,
对关系型数据库有关的理论研究和实践非常有意义。
因此,在导师的
指导下,在研究生阶段作者主要做数据库有关的开发工作。
最终,将理论知识和实践技能相结合,
开发出了这套分布式数据库系统。

分布式系统的研发是一个不小的挑战。作为这个系统的开发者,我首
先在理论方面做了很多学习和研究,包括硬件硬件有关的知识,和数据库事务
相关的理论
,以及各种分布式相关的协议。理论只能作为指导,而不能成为一个系统。
在实现方面,我主要深入学习了JAVA语言,对多线程和高性能网络开发技能也进行了
深入的学习。最终实现的这套数据库系统作者认为有如下优先:
\begin{enumerate}
	\item 从数据库引擎到SQL模块完全采用JAVA语言编写,具有跨平台和安全的特点。
	\item 利用哈希树实现了高性能的存储引擎,特别适合现在的应用程序。
	\item 在系统的架构设计上,采用了面向对象的思想,对各个模块进行了很好的。
	划分,有利于对系统进行进一步的完善。
	\item 从数据库系统本身加入了审计功能。
\end{enumerate}

当然,由于作者水平有限,本系统难免还有很多需要完善的地方,总结起来有下面几
个方面:
\begin{enumerate}
 \item 系统的数据恢复机制不够完善,后续需要对存储引擎进行进一步的开发,加入日志等功能。
\item 分布式数据库中数据迁移目前还没有实现,这是值得进一步研究的课题。
\item 关系型数据库大都有存储过程和触发器这些功能,本系统作为一个实验室产品,目前还没有完全的实现这些功能。
\end{enumerate}

作者认为,虽然现在出现了很多的Nosql数据库系统,但是关系型数据库的作用是无法替代的,电子商务和各种银行业务都需要关系型数据库的支持,所以对关系型数据库的学习和实践特别有意义。另外,数据库的安全问题越发严重,时常发生管理员篡改数据的情况发生,
所以作者觉得,每个数据库系统,都要从底层加入安全审计功能,这样我们的数据才能保证最基本的可靠性。关于本系统,需要做的事情还有很多,但是我相信随着进一步的开发,本系统的功能会进一步完善。