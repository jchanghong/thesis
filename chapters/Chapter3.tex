% !Mode:: "TeX:UTF-8"

\chapter{系统分析}
分析,设计,开发和测试是软件开发工程的不同阶段。分析包括需求分析和实现技术分析和选择,软件系统采用的技术是系统设计和实现的重要基础。因为不同的技术需要不同的设计,不同的实现过程。
系统需求分析
是软件开发过程的开始阶段,是软件生命周期中
的一个重要环节,对于整个软件开发过程以及软件产品的质量是至关重要的,因
为构建软件系统最艰难的一个部分就是准确的决定要构建什么,包括所有对人的界面、对机器的接口、以
及对其他软件系统的接口的需求都需要详细的记录下来。如果做错了会对最终系统造成相当大的损害,并且后
期难以调整。而后面的设计只是是将问题转换为解决方案的的创造性过程,
系统设计是在系
统分析的基础上研究系统如何实现需求中所描述的功能,给出满足需求的解决方
案,可见,没有经过仔细的系统分析就不能做好后面的系统设计和编码工作。
本章对分布式数据库JSQL进行需求分析,需求分析
主要包括整个系统的需求分析与功能分析,同时也包括了对各个功能模块的技术分析。
\section{系统需求分析}
需求阶段的目标是理解客户的问题和需要,需求分析是在综合分析用户对系
统提出的需求的基础上,构造一个能表达用户需求的需求说明,并以“软件需求规格说明书”的形式记录下本阶段工作的结果,为下一阶段的软件设计提供设计基础。
\subsection{系统功能需求}
本论文所研究的分布式数据库系统,主要对大数据量进行高效处理,需
要系统具备大容量存储和高速率运算。传统的关系数据库在移动互联网大数据量存储的要求下,出现了很多的问题,而现在的各种NOSQL数据虽然支持
大量数据的存储,但是对关系的操作又很少。因此,本文结合关系型技术与分
布式NOSQL技术,来设计能够对大量数据进行高效处理的分布式数据
库系统。作为项目的前期工作,本文所设计的分布式数据库系统的主要功能
如下:


	\begin{enumerate}
		\item 提供负载均衡功能
	\end{enumerate}

	本文所述的分布式系统,为了实现分布式系统中每个数据库节点的资源的合理分配,
	需要系统具有良好的负载均衡功能,即前端客户端发送的数据操作请求消息能够均衡的发送到各
	分布式数据库节点。因此本文提出了分布式管理几点,该几点的功能构成本文
	的分布式管理模块的主要功能,其为客服端选择合适的后端分布式数据库几点,从而达到负载均衡的效果。
	分布式管理节点只是一个功能节点,其上实现分布式负载均衡算法,为客服端选择合适的后端服务器,同时其也能提供后端数据库存储功能。
	
	\begin{enumerate}[resume]
		\item 提供数据库功能
	\end{enumerate}

	站在客户的角度,需要本系统能够提供增加、删除、修改、查询等数据库基
	本功能操作。本文结合非关系型数据库和关系数据库技术。实现了一种分布式数据库系统。
	对用户来说,本系统提供的功能接口和关系型数据库接口是一样的,也就是就是SQL接口,
	更准确的说,是Mysql接口,所以系统应该提供关系型数据库常用的功能。
	同时本系统利用非关系型数据库OrientDB存储,实现数据的高效存储和访问,利用数据库集群实现数据库
	存储功能的可靠性。
	
	\begin{enumerate}[resume]
		\item 提供管理监控功能
	\end{enumerate}

	JSQL系统在分布式管理节点和分布式数据库节点的结合下,从数据库的底层实现了数据库的安全审计的功能。
	在分布式管理节点,实现了审计监控的界面功能,管理员通过管理计算机连接分布式管理节点就可以对数据库进行管理和维护
	。同时也能进行监控报警功能的设置。


	因此,基于对整个分布式数据库系统的功能需求的分析,系统可分为两
	部分,提供负载均衡功能和管理监控功能的分布式管理节点,提供增加、删除、
	修改、查询等数据库基本功能操作的数据库节点。
具体的,提供的数据库功能主要包括:
\begin{enumerate}[fullwidth,itemindent=2em,listparindent=2em]
	\item 数据定义:分布式数据库系统能提供数据定义功能,数据定义能让用户定义数据模型结构。数据定义保存在数据库字典中。因为本文阐述的是一个多模型数据库,其数据定义相当灵活,可以定义字段,也可以以后增加和删除字段。
	\item 数据操作:分布式数据库系统提供数据操作功能
	,供用户实现对数据的追加、删除、更新、查询等操作。
	\item 数据库的运行管理:分布式数据库的运行管理功能是系统的运行控制和管理功能
	,包括多用户环境下的并发控制和安全管理,这些功能是系统正确运行的基础。
	\item 数据组织:分布式数据库系统要分类组织数据,
	包括数据字典等,需确定以何种文件结
	构和存取方式在存储级上组织这些数据。
	数据组织和存储的基本目标是提高存储空间的利用率,选择合适的存取方法提高存取效率。
	\item   数据库的保护:数据库中的数据是信息社会的战略资源,所以数据的保护至关重要。系统
	对数据库的保护通过4个方面来实现:数据库的恢复、数据库的并发控制、数据库的完整性控制
	、数据库安全性控制。还有对外界对数据库系统的操作情况进行记录,以便管理员对其分析。
	\item  数据库的维护:这一部分包括数据库的重组合重构以及性能监控等功能,这些功能分别由各个使用程序来完成。
\end{enumerate}
\subsection{系统功能用例}
用例图用于描述一组用例、参与者及它们之间的连接关系。一个用例描述了
一组动作序列,每一个序列表示系统的外部设施(系统的参与者)与系统本身的
交互。本系统的用例图是从参与者使用系统的角度来描述系统中的信息,即站在
系统外部查看系统应该具有何种的功能,而不是描述功能在系统内是如何实现
的。本文设计的分布式数据库系统的主要功能为新增数据、查询数据、修改
数据、删除数据。系统总体的用例图如图\ref{pic3/xuqiufenxi}所示。

\pic[htbp]{系统用例图}{}{pic3/xuqiufenxi}
分布式数据库的主要功能用例为新增数据、查询数据、修改数据、删除
数据等,下面对这几种操作的用例进行说明。
\begin{enumerate}[fullwidth,itemindent=2em,listparindent=2em]
	\item 新增数据是用户从客户端向系统发起新增数据请求,
	具体用例见表\ref{pic3/xinzeng}。
	\pictable[htbp]{新增数据用例}{}{pic3/xinzeng}
	\item 数据查询是指利用用户在客户端上输入指定数据标识作为关键字,查询
	该条数据的基本信息,具体用例见表\ref{pic3/chaxun}。
		\pictable[htbp]{查询数据用例}{}{pic3/chaxun}
\item 	删除数据是指当数据无效时,客户端发送删除信息请求分布式数据
	库系统删除该条数据,具体用例见表\ref{pic3/shanchu}。
		\pictable[htbp]{删除数据用例}{}{pic3/shanchu}
\item 	修改数据是指当数据需要修正时,管理员通过客户端发送修改消息请求,
	通知分布式数据库系统存入新数据
	,刷新数据变化部分,具体用例见表\ref{pic3/xiugai}。
	\pictable[htbp]{修改数据用例}{}{pic3/xiugai}
\end{enumerate}
\section{技术和框架分析}
\subsection{系统实现语言选择}
本数据库系统选择了JAVA语言和KOTLIN语言作为开发语言,
之所以这样选择,主要基于以下几个原因:
\begin{enumerate}[fullwidth,itemindent=2em,listparindent=2em]
	\item JAVA是跨平台的开发语言,用JAVA开发实现数据库系统,
	可用实现所有主流平台的数据库部署。
	\item JAVA是企业开发的首选语言,其安全性比本地语言更高。
	\item 非关系型数据库引擎OrienDB也是用JAVA开发的,为了方便调用。本系统也应该使用同一种语言。
	\item JAVA易于开发和调试,作为学生时期个人开发的数据库系统,
	一个人的精力是有限的,如何选择高效的开发工具和语言对我们来说非常重要。
\end{enumerate}
\subsection{网络实现技术分析}
所有的服务软件都需要实现网络模块,这样才能连接客服端的请求。
JSQL用java语言开发,主要用到的是java的网络开发模块。

网络IO的方式分为同步阻塞、同步非阻塞和异步非阻塞方式\citeup{thesis20}。

在JDK1.4出来之前,我们建立网络连接的时候采用同步阻塞的模式,
首先需要先在服务端启动一个ServerSocket,
然后在客户端启动Socket来连接服务端进行通信,
默认情况下服务端需要对每个请求建立一个单独线程等待客户端请求,
而客户端发送请求后,先咨询服务端是否能接受请求
,如果不能的话则会一直等待或者遭到连接拒绝请求,如果可以的话,客户端才会连接服务器。

NIO本身是基于事件驱动的思想来完成的
,其主要想解决同步阻塞的问题:
 在使用同步阻塞的网络应用中,如果要同时处理多个客
 户端请求
 ,就必须使用多线程来处理。也就是说,将每一个客户端
 请求分配给一个单独线程来单独处理。这样做虽然可以达到我们
 的要求,但这同时又会带来另外一个问题。由于每创建一个线程
 ,就要为这个线程分配一定的内存空间,而
 且操作系统本身也对线程的总数有一定的限制。如果客户端的连接请求过
 多,服务端程序可能会因为不堪重负而
 拒绝客户端的请求,甚至服务器可能会因此而崩溃。



Netty的主要目的是建立基于NIO的高性能协议服务器,分离和松散耦合网络和业务逻辑组件。
它可能会实现一个广为人知的协议,如HTTP或您自己的特定协议。
Netty的线程模型如如\ref{netty}所示。
\pic[htbp]{Netty}{}{netty}
Netty是一个非阻塞框架。与阻塞IO相比,这导致高吞吐量。
Netty使用事件驱动的应用程序范例,因此数据处理的流水线是一系列事件处理程序。

因为Netty的这些特效,利用它可以实现一个高性能的网络应用程序,所以本系统的网络模块也是用的Netty来
实现的,它支持高并发的客服端访问。
\subsection{通信协议分析}
每个服务器和客服端通信都要实现自己的通信协议,考虑到Mysql使用的广泛性
。Jsql采用Mysql的通信协议。这样就能支持更多的遗留客户端软件。
在MySQL数据库通信过程中。当一个客户端的连接请求到达,就会执行握手和权限验
证阶段。如果验证成功,会话开始。然后,客户端发送消息,服务器会以一个适合该发送命
令的数据类型的数据包或者一条消息进行回复。当客户端发送完成后,会发送一个
特殊的命令包,告诉服务器己发送,然后会话结束。mysql通信的基本单位是应用程序包。
多个指令可以合成一个包;答复可以包含多个包。
MySQL客户端与服务器的交互过程主要有两个阶段:握手认证阶段和命令执
行阶段:
\begin{enumerate}
	\item 握手认证阶段\\
	握手认证阶段在客户端与服务器建立连接后立即进行,交互过程如下:
	\begin{enumerate}
		\item 服务器发送给客户端握手初始化报文。
		\item 客服端回复服务器端登陆认证报文。
		\item 服务器发送给客户端认证结果报文。
	\end{enumerate}
	\item 命令执行阶段\\
	客户端和服务器认证成功后,就会进入命令执行阶段,交互过程如下:
	\begin{enumerate}
		\item 	客户端发送服务器执行命令报文。
		\item 服务器发送给客服端命令执行结果。
	\end{enumerate}
\end{enumerate}
MySQL客户端和服务器之间完整的交互过程如图\ref{mysqljh}所示。
\pic[htbp]{协议交互流程图}{}{mysqljh}

在实现mysql协议的同时,也需要对二进制协议进行接口定义,以便用户可以存储其他数据类型的数据。系统选择实现mysql协议,是因为mysql是最广泛使用的免费的数据库系统,其用户非常之多,实现mysql协议以后,mysql的所有工具和客户端都可以直接连接JSQL分布式数据库系统。
\subsection{SQL实现分析}
SQL是一种通用的数据库操纵语言,数据库和应用程序用户将数据库操作请求发送到数据库服务器,只需要说明操作目的和结果,不需要说明具体操作的执行过程。同样的语句,不同的数据库服务器会选择合适自己存储引擎的执行过程。这样每个数据库服务器就能实现不同的操作过程,实现不同的性能。

在关系数据库系统中,SQL是通用的操作语言,用户和数据库系统通过SQL语句交流数据和发送命令请求。用户发送SQL命令,然后数据库系统解析命令,返回命令结构。
虽然所有的关系数据库系统都支持SQL语句,但是不同的数据库系统支持的语句类型不一定相同。本论文所述的分布式系统选择实现和mysql一样的sql语句。

Mysql支持的sql语句和标准的sql语句不全一样,具体的,它支持下面这几种语句类型:
\begin{enumerate}
\item 数据定义语句     
\item 数据操作语句     
\item 交易和锁定声明     
\item 复制语句     
\item 准备的SQL语句语法     
\item 复合语句语法     
\item 数据库管理语句     
\item 效用声明  
\end{enumerate}   
其中每种语句类型又分为很多种sql语句,所以mysql支持的sql语句非常的多,
因为本系统选择了兼容mysql的协议,所以我们
也要解析这些不同的sql语句,本系统利用Druid框架实现了对SQL语句的解析功能。
其对客户端的命令进行词法分析后语法分析以后,生成预定的JAVA对象格式,这样我们
就能对其进一步的处理。当然,并不是所有的语句都能这样检查的处理,还有其他Druid不支持的
语句,我们只能自己对其进行解析。在对sql语句进行语法解析以后,接下来就需要进行语义分析。

结合分布式数据库系统元数据,对sql语句进行合法性检查。查找适合的数据库存储接口调用。返回客户端结果。
\subsection{存储引擎分析}
存储引擎是分布式存储系统的发动机,直接影响分布式数据库系统的性能和功能。
存储
系统的基本功能包括增加删除数据等,其中,在数据库引擎中,数据读取操作又分为随机读取
和顺序扫描。
每种存储引擎底层都基于一种数据结构。比如常用的哈希表结构和B+树结构。
本系统所用的OrientDB存储引擎底层就是用的B+树结构。
利用非关系数据库OrientDB实现可靠的数据存储,结合非关系型数据库的优点,
使得数据库系统更加容易实现集群,更加容易扩展。

和传统关系型数据库系统不同的是,在orientdb存储引擎中,关系是直接存储在每个记录的对象里面的。而在关系数据库系统中,数据之间的关系是通过外键关联的。这样在获取一个记录相关联的时候,就必须去查找另外的一个表记录。这样就会对磁盘进来查找。最严重的就是顺序扫描,其性能非常低下,这也就是为什么关系型数据库在外键很多的时候,性能严重下降的原因。而在orientdb数据库引擎中,要查找一个记录对应的记录,只需要按照记录中记录的地址去查找就可以了。关系直接存储在记录中,不需要再去查找外键。对提高性能非常重要。
\subsection{分布式实现分析}
虽然现在很多NOSQL数据库都实现了分布式,但是本系统还是利用Hazelcast
自己开发分布式功能。

Hazelcast是一个开源的内存数据网格,提供了一些引人注目的功能,包括:
\begin{enumerate}
	\item 集群节点发现和选举
	\item 分布式数据结构
\item 	分布式计算
	\item 分布式查询
	\item 聚类
	\item 高速缓存
	\item 多语言绑定
\item 	轻松嵌入到Java应用程序中
\end{enumerate}

该功能使Hazelcast成为应用程序中的多用途工具。
它可以用于简单的消息传递,缓存,键/值存储,
任务协调服务。与之前描述的其他一些服务不同,Hazelcast除了分布式配置和协调之外,
还可以用来解决应用程序中的多个问题。Hazelcast从一开始就被设计为一个分布式内存网格,
解决了主选举,网络弹性以及最终一致性等许多潜在的难题
。Hazelcast任务协调解决方案的一般概念是将信号/锁定义存储在内存中,并根据节点名称,
区域名称和可用许可证数量等标准,协调信标许可证的分配。
为了处理从群集中退出的节点,信号量允许基于支持时间的到期
,除了明确的释放。所有的集群,一致性和可用性要求将被授予Hazelcast内部,
使得Hazelcast解决方案尽可能简单。利用Hazelcast可以轻松开发出
稳定可靠的分布式集群功能。
\subsection{监控功能实现分析}
监控功能主要在分布式管理节点部署,管理员通过连接分布式管理节点,就能对分布式数据库进行监控。
对数据进行操作。
本系统实现最简单的监控模块,一个监控模块首先要存储
所有的日志数据,而且这个数据不能随意的更改,所以我改了Elasticsearch的源代码,
让它来存储所有的sql更新日志信息,其中的日志信息不能被篡改和删除,同时对其进行分析和统计,
然后用可视化的框架来显示出结果。
之所以用这个框架,主要是因为它是一种搜索引擎,能够非常快速的检束出我们需要的各种信息,
这对于信息审计来说非常的重要。利用这个框架我们实现了实时的安全审计功能,提高了数据库的安全性。
\section{本章小结}
本章对分布式数据库系统进行需求分析。首先,站在用户的角度
对系统的需求进行具体的分析,描述系统的功能需求,并结合系统用例图给出功
能用例。然后本章分析了系统每个功能模块所需要的技术,
每个模块的实现都要用到不同的技术,
对每种技术进行了分析。