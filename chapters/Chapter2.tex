% !Mode:: "TeX:UTF-8"

\chapter{相关理论和技术}
本章介绍数据库和分布式数据库有关的理论知识,以及本系统所需要用到的关键算法和技术。
\section{数据库系统}
\subsection{数据库系统概念}
数据库管理系统是一种管理软件,数据库管理员通过管理数据库管理对数据库进行具体的操作和维护操作。在数据库管理系统的统一管理和控制访问下,才能保证数据库中数据的一致性和完整性。数据库的用户只有通过数据库管理系统才能访问其中的数据,通过数据库管理系统,数据库管理员还能对其进行权限管理等安全维护操作。通过数据库管理系统,不同的应用程序和用户,可以安全的同时对其中存储的数据进行安全的访问,而且能保证数据的安全性和一致性。几乎所有数据库管理系统都提供
对数据的追加、删除等基本操作。
数据库管理系统是管理和操作数据库的核心软件,位于操作系统之上。
数据库管理系统提供了一个抽象,使得上层应用程序和用户能逻辑的操作存储在计算机上的物理数据
。有了数据库管理系统,数据库用户和管理员就只需要对其进行操作,对其发送命令就能处理具体的数据,而不能担心数据在计算机中具体的存储格式,也不能担心数据库在计算机系统的安全问题。
\subsection{关系数据库系统}
\subsubsection{关系模型}
关系模型是1969年被提出的用于数据库存储的数据模型,关系模型基于严格的数学理论,有严格的理论基础作为指导。
在关系模型中,所有的数据都被表示成数学上的关系,关系模型采用表的结构化存储来表示和存储数据和数据之间的关系。可以通过关系模型的规范化对关系模型中的数据进行简化,对建立一致性的数据存储有重要的作用。

在关系模型中,每种数据属于一种数据类型,也可以说是一种域。元组是属性的的结合,具体的理解中,可以理解为一条记录。而属性属于元组的一个元素,
是域和值的有序对。域和名字的有序对的集合就是关系变量,
关系变量是关系的表头。关系是元组的集合,也就是多条记录组成关系。尽管这些概念被严格的在数学上定义,但是在关系数据库中,可以用更加容易理解的可视化的方式来理解。在这种可视化的理解下,表就是关系,关系就是表。而关系理论中的元组类似于关系数据库中行的概念。
信息原理是关系模型的基本原理,在关系模型中,所有信息都表示为关系中的数据值。
关系模型本身是没有关系的,
然而,设计者通过在多个关系变量中使用相同的域名来模拟关系。
也就是说,
如果一个属性依赖于另一个属性,则在关系模型中通过参照完整性来强制这种依赖性。

关系数据库系统是建立在关系数据模型之上的软件系统,其通过数据中的代数等知识对数据进行操作,在关系数据库中,关系表现为一个表,该形式的表格作用的实质是存储具体的关系实体。这些表格通过属性能产生不同的关系类型,通过属性之间的关系,来模拟每个关系的变化。在关系数据库系统中,每个表格代表不同的数据类型,存储不同的数据。 每行数据包含一个唯一的数据实体,代表一个表的实例
,每个数据行则有不同的属性,这些属性通过列来表示。当定义一个关系表的时候,你能定义关系表上有哪些数据元素以及每种数据元素的类型。

在关系数据库中,通过关系模型对其数据进行严格的定义和操作,用户通过关系型数据库的操作需要,而不需要了解关系数据的具体存储格式对其进行操作。关系型数据库不但能根据关系模型存储结构化的关系数据,而且能保证存储数据的一致性,能支持数据的事务操作。
\subsection{事务与并发控制}
\subsubsection{事务}
事务能保证关系型数据库中数据的强一致性和正确性。一个事务表示对数据库系统的一系列的操作,关系型数据库管理系统能保证这些操作的正确执行,而不会破坏数据的完整性。
关系型数据库的事务有两个主要目的:
为了提供一个可靠的工作单元,在这个单元中,允许从故障中正确恢复,并保持数据库在任何情况下都是强一致性的。
即使在系统故障的情况下,关系型数据库也能保证事务的正确完成,避免数据的丢失和数据的不一致性。
事务也能在同时访问数据库的程序之间提供必要的隔离,如果没有提供这种隔离性
,程序的执行结果可能是错误的和不可预见的。
根据关系数据库的定义,数据库事务必须是原子的,一致的,隔离的和持久的。
在一个事务执行单元中,所有步骤必须正确完整的执行完成。
此外,关系数据库系统必须将每个事务与其他事务隔离开来
,不能产生不一致性的结果,同时,事务的执行结果必须存储在永久性存储介质中,不能丢失。

数据库完整性是事务的一种重要保证,如果不能保证数据的一致性,那么关系型数据库将是不可靠的。
每个事务都是不同的执行步骤组成,在事务执行过程中,每个步骤要不全部完成,要不全部失败,没有中间结果。
事务的执行必须是可预测的,关系型数据库不准出现不可预测的不一致性的结果,这样数据的完整性不能得到保障,就不能正确的使用数据。
\subsubsection{并发控制}
在计算机和数据库领域中,特别是在计算机编程,操作系统
,关系数据库领域,并发控制确保并发操作有正确的执行结果
。
在关系型数据库系统中,管理软件由不同的组件组成。每个组
件被设计为正确地操作,即符合某些一致性规则。在数据库操作中,一个组件可能影响另一个组件。一个事务可能影响另一个事务,从而出现不一致性甚至错误的结果。并发控制技术能保证数据的一致性,
但是,并发控制引入了需要大量额外的复杂性,并发算法的执行需要开销。
并发控制失败可能导致数据损坏,从而导致关系型数据库存储数据的失败。

在关系型数据库中,并发控制是正确性的必须要素。并发控制是任何系统中的正
确性的基本要素,在关系型数据库中,不同的事务执行过程中,可以访问相
同的数据,从而产生不正确的结果。在关系型数据库系统中,出现了很多并发控制技术来预防这样的不正确性。
下面是并发控制的主要方法:

	\begin{enumerate}
		\item 锁定
	\end{enumerate}

	通过锁定一个事务操作的对象,防止其他事务的访问。就能保证不同事务之间的正确执行而不能相互干扰,从而防止出现不正确的并发错误。
	
	\begin{enumerate}[resume]
		\item 串行化
	\end{enumerate}

 检查计划图中的周期并通过中止来破坏它们。
 \begin{enumerate}[resume]
 	\item 时间戳排序
 \end{enumerate}

  为每个事务分配不同的时间,让其按照不同的时间执行,这样可以防止同时访问数据库的时候产生不同的结果,影响数据的正确性。
   
    \begin{enumerate}[resume]
   	\item 承诺排序
   \end{enumerate}

 为不同的事务分配不同的顺序,保证在同一个事务中不可能访问到其他事务的执行结果,防止每个事务的相互影响。
 
\begin{enumerate}[resume]
	\item 多分支并发控制
\end{enumerate}

通过在每次写入对象时生成新版本的数据库对象,
	并根据调度方式允许事务对每个对象的最
	后相关版本的读取操作来增加并发性和性能。
	
	\begin{enumerate}[resume]
		\item 索引并发控制
	\end{enumerate}

将访问操作同步到索引,
而不是用户数据。专业方法提供了显着的性能提升。

每个事务都为其访问的数据维护一个私有工作空间,
只有在事务提交之后,其更改的数据才会在事务外部变得可见。
这种模式在许多情况下提供了不同的并发控制行为,并带来好处。
\subsection{非关系型数据库系统}
随着移动互联网网站的快速增长,传统的关系
数据库在应付移动互联网网站,特别是超大规模和高
并发的动态网站就显得力不从心
,出现了很多难以克服的问题,所以就出现了很多非关系型数据系统。
非关系型数据库系统由于其非常适合很多大型移动互联网网站数据存储,所以近几年得到了高速的发展,在很多互联网公司中都有不同的应用。非关系型数据库的产生
就主要是为了解决大规模数据的存储问题,特别是非结构化数据的存储问题。大数据量的存储需求给非关系型数据库带了新的发展机遇。
虽然非关系型数据库的流行还没有多久的时间,但是
不可否认,现在已经对关系型数据库产生了非常大的影响。虽然早起非关系型数据库都比较简单,不过现在很多非关系型数据库已经变得更加成熟,越来越适合现在数据的存储。不过,现在很多非关系型数据库不得不重写,已应对新的存储需求,从一开始,其就缺少很多关系型数据库的特性,虽然其抛弃了很多关系型数据库的特性,使得其在一些方法,比关系型数据库更加适合。但是
关系型数据库系统的很多功能是很多移动互联网应用所必须要满足的,而要满足这些需求,之前的很多非关系型数据库就必须要经过重写。

非关系型数据库并没有准确的定义,
但是他们都普遍存在下面一些共同特征:
\begin{enumerate}[fullwidth,itemindent=2em]
	\item   不需要预定义模式:在非关系型数据库系统中,不需要事先定义数据模式,也不需要预定义表结构。数据库系统中
	的每条记录都可能会有不同的属性和格式。当插入和修改数据时,并不需要定义它们的存储模式。
\item 	无共享架构:非关系型数据库系统在分布式部署中,其没有共享的架构,存储在分布式集群中不同的存储节点。客户端从最近的网络节点存取数据,比通过网络获取数据能提高系统的性能。
\item 	弹性可扩展:可以在系统运行的时候,动态增加或者删除结点。
不需要停机维护,数据可以自动迁移。
\item 	分区:非关系型数据库并不是将数据存放于同一个节点,非关系型数据库
需要将数据进行分片分区,将记录分散在季芹中的多个节点上面。并且通常分区的同时还要做数据复制
。这样既提高了数据库系统中的并行性能,又能保证集群系统中没有单点失效的问题。
\item 	异步复制:和很多关系型数据库系统不同的是,非关系型数据库中的复制,
往往就是基于日志的异步复制。因为这样,数据可以尽快地写入一个节点,
不会被网络传输引起迟延。这样做的缺点是并不总是能保证一致性,而非关系型数据库系统本身就没有保证强一致性,
这样的方式在出现季芹故障的时候,就可能会丢失少量的数据,在一些场景下,这并不是很严重的问题。
\item 	BASE:和关系型数据库的强一致性不同,非关系型数据库系统采用的是弱一致性,也就是说,并不能保证数据一定可靠,它只提供必须的一致性,这样就能提高更好的性能。
\end{enumerate}
\subsubsection{OrientDB介绍}
OrientDB是一种非关系型数据库系统,虽然它不是关系数据库系统,但是
其底层的存储引擎支持关系型数据库中的事务。所以本系统在后面就直接用的这个这个存储引擎
减少系统的开发时间,同时也增加系统的稳定性。同时能够结合非关系型存储引擎和关系型数据库功能,能为系统提供更好的实用性。
\section{分布式数据库}
\subsection{分布式数据库概述}
分布式数据库是一个数据库
,其中存储设备没有全部连接到一个共同的处理器。
它可以存储在位于相同物理位置的多台计算机中; 
或者可以分散在互连计算机的网络上。
与其中处理器紧密耦合并构成单个数据
库系统的并行系统不同,分布式数据库系统由分布式在不同的分布式集群计算机组成,没有共享共同的物理组件。
\subsection{分布式数据库的特点}

	\begin{enumerate}
		\item 数据独立性
	\end{enumerate}

	在分布式数据库系统中,其中的每个分布式系统节点都具有数据独立性。分布式数据库独立性包括
	数据的逻辑独立性以及物理独立性,逻辑独立性是指数据库用户的应用程序与分布式数据库的逻
	辑结构是彼此相互独立的,也就是说,当数据的逻辑结构发送改变时,数据库用户程序可以保持不变;分布式数据库的物理
	独立性是指数据库用户的应用程序和存储在磁盘上的数据库中数据是相互独立的。也就是说,
	数据在磁盘上怎样存储由分布式数据库系统来管理,用户程序就不需要了解怎么样存储,这样当数据的物
	理存储改变的时候,应用程序就不用改变,保证了应用程序的迁移性和可用性。
	

	\begin{enumerate}[resume]
		\item 分布透明性
	\end{enumerate}

	分布式数据库的分布透明性是指,分布式数据库的用户不需要关心数据是怎么分布的,不需要关心数据存储在什么地方,逻辑上,不需要知道数据库的逻辑数据模型,在物理上,不需要知道分布式系统的物理架构。有了分布式数据库的分布式透明性,当数据库的物理存储位置发生改变是,其数据库用户应用程序不需要知道变化,能进行保持运行。这样就能保证在分布式数据库系统在系统升级和数据迁移时候,上面的应用程序不需要做任何的改变就能适合新的物理价格。这样我们就能从容的改变分布式的物理分布式,而不用担心会影响到应用程序。

	\begin{enumerate}[resume]
		\item 节点自治性
	\end{enumerate}

	在分布式数据库系统中,每个分布式集群中的单个数据库功能和整个分布式数据库功能是一样的,用户不需要知道其连接的是一个单个数据库还是分布式集群。分布式数据库的管理可以用统一的方式对分布式数据库节点进行管理。每个节点都像单独的数据库一样,能够自己管理其本地的数据存储

	\begin{enumerate}[resume]
		\item 复制透明性
	\end{enumerate}

	分布式数据库系统的用户不用关心数据库在集群中各个节点的复制情况。分布式数据库中的数据复制,一般更
	新都由分布式数据库系统自动完成,在分布式数据库管理系统中,可以透明的把一个节点的数据复制到其
	他节点存放,,应用程序不需要关心这些变化,应用程序可以使用复制到本地的数据节点在本地完成分布式数据库有关的操作,避免
	通过网络传输数据,这样就能提高分布式数据库系统的整体效率和性能。对于分布式数据库数据的更新操作,其要在所有的分布式数据库节点传播,这同样对分布式数据库的用户透明。

	\begin{enumerate}[resume]
		\item 易于扩展性
	\end{enumerate}

	在分布式数据库系统中,可能会遇到系统性能和存储不能达到要求的情况,这样我们就遇到对分布式数据库系统进行扩展。如果服分布式数据库系统软
	件支持透明的扩展,那么就可以在不影响数据库用户的情况下,动态的增加多个服务器来进一步扩展分布式系统的性能,同时也能根据当前系统的情况。减少系统节点,减少系统的资源浪费。
\subsection{数据分布和负载均衡}
分布式系统如何拆解输入数据,将数据分发到不同的机器中。下面将介绍几种不同的数据分布方式。 
\subsubsection{哈希分布}
哈希分布方式可能是最分布式数据库系统中最常见的数据分布方式,其方法就是是按照数据的某一元素特征计算计算机哈希值,
然后将哈希值与分布式机器中的机器机器建立一种映射关系,从而将具有不同哈希值的数据分
布存储到不同的机器上。
只要哈希函数选择的好,分布式数据库库的数据就能均匀到分发到集群中不同机器节点中
。在哈希分布式中,分布式系统需要管理的元信息很少,只需要知道哈希函数和计算机集群中节点个数。 
但是这种方法有个很明显的缺点,那就是其扩展性很差。比如,如果想把集群规模扩大一倍,那么
所有集群中的的数据需要被重新迁移到不同的节点中。针对哈希分布方式扩展性差的问题,一种思路是
不再简单的将哈希值与分布式集群中机器数做除法取模映射,而是将哈希和集群集群的对应
关系作为元数据由专门的元数据服务器存储管理。访问数据时,首先计算哈希
值并查询元数据服务器,获得该哈希值对应的分布式机器节点。不过,这样一样就需要保存很多的元数据,在分布式集群集群数变得很大时,这样的方法就存储很大的性能问题。
\subsubsection{顺序分布}
在分布式系统中,按数据范围分布是另一个常见的数据分布方式,
将数据按关键字的范围划分成不同的区间,使得分布式集群中每台服务器节点处理不同区间的部分数据。
比如我们可以将用户划分到不同的区间,分区到分布式系统中不同的节点存储。
用数据范围来解决数据的分布式,实现动态划分范围空间,实现负载均衡。 
这样方式分布数据,需要保存什么区间的数据存储到什么数据库节点上面去,这样的信息可以存储在专门的元数据节点服务器中。 
在实际分布式系统中,我们一般也不按照某一维度划分所有数据范围,而是使用全部数据划分范围,
这样就能避免数据倾斜的问题。 
使用这样的分布方式,可以在不同的机器之间迁移数据,而不会影响到过多的分布式数据库节点,而只需要在元数据库服务器中,更新分布式区间和数据库节点之间的对应关系,这是哈希分布式方式所不能做到的。另外,当集群需要扩容时,也可以随意添加机器,不需要迁移数据,而且不限为倍增的方式,
只需将原机器上的部分数据分区迁移到新加入的机器上就可以完成集群扩容。
这样的分布方式问题就是需要存储所有的数据分区到数据库节点之间的对应关系,当数据库的容量增加时,元数据可能会成为分布式数据库系统的瓶颈。
\subsubsection{一致性哈希}
一致性哈希的分布式方式是为了克服简单哈希分布的缺点。
在简单哈希分布式中,当增加数据库节点的个数的时候,我们需要迁移非常多的数据。这样就性能系统的扩展时间,影响系统的性能。
在一致性哈希中,分布式数据库节点的哈希值保持在一个圆圈上面。然后根据关键字计算哈希值,在圆圈上面找最接近的哈希值。然后就把这个哈希值对应的数据存储在对应哈希点的分布式数据库节点上。
在增加或者删除数据库节点的时候,只需要迁移相邻分布式节点之间的数据,其他节点的数据可以保持不移动。这样就能提高分布式系统的性能。
使得更加容易扩展其容量和性能。 
\subsubsection{负载均衡}
在分布式数据库系统中,负载均衡是用来平衡各种网络和服务器资源的使用的。通过负载均衡,客户端可以选择合适的服务器来操作分布式数据的存储。避免一些分布式节点因为过载而不能提供服务,避免一些分布式数据节点因为得不到客户端的请求,而浪费分布式系统整体的资源使用率。
通过使用负载均衡,能使系统高效的完成客户端的请求,同时避免资源的浪费。
负载均衡功能可以通过硬件和软件的方式来实现,软件负载均衡是根据不同的负载均衡算法来实现的。
有很多种负载均衡的算法可用,下面是常用的一些算法:



	\begin{enumerate}
		\item 轮询法
	\end{enumerate}

	轮询法可能是是负载均衡算法中最常用最简单的算法,这种算法也很容易理解,同时也容易实现。 
	轮询法是指分布式系统中负载均衡服务器将客户端的请求按顺序
	轮流分配到分布式系统后端服务器节点上,以达到负载均衡后端数据库节点资源的目的。 	

\begin{enumerate}[resume]
	\item 加权轮询法
\end{enumerate}

	简单的轮询法没有考虑分布式系统中每个节点性能的差异,也没有考虑分布式系统节点的当前负载状态,所以不是高效的负载均衡算法。加权轮询法则是考虑了分布式数据库节点中每个节点性能的差异,当客户端连接到来的时候,在为其选择分布式节点的时候,考虑到分布式节点的性能和当前的负载状态。为其选择更适合的分布式数据库节点。

\begin{enumerate}[resume]
	\item 最小连接数法
\end{enumerate}

	最小连接数法也很简单,其根据分布式系统中当前节点的连接数来分配节点。为客户端选择当前连接数最少的数据库节点。这种方法认为,当前连接数最少的服务器节点,其当前的处理能力最强,所以应该分配给客户端。但是这种方法没有考虑到分布式系统中不同节点的性能差异。所以也不是很好的算法。

\begin{enumerate}[resume]
	\item 随机法
\end{enumerate}

	随机法很简单,其利用数学上的概率知识,随机的从分布式节点中选择一个节点。从概率上来说。这样每个分布式节点被分配的几率是一样的。经常系统的运行,每个分布式节点的连接数应该是差不多的。但是这样的算法,同样是没有考虑系统当前的机器负载状态和性能差异。所以也不是很好的算法。
\begin{enumerate}[resume]
	\item 源地址哈希法
\end{enumerate}

	这种算法的想法是,根据客户端的一种信息,一般是网络地址。然后计算哈希值。然后和分布式数据库中节点个数取模。就能得到一个值,然后根据这个值。选择分布式节点。这样的方法。首先需要提前排序好分布式数据库节点顺序,所以当其中一台计算机节点失败的时候,就可能需要重新计算所有的哈希值。影响之前的分配。所以这样的方式也不是好的分配算法。
	
	好的负载均衡算法影响是动态的负载均衡算法,其能考虑到不同的分布式节点的性能差异,同时能考虑到当前每个节点的负载信息。这样就能正确的分配的分布式节点。本章后面提出了一种新的分布式负载均衡算法。其不但能够满足这里提出的要求,而且在分布式上运行,能避免单点故障。
\subsection{数据复制和一致性}
\subsubsection{复制的概述}
复制是分布式数据库系统中保证数据安全的唯一手段。
在分布式数据库系统中,
复制可以分为同步复制和异步复制,
两者的区别在于前者需要等待所有分布式节点中副本返回写入确认,
而后者只需要一个返回确认即可。在分布式数据库系统中,复制是保证数据安全的手段,但是在分布式系统中,不同你的复制副本需要保证一致性,不然数据就是不可用的。在分布式数据库系统中。有几种复制方式来保证数据的一致性。

第一种保证副本数据一致性的方式叫主从复制。主从之间可以是异步复制,也可以是同步复制。
例如MySQL系统中,在默认情况下采用异步复制,但是异步复制容易引起数据丢失的问题
,比如主从结构中,当主节点的写入请求还没有复制到从节点的时候,主节点就就挂掉了
,这样当从节点被选为新的主节点之后,主节点在这之前还没有同步的数据就
会被丢失。即便采用了同步复制,数据也不能保证安全
,比如,当主接收写入请求然后发到从节点,从节点写入成
功后并发送确认给主,但是这个时候,主节点正准备发送确
认信息给客户端时主节点挂了,那么客户
端就会认为提交失败,可是这个时候从节点已经提交成功了,如果这个时候从
节点被提升为主,那么就出现问题了,数据不一致了。
在主从复制结构里,异步复制有更好的性能。同步复制则相对更加安全,所以分布式数据库系统中,往往结合这两种复制方式。

除了主从复制,另一种保证数据一致性的复制方式口水引入分区一致性算法Paxos,
又叫复制状态机。复制状态机在分布式数据
库开发的很多例子都可以遇到,比如,
Google Spanner在单分区内就采用了这样的设计。
复制状态机主要用于满足复制的两点需求
,第一,客户端在面对任何一个副本时都需要具备完全一致的访问行为;
第二,每个副本在执行请求时都需要按照完全一致的顺序来进行。
\subsubsection{一致性和可用性}
在分布式数据库理论中CAP定理,又被称作布鲁尔定理,有特别重要的作用,
它指出对于一个分布式数据库系统来说,不可能同时满足以下三点:
\begin{enumerate}
	\item 一致性
	\item 可用性
	\item 容忍网络分区
\end{enumerate}
根据定理,分布式系统最多只能满足三项中的两项而不能满足全部三项。
在实现分布式数据库系统的时候,一般都要选择实现的功能特性。
因为不可能一个系统满足三个特性。比如,在很多非关系型数据库系统中,抛弃了强一致性,而选择了可用性和分区容忍性。
在关系型数据库中,则必须满足强一致性特性,所以就只能满足可用性和分区容忍性。这也是为什么关系型数据库不能很好的在分布式系统部署的原因。
在分布式数据库系统中,一般选择了减少强一致性要求来实现可用性和分区容忍性。从而使得数据库在分布式环境下能部署。
\section{算法和技术}
\subsection{分布式负载均衡算法}
负载均衡是按照一种的负载均衡算法,分发网络和计算机处理资源。提供一种均衡利用分布式系统资源的方式。提高系统的数据处理能力。避免单点故障,避免资源的浪费。

负载均衡算法包括静态负载平衡以及动态负载平衡。只是根据一定的静态信息来实现的算法就是静态负载均衡算法;考虑每个分布式节点的处理能力和每个节点当前的负载信息的算法,是一种动态的负载均衡算法。动态算法能够根据系统当前的状态,选择更加合适的分布式节点,动态均衡技术又可以分为集中式方法和分布式方法,在集中方法中,单个节点
负责管理内部整个节点的负载状态信息;
分布式方法中,每个节点,
通过收集独立构建自己的负载状态信息,然后把负载信息分享给其他节点。

论文提出一种新的分布式负载均衡算法,其为后面系统负载均衡功能的实现提供了详细
的理论和架构指导。下面对分布式负载均衡算法的步骤进行详细的说明。

第一步,分布式数据库集群中的节点根据一定的方法选出一个节点作用分布式管理节点。管理节点除了能存储数据,响应用户的数据操作请求以外,还能作为分布式管理节点。分布式管理节点的作用就是为数据库客户端选择正确的后端数据库服务器,作为负载均衡功能的总代理。

第二步,分布式数据库集群中
每个服务器节点根据公式\ref{fuzai2}得到自己的权重值P,然后发送到分布式管理节点存储。其中
$ L_c $为服务器节点CPU的个数, $ L_r $为当前服务器总内存。
\begin{equation}
P=L_c * L_r \label{fuzai2}
\end{equation}

第三步,分布式数据库节点每过1秒就根据公式\ref{fuzai3}得到自己当前的负载S,同时发送到分布式管理节点。其中$ C $为当前CPU使用率, $ R $为当前服务器内存剩余量,单位为兆。
\begin{equation}
S =
\begin{cases}
1 & \text{if } C >0.9 \: \text{or} \: R<300,\\
0 & \text{if } \text{other}.
\end{cases}   \label{fuzai3}
\end{equation}

第四步,分布式管理节点根据根据每个服务器节点的权重排序,然后使其排列成一个圆形。最后一个服务器下一个服务器为第一个服务器,保存在数据库的元数据中。

第五步,
遍历数据库的元数据库,依次检查其中的每个服务器节点,检查其当前的负载S。
如果$ S=0 $就返回当前服务器的地址,算法结束,下一次遍历的时候从下一个服务器节点开始检查;如果$ S=1 $就跳过这个节点,再检查下一个节点,如此循环,直到算法正确结束或者返回给客户端错误消息。

图\ref{pic2/fuzai}示出了负载均衡算法的一个例子。首先,我们给出4个分布式服务器节点$ VM1 $到VM4。根据每个节点的CPU和内存情况,计算出其权值,然后排序出一个圆。
第一次,我们选择权值最大的节点VM1,检查其当前负载,并不是1,所以返回服务器VM1的地址给客户端,算法结束。第二次算法开始的时候,从节点VM2开始检查,返回VM2的地址,算法结束。依次类推。
直到找到一个服务器地址为止。该算法考虑了基于CPU,RAM的每个服务器的基本负载。
时间复杂度为线性的,复杂度较低。采用的分布式的负载均衡算法,同时也避免了单点故障问题,
为后面的负载均衡功能实现提供了算法理论基础。
\pic[htbp]{负载均衡示意图}{}{pic2/fuzai}
\subsection{副本和分布式MVCC技术}
\subsubsection{副本的概念}
副本指在分布式数据库系统中为数据提供的冗余副本。对于数据副本指在不同的节
点上持久化同一份数据,当出现某一个节点的存储的数据丢失时,可以从其他副本上读到数据。数据副
本是分布式系统解决数据丢失异常的唯一手段。
\subsubsection{副本一致性}
分布式数据库系统通过副本控制协议,使得数据库用户在任何情况下操作和读取数据副本,都能保持每个副本数据之间保持相同,称之为副本一致性。副本一致性是针对分布式数据库系统而言的,不是针对某一个
副本而言。
分布式数据库系统为了提高可用性,一定会使用副本的机制,就会引发副本一致性的问题。
\subsubsection{分布式 MVCC}
分布式 MVCC技术不但能解决分布式副本一致性问题,还能实现分布式事务。
假设在一个分布式数据库系统中,数据更新操作以
单个事务进行,每个事务包括若干个对不同节点的更新操作。更新事务必须具有原子性,即事务中
的所有更新操作要么同时在各个节点生效,要么都不生效。假设不存在并发的事务,即上一个事务
成功提交后才进行下一个事务。

基于 MVCC 的分布式事务一致性解决的方法是:为每个分布式事务分配一个递增的事务编号,这个编号同时也代表了
数据的版本号。当事务在各个节点上执行时,各个节点还需记录更新操作及事务编号,当事务在各
个节点都完成后,在全局元信息中记录最近事务的编号。在读取数据时,先读取元信息中已成功的
最大事务编号,只读取更新操作编号小于等于最后最大已成功提交事务
编号的操作,并将这些操作应用到基础数据形成读取结果。在删除数据的时候,并不真的删除数据,而只是做一个标记,记录当前删除事务的事务编号。
读取数据的时候,当检查下删除编辑,就跳过这个记录。
\section{本章小结}
本节介绍数据库有关的理论知识。包括硬件知识和数据库事务相关理论,以及分布式系统系统有关
的算法和协议。
另外,本章最后讨论了目前几种成熟的分布式的系统,作为实例学习和学习研究。